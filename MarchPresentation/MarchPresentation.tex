\documentclass[10pt,pdf,utf8,russian,aspectratio=169]{beamer}
\usepackage[T2A]{fontenc}
\usetheme{Montpellier}
\usepackage{setspace}
\usepackage{amsmath}
\usepackage{pgfplots}
\usepackage[utf8]{inputenc}
\usepackage{tikz-cd}
\usepackage[all, 2cell]{xy}
\usepackage{amssymb}
\usepackage{verbatim}
\usepackage[all]{xy}
\usepackage{tikz}
\usepackage{bussproofs}
\usepackage{dsfont}
\usepackage{mathabx}
\usepackage{animate}
\usetikzlibrary{graphs}
\usetikzlibrary{arrows}
\usepackage{hyperref}
\usepackage[english,russian]{babel}
\usepackage{listings}
\usepackage{color}
\usepackage{tikz}
\usepackage{listings}
\newtheorem{defin}{Определение}
\newtheorem{theor}{Теорема}
\newtheorem{lem}{Лемма}
\title{Категорная модель модального лямбда-исчисления, основанного на интуиционистской логике.}
\institute{МГУ}
\author{Даня Рогозин}
\date{Март, 2018}
\begin{document}

\maketitle

\begin{frame}
  \frametitle{Мотивация. Функциональное программирование на языке Haskell.}
  \begin{itemize}
    \item Обратимся в рамках мотивации к функциональному программированию на таких языках, как
      Haskell, Purescript, Elm или Idris;
    \item Без ограничения общности разделим типы в языке Haskell (или в любом другом из языков выше) на две части:
    простые типы и параметризованные;
    \item Простые типы (\verb"Int", \verb"String", \verb"Char", etc) -- это привычные типы данных;
    \item Параметризованные типы (\verb"List Int", \verb"Maybe Char", \verb"IO String") используются для
    вычислений в рамках оговоренного вычислительного контекста;
    \item Аналогично можно и разделить функции.
  \end{itemize}
\end{frame}

\begin{frame}
\frametitle{Мотивация. Функтор.}
Класс типов \verb"Functor" -- это общий интерфейс для ``выполнения действия над параметризованным типом, обобщение функции \verb"map" на списках'':

\vspace{\baselineskip}

\verb"class  Functor f  where"

\quad\quad \verb"fmap :: (a -> b) -> f a -> f b".

\end{frame}

\begin{frame}
\frametitle{Motivation. Monad.}
Согласно Hackage: ``С точки зрения хаскеллиста лучше всего определять монаду как тип данных для произвольных действий''.
В частности, вычисления в мире ввода-вывода -- частный случай монадических вычислений.

\vspace{\baselineskip}

(Старое) определение монады:

\verb"class  Functor m => Monad m  where"

\quad\quad \verb"return : a -> m a"

\quad\quad \verb"(>>=) :: m a -> (a -> m b) -> m b".

\vspace{\baselineskip}

Монадическая композиция (композиция действий):

\verb"(>=>) :: Monad m => (a -> m b) -> (b -> m c) -> a -> m c"

\end{frame}

\begin{frame}
\frametitle{Конечная цель: аппликативные функторы.}


Аппликативные функторы сильнее функторов и слабее монад:

\vspace{\baselineskip}

\verb"class Functor f => Applicative f where"

\quad\quad \verb"pure :: a -> f a"

\quad\quad \verb"(<*>) :: f (a -> b) -> f a -> f b"

\vspace{\baselineskip}

Используя аппликативный функтор, мы можем вложить значение в вычислительный контекст \verb"f" с помощью \verb"pure" и выполнить
аппликацию внутри \verb"f" применением \verb"<*>".

\vspace{\baselineskip}

Использование:
\begin{itemize}
\item Обобщение \verb"fmap" для функции произвольной арности:

\verb"pure f <*> a1 <*> ... <*> an"
\item Парсинг;
\item Монада в современном Haskell является наследником аппликатива.
\end{itemize}
\end{frame}

\begin{frame}
\frametitle{Монадические вычисления в теории.}

1) \emph{Eugenio Moggi.} ``Notions of computation and monads.'' Inf. Comput., 93(1): 55--92, 1991.

\vspace{\baselineskip}

2) \emph{Frank Pfenning and Rowan Davies.}  ``A judgmental reconstruction of modal logic.'' Mathematical. Structures in Comp. Sci. 11, 4 (August 2001), 511---540.

\vspace{\baselineskip}

3) \emph{Bierman, G., and De Paiva, V.}. On an Intuitionistic Modal Logic. Studia Logica: An International Journal for Symbolic Logic, 65(3), 2000. 383--416.

etc...
\end{frame}

\begin{frame}
\frametitle{Аппликативные функторы.}

К сожалению, аппликативный функтор является далеко не самой известной концепцией за вне сообщества хаскеллистов.
Возможная причина: аппликативные функторы рассмотрены с программистской точки зрения, без теоретического рассмотрения,
то есть теоретико-доказательного построения синтаксиса и алгебраической (категорной) модели.

\vspace{\baselineskip}

Пример нескольких работ:

\vspace{\baselineskip}

1) \emph{Conor McBride and Ross Paterson.} ``Applicative Programming with Effects.'' Journal of Functional
Programming 18:1 (2008), pages 1--13.

2) \emph{Ross Paterson.} ``Constructing Applicative Functors.''  Mathematics of Program Construction, Madrid,
2012, Lecture Notes in Computer Science vol. 7342, pp. 300--323, Springer, 2012.

\vspace{\baselineskip}

Белое пятно: стоит рассмотреть модальное лямбда-исчисление, которые могло бы
аксиоматизировать вычисления с аппликативным функтором и имело хорошую алгебраическую модель.

\end{frame}

\begin{frame}
\frametitle{Интуиционистская эпистемическая логика IEL$^{-}$.}


Данную проблему удобно решать, если мы располагает некоторой конструктивной модальной логикой с хорошими аксиомами,
по которой мы можем построить интересное нам модальное лямбда-исчисление:

\vspace{\baselineskip}

\begin{defin} Интуиционистская эпистемическая логика IEL$^{-}$:

  1) Аксиомы IPC;

  2) $\textbf{K}(A \to B) \to (\textbf{K}A \to \textbf{K}B)$ (нормальность);

  3) $A \to \textbf{K}A$ (ко-рефлексия);

  Правило: MP.

\end{defin}

\vspace{\baselineskip}

1) \emph{Artemov S., Protopopescu T.} (2014, June). Intuitionistic epistemic logic. ArXiv, math.LO 1406.1582v1.

2) \emph{Krupski V. N., Alexey Y.} ``Sequent calculus for intuitionistic epistemic logic IEL'' // Logical
Foundations of Computer Science -- Vol. 9537 of Lecture Notes in Computer Science. -- Springer, 2016. -- P. 187–201.

\end{frame}

\begin{frame}
  \frametitle{Натуральный вывод для IEL$^{-}$.}

  \begin{defin} Hатуральное исчисление NIEL$^{-}$ для интуиционистской эпистемической логики IEL$^{-}$ -- это
  расширение натурального исчисления для интуиционистской логики высказываний с добавлением следующих правил вывода для модальности:


    \begin{prooftree}
      \AxiomC{$\Gamma \vdash A$}
      \RightLabel{${\bf K}_I$}
      \UnaryInfC{$\Gamma \vdash {\bf K}A$}
  \end{prooftree}

    \begin{prooftree}
    \AxiomC{$\Gamma \vdash {\bf K} A_1, \dots, \Gamma \vdash {\bf K} A_n $}
    \AxiomC{$A_1,\dots,A_n \vdash B$}
    \BinaryInfC{$\Gamma \vdash {\bf K} B$}
    \end{prooftree}
  \end{defin}

\end{frame}

\begin{frame}
  \frametitle{Натуральный вывод для IEL$^{-}$.}

  \begin{lem}
    $\Gamma \vdash_{\text{NIEL}^{-}} A \Rightarrow$ IEL$^{-} \vdash \bigwedge \Gamma \rightarrow A$.
  \end{lem}

  \begin{proof}
  Индукция по построению вывода. Рассмотрим модальные случаи.

  \vspace{\baselineskip}

  1) Если $\Gamma \vdash_{\text{NIEL}^{-}} A$, тогда $\text{IEL}^{-} \vdash \bigwedge \Gamma \rightarrow {\bf K}A$.

  $\begin{array}{lll}
  (1) & \bigwedge \Gamma \rightarrow A & \text{предположение индукции}\\
  (2) & A \rightarrow {\bf K}A &\text{ко-рефлексия}\\
  (3) & (\bigwedge \Gamma \rightarrow A) \rightarrow ((A \rightarrow {\bf K}A) \rightarrow (\bigwedge \Gamma \rightarrow {\bf K}A))&\text{теорема IPC}\\
  (4) & (A \rightarrow {\bf K}A) \rightarrow (\bigwedge \Gamma \rightarrow {\bf K}A) &\text{из (1), (3) и MP}\\
  (5) & \bigwedge \Gamma \rightarrow {\bf K}A &\text{из (2), (4) и MP}\\
  \end{array}$

  \end{proof}

\end{frame}

\begin{frame}
  \frametitle{Натуральный вывод для IEL$^{-}$.}
\begin{proof}
  2) Если $\Gamma \vdash_{\text{NIEL}^{-}} {\bf K} \vec{A}$ и $\vec{A} \vdash B$, то $\text{IEL}^{-} \vdash \bigwedge \Gamma \rightarrow {\bf K}B$.

\begin{small}
  $\begin{array}{lll}
  (1) &\bigwedge \Gamma \rightarrow \bigwedge \limits_{i = 1}^{n} {\bf K} A_i & \text{предположение индукции} \\
  (2) &\bigwedge \limits_{i = 1}^{n} {\bf K} A_i \rightarrow {\bf K} \bigwedge \limits_{i = 1}^{n} A_i& \text{теорема IEL$^{-}$} \\
  (3) &\bigwedge \Gamma \rightarrow {\bf K} \bigwedge \limits_{i = 1}^{n} A_i & \text{по (1), (2) и правилу силлогизма} \\
  (4) &\bigwedge \limits_{i = 1}^{n} A_i \rightarrow B& \text{предположение индукции} \\
  (5) &(\bigwedge \limits_{i = 1}^{n} A_i \rightarrow B) \rightarrow {\bf K} (\bigwedge \limits_{i = 1}^{n} A_i \rightarrow B)& \text{ко-рефлексия}\\
  (6) &{\bf K} (\bigwedge \limits_{i = 1}^{n} A_i \rightarrow B)& \text{из (4), (5) и MP} \\
  (7) &{\bf K} \bigwedge \limits_{i = 1}^{n} A_i \rightarrow {\bf K} B & \text{по (6) и по нормальности} \\
  (8) &\bigwedge \Gamma \rightarrow {\bf K} B & \text{по (3), (7) и правилу силлогизма}
  \end{array}$
\end{small}
\end{proof}
\end{frame}

\begin{frame}
  \frametitle{Натуральный вывод для IEL$^{-}$.}

  \begin{lem}
  $ $
  Если $\text{IEL}^{-} \vdash A$, то $\text{NIEL}^{-} \vdash A$.
  \end{lem}

  \begin{proof}
  Построение выводов для модальных аксиом в $\text{NIEL}^{-}$.
  \end{proof}

\end{frame}

\begin{frame}
  \frametitle{Модальное лямбда-исчисление по IEL$^{-}$}

  \begin{defin} Модальное $\lambda$-исчисление, основанное на исчислении IEL$^{-}$:
      \begin{prooftree}
        \AxiomC{$\Gamma \vdash M : A$}
        \RightLabel{${\bf K}_I$}
        \UnaryInfC{$\Gamma \vdash {\bf pure \: } \: M : {\bf K}A $}
      \end{prooftree}

    \begin{prooftree}
      \AxiomC{$\Gamma \vdash \vec{M} : {\bf K} \vec{A}$}
      \AxiomC{$\vec{x} : \vec{A} \vdash N : B$}
      \RightLabel{$\text{let}_{{\bf K}}$}
      \BinaryInfC{$\Gamma \vdash {\bf let \: pure \:} \vec{x} = \vec{M} {\: \bf in \: } N : {\bf K} B$}
    \end{prooftree}
  \end{defin}

  $\Gamma \vdash \vec{M} : {\bf K} \vec{A}$ -- это синтаксический сахар для $\Gamma \vdash M_1 : {\bf K}A_1,\dots,\Gamma \vdash M_n : {\bf K}A_n$ и $\vec{x} : \vec{A} \vdash N : B$ -- это краткая форма для $x_1 : A_1, \dots, x_n : A_n \vdash N : B$.
  ${\bf let \: pure \:} \vec{x} = \vec{M} {\: \bf in \: } N$ -- это мгновенное локальное связывание в терме $N$.
  Мы будем использовать такую краткую форму вместо ${\bf let \: pure\:} x_1,\dots,x_n = M_1,\dots,M_n {\: \bf in \:} N$.

\end{frame}

\begin{frame}
  \frametitle{Примеры деревьев вывода}

  \begin{prooftree}
  \AxiomC{$x : A \vdash x : A$}
  \UnaryInfC{$x : A \vdash {\bf pure \:} x : {\bf K} A$}
  \UnaryInfC{$\vdash (\lambda x. {\bf pure \: } x) : A \to {\bf K}A$}
  \end{prooftree}

  \begin{prooftree}
  \AxiomC{$f : {\bf K} (A \to B) \vdash f : {\bf K} (A \to B)$}
  \AxiomC{$x : {\bf K} A \vdash x : {\bf K} A $}
  \AxiomC{$g : A \to B \vdash g : A \to B$}
  \AxiomC{$y : A \vdash y : A$}
  \RightLabel{$\to_e$}
  \BinaryInfC{$g : A \to B, y : A \vdash g y : B$}
  \RightLabel{$\text{let}_{{\bf K}}$}
  \TrinaryInfC{$f : {\bf K} (A \to B), x : {\bf K} A \vdash {\bf let \: pure \:} g, y  = f, x {\: \bf in \:} g y : {\bf K} B$}
  \UnaryInfC{$f : {\bf K} (A \to B) \vdash \lambda x. {\bf let \: pure\:} g, y = f, x {\: \bf in \:} g y : {\bf K} A \to {\bf K} B$}
  \UnaryInfC{$\vdash \lambda f. \lambda x. {\bf let \:pure \:} g, y = f, x {\: \bf in \:} g y : {\bf K}(A \to B) \to {\bf K} A \to {\bf K} B$}
  \end{prooftree}

\end{frame}

\begin{frame}
  \frametitle{Подстановка}

  \begin{defin} Подстановка:

  1) $x [x := N] = N$, $x [y := N] = x$;

  2) $(M N) [x := N] = M[x := N] N [x := N]$;

  3) $(\lambda x. M) [x := N] = \lambda x. M [y := N]$, $y \in FV(M)$;

  4) $(M, N)[x := P] = (M[x := P], N [x := P])$;

  5) $(\pi_i M) [x := P] = \pi_i (M[x := P])$, $i \in \{ 1, 2\}$;

  6) $({\bf pure \: } M) [x := P] = {\bf pure \: } (M [x := P])$;

  7) $({\bf let \: pure \:}\vec{x} = \vec{M} {\: \bf in \:} N) [y := P] = {\bf let \: pure \:} \vec{x} = (\vec{M} [y := P]) {\: \bf in \:} N$.
  \end{defin}

\end{frame}

\begin{frame}
  \frametitle{Редукция}

  \begin{defin} Правила $\beta$-редукции и $\eta$-редукции:

  1) $(\lambda x. M) N \rightarrow_{\beta} M [x := N]$;

  2) $\pi_1 \langle M, N \rangle \rightarrow_{\beta} M$;

  3) $\pi_2 \langle M, N \rangle \rightarrow_{\beta} N$;

  4) $\begin{array}{llll}
  & {\bf let \: pure \:} \vec{x}, y, \vec{z} = \vec{M}, {\bf let \: pure \: } \vec{w} = \vec{N} {\: \bf in \: } Q, \vec{P} {\: in \:} R \rightarrow_{\beta} & \\
  & {\bf let \: pure \:} \vec{x}, \vec{w}, \vec{z} = \vec{M}, \vec{N}, \vec{P} {\: \bf in \: } R [y := Q]
  \end{array}$

  5) ${\bf let \: pure \:} \vec{x} = {\bf pure \:} \vec{M} {\: \bf in \:} N \rightarrow_{\beta} {\bf pure \:} N [\vec{x} := \vec{M}]$

  6) ${\bf let \: pure \:} \underline{\quad} = \underline{\quad} {\: \bf in \:} M \rightarrow_{\beta} {\bf pure \:} M$, где \underline{\quad} -- это пустая последовательность термов.

  7) $\lambda x. f x \rightarrow_{\eta} f$;

  8) $\langle \pi_1 P, \pi_2 P \rangle \rightarrow_{\eta} P$;

  9) ${\bf let \: pure \:} x = M {\: \bf in \: } x \rightarrow_{\eta} M$;

  \end{defin}

\end{frame}

\begin{frame}
  \frametitle{Метатеоретические свойства системы}

  \begin{theor} Редукция субъекта

  Если $\Gamma \vdash M : A$ и $M \twoheadrightarrow_{\beta \eta} N$, тогда $\Gamma \vdash N : A$

  \end{theor}

  \begin{theor}
  Отношение $\twoheadrightarrow_{\beta}$ сильно нормализуемо;
  \end{theor}

  \begin{theor}
  Отношение $\twoheadrightarrow_{\beta}$ конфлюентно.
  \end{theor}

  \begin{theor}
  Нормальная форма $\lambda_{{\bf K}}$ со стратегией вычисления с вызовом по имени обладает свойством подформульности: если $M$ в нормальной форме, то всего его подтермы также в нормальной форме.

  \end{theor}


\end{frame}

\begin{frame}
  \frametitle{Категорная модель. Теоретико-категорные прелиминарии.}

  \begin{defin}

    Моноидальная категория -- это категория $\mathcal{C}$ с дополнительной структурой:
    \begin{itemize}
      \item Бифунктор $\otimes : \mathcal{C} \times \mathcal{C} \to C$, который мы будем называть тензором;
      \item Единица $\mathds{1}$;
      \item Изоморфизм, который мы будем называть ассоциатором: $\alpha_{A,B,C}: (A \otimes B) \otimes C \cong A \otimes (B \otimes C)$;
      \item Изоморфизм  $L_A : \mathds{1} \otimes A \cong A$;
      \item Изоморфизм $R_A : A \otimes \mathds{1} \cong A$;
      \item Первое условие когерентности (пятиугольник Маклейна);
      \item Второе условие когерентности (тождество треугольника).
    \end{itemize}
      \end{defin}

Легко видеть, что декартово замкнутая категория -- это частный случай (симметрической) моноидальной категории, в котором тензор -- это произведения, а единица -- это терминальный объект.


\end{frame}

\begin{frame}
  \frametitle{Категорная модель. Теоретико-категорные прелиминарии. Пятиугольник Маклейна.}

  \xymatrix{
    & (A \otimes (B \otimes C)) \otimes D \ar[dr]^{\alpha_{A,B \otimes C,D}}\\
    ((A \otimes B) \otimes C) \otimes D \ar[d]_{\alpha_{A \otimes B, C, D}} \ar[ur]^{\alpha_{A,B,C} \otimes id_D \quad} && A \otimes ((B \otimes C) \otimes D) \ar[d]^{id_A \otimes \alpha_{B,C,D}}\\
    (A \otimes B) \otimes (C \otimes D) \ar[rr]_{\alpha_{A,B,C \otimes D}}&& A \otimes (B \otimes (C \otimes D))
  }

\end{frame}

\begin{frame}
  \frametitle{Категорная модель. Теоретико-категорные прелиминарии. Тождество треугольника.}

  \xymatrix{
  && (A \otimes \mathds{1}) \otimes B \ar[rr]^{\alpha_{A, \mathds{1}, B}} \ar[dr]_{R_A \otimes id_B} && A \otimes (\mathds{1} \otimes B) \ar[dl]^{id_A \otimes L_B} \\
  &&& A \otimes B
  }

\end{frame}

\begin{frame}
  \frametitle{Категорная модель. Теоретико-категорные прелиминарии.}

  \begin{defin} Нестрогий моноидальный функтор

    Пусть $\langle \mathcal{C}, \otimes_1, \mathds{1}_{\mathcal{C}} \rangle$ и $\langle \mathcal{D}, \otimes_2, \mathds{1}_{\mathcal{D}} \rangle$ моноидальные категории.

    Нестрогий моноидальный функтор $\mathcal{F} : \langle \mathcal{C}, \otimes_1, \mathds{1} \rangle \to \langle \mathcal{D}, \otimes_2, \mathds{1}' \rangle$ это функтор
    $\mathcal{F} : \mathcal{C} \to \mathcal{D}$ с дополнительными естественными преобразованиями:

    \begin{itemize}
    \item $u : \mathds{1}_{\mathcal{D}} \to \mathcal{F}\mathds{1}_{\mathcal{C}}$;
    \item $\ast_{A, B} : \mathcal{F}A \otimes_{\mathcal{D}} \mathcal{F}B \to \mathcal{F}(A \otimes_{\mathcal{C}} B)$;
    \item Три условия когеретности: ассоциативность, свойство левой и правой единиц.
    \end{itemize}
  \end{defin}
\end{frame}

\begin{frame}
  \frametitle{Категорная модель. Теоретико-категорные прелиминарии. Accoциативность.}

  \xymatrix{
    && (\mathcal{F}A \otimes_{\mathcal{D}} \mathcal{F}B) \otimes_{\mathcal{D}} \mathcal{F}C \ar[d]_{\ast_{A,B} \otimes_{\mathcal{D}} id_{\mathcal{F}B}}
    \ar[rr]^{\alpha^{\mathcal{D}}_{{\mathcal{F}A, \mathcal{F}B, \mathcal{F}C}}} && \mathcal{F}A \otimes_{\mathcal{D}} (\mathcal{F}B \otimes_{\mathcal{D}} \mathcal{F}C) \ar[d]^{id_{\mathcal{F}A} \otimes_{\mathcal{D}} \ast_{B,C}}\\
    && \mathcal{F}(A \otimes_{\mathcal{C}} B) \otimes_{\mathcal{D}} \mathcal{C} \ar[d]_{\ast_{A \otimes_{\mathcal{C}} B, C}} && \mathcal{F}A \otimes_{\mathcal{D}} \mathcal{F}(B \otimes_{\mathcal{C}} C) \ar[d]^{\ast_{A, B \otimes_{\mathcal{C}} C}}\\
    && \mathcal{F}((A \otimes_{\mathcal{C}} B) \otimes_{\mathcal{C}} C) \ar[rr]_{\mathcal{F}(\alpha^{\mathcal{C}}_{A,B,C})}&& \mathcal{F}(A \otimes_{\mathcal{C}} (B \otimes_{\mathcal{C}} C))
  }

\end{frame}

\begin{frame}
  \frametitle{Категорная модель. Теоретико-категорные прелиминарии. Свойство левой единицы.}

  \xymatrix{
  &&  \mathds{1}_{\mathcal{D}} \otimes_{\mathcal{D}} \mathcal{F}A \ar[d]_{L^{\mathcal{D}}_{\mathcal{F}A}}\ar[rr]^{u \otimes_{\mathcal{D}} id_{\mathcal{F}A}} && \mathcal{F}\mathds{1}_{\mathcal{C}} \otimes_{\mathcal{D}} \mathcal{F}A \ar[d]^{\ast_{\mathds{1}_{\mathcal{C}}, A}} \\
  &&  \mathcal{F}A && \mathcal{F}(\mathds{1}_{\mathcal{C}} \otimes_{\mathcal{C}} A) \ar[ll]_{\mathcal{F}(L^{\mathcal{C}}_A)}
  }

\end{frame}

\begin{frame}
  \frametitle{Категорная модель. Теоретико-категорные прелиминарии.}

  \begin{defin}

  Тензорно-сильный функтор -- это эндофунктор над моноидальной категорией с дополнительным естественным преобразованием:

  \begin{center}
  $\begin{array}{lll}
    \tau_{A, B} : A \otimes \mathcal{K}B \rightarrow \mathcal{K}(A \otimes B)
  \end{array}$
  \end{center}

  \xymatrix{
  (A \otimes B) \otimes \mathcal{K}C \ar[d]_{\alpha_{A, B, \mathcal{K}C}}\ar[rrrr]^{\tau_{A \otimes B, C}} &&&& \mathcal{K}((A \otimes B) \otimes C) \ar[d]^{\mathcal{K}(\alpha_{A,B,C})}\\
  A \otimes (B \otimes \mathcal{K}C) \ar[rr]_{id_A \otimes \tau_{B,C}} && A \otimes \mathcal{K}(B \otimes C) \ar[rr]_{\quad \tau_{A, (B \otimes C)}} && \mathcal{K}(A \otimes (B \otimes C)) \\
  }
  \xymatrix{
  &&& \mathds{1} \otimes \mathcal{K}A \ar[drr]_{R_{\mathcal{K}A}} \ar[rr]^{\mu_{\mathds{1}, A}} && \mathcal{K}(\mathds{1} \otimes A) \ar[d]^{\mathcal{K}(R_A)}\\
  &&&&& \mathcal{K}A
  }
  \end{defin}
\end{frame}

\begin{frame}
  \frametitle{Определение аппликативного функтора.}

  \begin{defin}
    Аппликативный функтор -- это тройка $\langle \mathcal{C}, \mathcal{K}, \eta \rangle$,
  где $\mathcal{C}$ -- это моноидальная категория, $\mathcal{K}$ - это тензорно-сильный нестрогий моноидальный эндофунктор и
  $\eta : Id_{\mathcal{C}} \Rightarrow \mathcal{K}$ -- это естественное преобразование, такое, что:

  (сейчас будут условия когерентности для $\eta$)
  \end{defin}
\end{frame}

\begin{frame}
  \frametitle{Условия когерентности для $\eta$}

  \begin{itemize}
  \item $u = \eta_{\mathds{1}}$;
  \item $\ast_{A,B} \circ (\eta_A \otimes \eta_B) = \eta_{A \otimes B}$:

  \xymatrix
  {
  &&& A \otimes B \ar[rr]^{\eta_A \otimes \eta_B} \ar[drr]_{\eta_{A \otimes B}} && \mathcal{K}A \otimes \mathcal{K}B \ar[d]^{\ast_{A,B}} \\
  &&&&& \mathcal{K}(A \otimes B)
  }

  \item $\tau_{A, B} = \ast_{A, B} \circ \eta_{A} \otimes id_{\mathcal{K}B}$;

  \item $\tau_{A, B} \circ id_A \otimes \eta_B = \eta_{A \otimes B}$:

  \xymatrix
  {
  &&& A \otimes B \ar[drr]_{\eta_{A \otimes B}} \ar[rr]^{id_A \otimes \eta_B} && A \otimes \mathcal{K}B \ar[d]^{\tau_{A,B}} \\
  &&&&& \mathcal{K}(A \otimes B)
  }
  \end{itemize}
\end{frame}

\begin{frame}
  \frametitle{Теоретико-категорная семантика.}

  \begin{theor} Корректность
Пусть $\Gamma \vdash M : A$ и $M =_{\beta\eta} N$, тогда $[\![\Gamma \vdash M : A]\!] = [\![\Gamma \vdash N : A]\!]$
  \end{theor}

  Интерпретация модальных правил:

  \begin{prooftree}
  \AxiomC{$[\![\Gamma \vdash M : A]\!] = [\![M]\!] : [\![\Gamma]\!] \rightarrow [\![A]\!]$}
  \UnaryInfC{$[\![\Gamma \vdash {\bf pure \:} M : {\bf K}A]\!] := [\![\Gamma]\!] \xrightarrow{[\![M]\!]}
  [\![A]\!] \xrightarrow{\eta_{[\![A]\!]}} \mathcal{K}[\![A]\!]$}
  \end{prooftree}

\begin{small}
  \begin{prooftree}
    \AxiomC{$[\![\Gamma \vdash \vec{M} : {\bf K} \vec{A}]\!] = \langle [\![M_1]\!],\dots, [\![M_n]\!] \rangle : [\![\Gamma]\!] \rightarrow \prod \limits_{i=1}^n \mathcal{K}[\![A_i]\!]$}
    \AxiomC{$[\![\vec{x} : \vec{A} \vdash N : B]\!] = [\![N]\!] : \prod \limits_{i=1}^n [\![A_i]\!] \rightarrow [\![B]\!]$}
    \BinaryInfC{$[\![\Gamma \vdash {\bf let \: pure \:} \vec{x} = \vec{M} {\: \bf in \: } M : {\bf K} B]\!] = \mathcal{K}([\![N]\!]) \circ \ast_{[\![A_1]\!],\dots,[\![A_n]\!]} \circ \langle [\![M_1]\!],\dots, [\![M_n]\!] \rangle : [\![\Gamma]\!] \rightarrow \mathcal{K}[\![B]\!]$}
  \end{prooftree}
\end{small}
\end{frame}

\begin{frame}
  \frametitle{Теоретико-категорная семантика.}

  \begin{lem} Интерпретация сохраняет подстановку.

  $[\![M [x_1 := M_1,\dots, x_n := M_n]]\!] = [\![M]\!] \circ \langle [\![M_1]\!], \dots, [\![M_n]\!] \rangle$.
  \end{lem}

  \begin{lem} Интерпретация сохраняет редукцию.

      Пусть $\Gamma \vdash M : A$ и $M \twoheadrightarrow_{\beta \eta} N$, тогда $[\![\Gamma \vdash M : A]\!] = [\![\Gamma \vdash N : A]\!]$;
  \end{lem}
\end{frame}

\begin{frame}
  \frametitle{Теоретико-категорная семантика. Пример.}

  1) $[\![\Gamma \vdash {\bf let \: pure \:} \vec{x} = {\bf pure \:} \vec{M} {\: \bf in \:} N : {\bf K}B]\!] = [\![\Gamma \vdash {\bf pure \:} N [\vec{x} := \vec{M}] : {\bf K}B]\!]$

  \vspace{\baselineskip}

  $\begin{array}{lll}
  &[\![\Gamma \vdash {\bf let \: pure \:} \vec{x} = {\bf pure \:} \vec{M} {\: \bf in \:} N : {\bf K}B]\!] = & \\
  &\text{\quad\quad\quad\quad\quad\quad интепретация}& \\
  &\mathcal{K}([\![N]\!]) \circ \ast_{[\![A_1]\!],\dots,[\![A_n]\!]} \circ \langle \eta_{[\![A_1]\!]} \circ [\![M_1]\!],\dots,\eta_{[\![A_n]\!]} \circ [\![M_n]\!] \rangle = \\
  &\text{\quad\quad\quad\quad\quad\quad свойство пары морфизмов}& \\
  &\mathcal{K}([\![N]\!]) \circ \ast_{[\![A_1]\!],\dots,[\![A_n]\!]} \circ (\eta_{[\![A_1]\!]} \times \dots \times \eta_{[\![A_n]\!]}) \circ \langle [\![M_1]\!], \dots, [\![M_n]\!]\rangle =& \\
  &\text{\quad\quad\quad\quad\quad\quad по определению аппликативного функтора}& \\
  &\mathcal{K}([\![N]\!]) \circ \eta_{[\![A_1]\!] \times \dots \times [\![A_n]\!]} \circ \langle [\![M_1]\!], \dots, [\![M_n]\!] \rangle =& \\
  &\text{\quad\quad\quad\quad\quad\quad естественность $\eta$}& \\
  &\eta_{[\![B]\!]} \circ [\![N]\!] \circ \langle [\![M_1]\!], \dots, [\![M_n]\!] \rangle =& \\
  &\text{\quad\quad\quad\quad\quad\quad по лемме об одновременной подстановке}& \\
  &\eta_{[\![B]\!]} \circ [\![N [\vec{x} := \vec{M}]]\!] = & \\
  &\text{\quad\quad\quad\quad\quad\quad интерпретация}& \\
  &[\![\Gamma \vdash {\bf pure \:} (N [\vec{x} := \vec{M}]) : {\bf K}B]\!]&
  \end{array}$

\end{frame}


\begin{frame}
  \frametitle{Теоретико-категорная семантика. Полнота.}

  \begin{defin} Эквивалетность на парах вида переменная-терм:
    $ $

    Определим такое бинарное отношение  $\sim_{A, B} \subseteq \mathbb{V} \times \Lambda_{{\bf K}}$, что:

    $(x, M) \sim_{A, B} (y, N) \Leftrightarrow x : A \vdash M : B \:\: \& \:\: y : A \vdash N : A \:\: \& \:\: M =_{\beta \eta} N [y := x]$.
  \end{defin}

Обозначим класс эквивалентности как $[x, M]_{A, B} = \{ (y, N) \: | \: (x, M) \sim_{A, B} (y, N) \}$ (ниже мы будем опускать индексы).

\end{frame}

\begin{frame}
  \frametitle{Теоретико-категорная семантика. Полнота.}

  \begin{defin} Категория $\mathcal{C}(\lambda)$:
  \begin{itemize}
    \item $Ob_{\mathcal{C}} = \{ \hat{A} \: | \: A \in \mathbb{T} \} \cup \{ \mathds{1} \}$;
    \item $Hom_{\mathcal{C}(\lambda)}(\hat{A},\hat{B}) = (\mathbb{V} \times \Lambda_{{\bf K}})/_{\sim_{A, B}}$;
    \item Пусть $[x, M] \in Hom_{\mathcal{C}(\lambda)}(\hat{A},\hat{B})$ и $[y,N] \in Hom_{\mathcal{C}(\lambda)}(\hat{B},\hat{C})$, тогда $[y,M] \circ [x, M] = [x, N [y := M]]$;
    \item Тождественный морфизм $id_{\hat{A}} = [x,x] \in Hom_{\mathcal{C}(\lambda)(\hat{A})}$;
    \item Терминальный объект $\mathds{1}$;
    \item $\widehat{A \times B} = \hat{A} \times \hat{B}$;
    \item Каноническая проекция: $[x, \pi_i x] \in Hom_{\mathcal{C}(\lambda)}(\hat{A_1} \times \hat{A_2},\hat{A_i})$ for $i \in \{ 1, 2 \}$;
    \item $\widehat{A \to B} = \hat{B}^{\hat{A}}$;
    \item Вычисляющая стрелка $\epsilon = [x, (\pi_2 x) (\pi_1 x)] \in Hom_{\mathcal{C}(\lambda)(\hat{B}^{\hat{A}} \times \hat{A}, \hat{B})}$.
  \end{itemize}
  \end{defin}

\end{frame}

\begin{frame}
  \frametitle{Теоретико-категорная семантика. Полнота.}

Необходимо показать, что ${\bf K}$ -- это аппликативный функтор над декартово замкнутой категорией $\mathcal{C}(\lambda)$.

\end{frame}

\begin{frame}
  \frametitle{Теоретико-категорная семантика. Полнота.}

  \begin{defin}
    Определим эндофунктор $\mathcal{K} : \mathcal{C}(\lambda) \to \mathcal{C}(\lambda)$ таким образом, что
  для любых $[x,M] \in Hom_{\mathcal{C}(\lambda)}(\hat{A},\hat{B}), {\bf K}([x,M]) = [y, {\bf let \: pure \:} x = y {\: \bf in \:} M] \in Hom_{\mathcal{C}(\lambda)}({\bf K} \hat{A}, {\bf K} \hat{B})$
  (обозначения: $\text{fmap } f$ для произвольной стрелки $f$).

  \end{defin}

  \begin{lem} Функториальность

  \begin{itemize}
    \item $\text{fmap }(g \circ f) = \text{fmap }(g) \circ \text{fmap }(f)$;
    \item $\text{fmap }(id_{\hat{A}}) = id_{{\bf K}\hat{A}}$.
  \end{itemize}
  \end{lem}

\end{frame}

\begin{frame}
  \frametitle{Теоретико-категорная семантика. Полнота.}

  \begin{defin}

    Определим естественные преобразования:

  \begin{itemize}
    \item $\eta:Id \Rightarrow \mathcal{K}$, такое, что $\forall \hat{A} \in Ob_{\mathcal{C}(\lambda)}$, $\eta_{\hat{A}} = [x, {\bf pure \:} x] \in Hom_{\mathcal{C}(\lambda)}(\hat{A}, {\bf K}\hat{A})$;
    \item $\ast_{A,B}:{\bf K}\hat{A} \times {\bf K}\hat{B} \to {\bf K}(\hat{A} \times \hat{B})$, такое, что $\forall \hat{A}, \hat{B} \in Ob_{\mathcal{C}(\lambda)}, \ast_{\hat{A},\hat{B}} = [p, {\bf let \: pure \:} x,y = \pi_1 p, \pi_2 p {\: \bf in \:} \langle x, y \rangle] \in Hom_{\mathcal{C}(\lambda)}({\bf K}A \times {\bf K}B, {\bf K}(A \times B))$.
  \end{itemize}
  \end{defin}

  \begin{lem}
    {\bf K} нестрогий моноидальный функтор.
  \end{lem}

  \begin{lem} Естественность и когерентность $\eta$:
  \begin{itemize}
    \item $\text{fmap } f \circ \eta_A = \eta_B \circ f$;
    \item $\ast_{\hat{A},\hat{B}} \circ (\eta_{A} \times \eta_{B}) = \eta_{\hat{A} \times \hat{B}}$;
    \item $\tau_{A, B} \circ id_A \times \eta_B = \eta_{\widehat{A \times B}}$.
  \end{itemize}
  \end{lem}

\end{frame}

\begin{frame}
  \frametitle{Теоретико-категорная семантика. Полнота. Пример.}

  $\begin{array}{lll}
  & \tau_{\hat{A}, \hat{B}} \circ id_{\hat{A}} \times \eta_{\hat{B}} = & \\
  &\quad\quad\quad\quad\quad\quad \text{раскрытие}& \\
  &[q, {\bf let \: pure \:} x, y = {\bf pure \:} (\pi_1 q), \pi_2 q {\: \bf in \:} \langle x, y \rangle] \circ [p, \langle \pi_1 p, {\bf pure \:} (\pi_2 p) \rangle] = & \\
  &\quad\quad\quad\quad\quad\quad \text{композиция}& \\
  &[p, {\bf let \: pure \:} x, y = {\bf pure \:} (\pi_1 q), \pi_2 q {\: \bf in \:} \langle x, y \rangle [q := \langle \pi_1 p, {\bf pure \:} (\pi_2 p) \rangle]]& \\
  &\quad\quad\quad\quad\quad\quad \text{подстановка}& \\
  &[p, {\bf let \: pure \:} x, y = {\bf pure \:} (\pi_1 p), {\bf pure \:} (\pi_2 p) {\: \bf in \:} \langle x, y \rangle ] = & \\
  &\quad\quad\quad\quad\quad\quad \text{правило редукции}& \\
  &[p, {\bf \: pure \:} (\langle x, y \rangle) [x := \pi_1 p, y := \pi_2 p]]& \\
  &\quad\quad\quad\quad\quad\quad \text{подстановка}& \\
  &[p, {\bf \: pure \:} (\langle \pi_1 p, \pi_2 p \rangle)] = & \\
  &\quad\quad\quad\quad\quad\quad \text{$\eta$-редукция}& \\
  &[p, {\bf \: pure \:} p] =& \\
  &\quad\quad\quad\quad\quad\quad \text{определение $\eta$}& \\
  &\eta_{\widehat{A \times B}}&
  \end{array}$

\end{frame}


\begin{frame}
  \frametitle{Теоретико-категорная семантика. Полнота.}

  \begin{defin}
    $u_{\mathds{1}} = [\sqbullet, {\bf let \: pure \:} \underline{\quad} = \underline{\quad} {\: \bf in \:} \sqbullet]$.
  \end{defin}

  \begin{lem}
    $u_{\mathds{1}} = \eta_{\mathds{1}}$
  \end{lem}

  \begin{defin}
    $\tau_{\hat{A}, \hat{B}} = [p, {\bf let \: pure \:} x, y = {\bf pure \:} (\pi_1 p), \pi_2 p {\: \bf in \:} \langle x, y \rangle]$
  \end{defin}

  \begin{lemma}
    \begin{itemize}
      \item $\text{fmap } \alpha_{\hat{A},\hat{B},\hat{C}} \circ \tau_{\hat{A} \times \hat{B}, \hat{C}} = \tau_{\hat{A}, \hat{B} \times \hat{C}} \circ (id_{\hat{A}} \times \tau_{\hat{B}, \hat{C}}) \circ \alpha_{\hat{A}, \hat{B}, {\bf K}\hat{C}}$;
      \item ${\bf K}(R_{\hat{A}}) \circ \tau_{\mathds{1}, \hat{A}} = R_{{\bf K}\hat{A}}$.
    \end{itemize}
    где $\alpha_{\hat{A},\hat{B},\hat{C}} = [p, \langle \pi_1 (\pi_1 p), \langle \pi_2 (\pi_1 p), \pi_2 p \rangle \rangle]$ и
    $R = \pi_2$.
  \end{lemma}

\end{frame}

\begin{frame}
  \frametitle{Теоретико-категорная семантика. Полнота.}

  Из рассмотренных выше лемм легко заключить, что {\bf K} -- аппликативный функтор.

  Положим $[\![\Gamma \vdash M : A]\!] = [x, M [x_i := \pi_i x]]$, тогда $M =_{\beta \eta} N \Leftrightarrow [\![\Gamma \vdash M : A]\!] = [\![\Gamma \vdash N : A]\!]$.

\end{frame}

\begin{frame}
  \frametitle{Спасибо за внимание!}

  Черепашка ниндзя Донателло пишет представителя класса типов \verb"Applicative":

  \animategraphics[loop,controls,width=0.5\linewidth]{10}{giphy-}{0}{3}

\end{frame}


\end{document}
