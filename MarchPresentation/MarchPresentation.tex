\documentclass[10pt,pdf,utf8,russian,aspectratio=169]{beamer}
\usepackage[T2A]{fontenc}
\usetheme{Montpellier}
\usepackage{setspace}
\usepackage{amsmath}
\usepackage{pgfplots}
\usepackage[utf8]{inputenc}
\usepackage{tikz-cd}
\usepackage[all, 2cell]{xy}
\usepackage{amssymb}
\usepackage{verbatim}
\usepackage[all]{xy}
\usepackage{tikz}
\usepackage{bussproofs}
\usepackage{dsfont}
\usepackage{mathabx}
\usepackage{animate}
\usetikzlibrary{graphs}
\usetikzlibrary{arrows}
\usepackage{hyperref}
\usepackage[english,russian]{babel}
\usepackage{listings}
\usepackage{color}
\usepackage{tikz}
\usepackage{listings}
\newtheorem{defin}{Определение}
\newtheorem{theor}{Теорема}
\newtheorem{lem}{Лемма}
\title{Категорная модель модальной теории типов, основанного на интуиционистской эпистемической логике.}
\author{Даниил Рогозин}
\date{Март, 2018}
\begin{document}

\maketitle

\begin{frame}
  \frametitle{Мотивация. Функциональное программирование на языке Haskell.}
  \begin{itemize}
    \item Обратимся в рамках мотивации к функциональному программированию на таких языках, как
      Haskell, Purescript, Elm или Idris;
    \item Без ограничения общности разделим типы в языке Haskell (или в любом другом из языков выше) на две части:
    простые типы и параметризованные;
    \item Простые типы (\verb"Int", \verb"String", \verb"Char", etc) -- это привычные типы данных;
    \item Параметризованные типы (\verb"List Int", \verb"Maybe Char", \verb"IO String") используются для
    вычислений в рамках оговоренного вычислительного контекста;
    \item Аналогично можно и разделить функции.
  \end{itemize}
\end{frame}

\begin{frame}
\frametitle{Мотивация. Функтор.}
Класс типов \verb"Functor" -- это общий интерфейс для ``выполнения действия над параметризованным типом, обобщение функции \verb"map" на списках'':

\vspace{\baselineskip}

\verb"class  Functor f  where"

\quad\quad \verb"fmap :: (a -> b) -> f a -> f b".

\end{frame}

\begin{frame}
\frametitle{Конечная цель: аппликативные функторы.}


Аппликативные функторы сильнее функторов и слабее монад:

\vspace{\baselineskip}

\verb"class Functor f => Applicative f where"

\quad\quad \verb"pure :: a -> f a"

\quad\quad \verb"(<*>) :: f (a -> b) -> f a -> f b"

\vspace{\baselineskip}

Используя аппликативный функтор, мы можем вложить значение в вычислительный контекст \verb"f" с помощью \verb"pure" и выполнить
аппликацию внутри \verb"f" применением \verb"<*>".

\vspace{\baselineskip}

Использование:
\begin{itemize}
\item Обобщение \verb"fmap" для функции произвольной арности:

\verb"liftAn = pure f <*> a1 <*> ... <*> an"
\item Парсинг;
\item Монада в современном Haskell является наследником аппликатива.
\end{itemize}
\end{frame}

\begin{frame}
\frametitle{Монадические вычисления в теории.}

1) \emph{Eugenio Moggi.} ``Notions of computation and monads.'' Inf. Comput., 93(1): 55--92, 1991.

\vspace{\baselineskip}

2) \emph{Frank Pfenning and Rowan Davies.}  ``A judgmental reconstruction of modal logic.'' Mathematical. Structures in Comp. Sci. 11, 4 (August 2001), 511---540.

\vspace{\baselineskip}

3) \emph{Bierman, G., and De Paiva, V.}. On an Intuitionistic Modal Logic. Studia Logica: An International Journal for Symbolic Logic, 65(3), 2000. 383--416.

etc...
\end{frame}

\begin{frame}
\frametitle{Аппликативные функторы.}

К сожалению, аппликативный функтор является далеко не самой известной концепцией за вне сообщества хаскеллистов.
Возможная причина: аппликативные функторы рассмотрены с программистской точки зрения, без теоретического рассмотрения,
то есть теоретико-доказательного построения синтаксиса и алгебраической (категорной) модели.

\vspace{\baselineskip}

Пример нескольких работ:

\vspace{\baselineskip}

1) \emph{Conor McBride and Ross Paterson.} ``Applicative Programming with Effects.'' Journal of Functional
Programming 18:1 (2008), pages 1--13.

2) \emph{Ross Paterson.} ``Constructing Applicative Functors.''  Mathematics of Program Construction, Madrid,
2012, Lecture Notes in Computer Science vol. 7342, pp. 300--323, Springer, 2012.

\vspace{\baselineskip}

Белое пятно: стоит рассмотреть модальное лямбда-исчисление, которые могло бы
аксиоматизировать вычисления с аппликативным функтором и имело хорошую алгебраическую модель.

\end{frame}

\begin{frame}
\frametitle{Интуиционистская эпистемическая логика IEL$^{-}$.}


Данную проблему удобно решать, если мы располагает некоторой конструктивной модальной логикой с хорошими аксиомами,
по которой мы можем построить интересное нам модальное лямбда-исчисление:

\vspace{\baselineskip}

\begin{defin} Интуиционистская эпистемическая логика IEL$^{-}$:

  1) Аксиомы IPC;

  2) $\textbf{K}(A \to B) \to (\textbf{K}A \to \textbf{K}B)$ (нормальность);

  3) $A \to \textbf{K}A$ (ко-рефлексия);

  Правило: MP.

\end{defin}

\vspace{\baselineskip}

1) \emph{Artemov S., Protopopescu T.} (2014, June). Intuitionistic epistemic logic. ArXiv, math.LO 1406.1582v1.

2) \emph{Krupski V. N., Alexey Y.} ``Sequent calculus for intuitionistic epistemic logic IEL'' // Logical
Foundations of Computer Science -- Vol. 9537 of Lecture Notes in Computer Science. -- Springer, 2016. -- P. 187–201.

\end{frame}

\begin{frame}
  \frametitle{Натуральный вывод для IEL$^{-}$.}

  \begin{defin} Hатуральное исчисление NIEL$^{-}$ для интуиционистской эпистемической логики IEL$^{-}$ -- это
  расширение натурального исчисления для интуиционистской логики высказываний с добавлением следующих правил вывода для модальности:


    \begin{prooftree}
      \AxiomC{$\Gamma \vdash A$}
      \RightLabel{${\bf K}_I$}
      \UnaryInfC{$\Gamma \vdash {\bf K}A$}
  \end{prooftree}

    \begin{prooftree}
    \AxiomC{$\Gamma \vdash {\bf K} A_1, \dots, \Gamma \vdash {\bf K} A_n $}
    \AxiomC{$A_1,\dots,A_n \vdash B$}
    \BinaryInfC{$\Gamma \vdash {\bf K} B$}
    \end{prooftree}
  \end{defin}

\end{frame}


\begin{frame}
  \frametitle{Модальное лямбда-исчисление по IEL$^{-}$}

  \begin{defin} Модальное $\lambda$-исчисление, основанное на исчислении IEL$^{-}$:
      \begin{prooftree}
        \AxiomC{$\Gamma \vdash M : A$}
        \RightLabel{${\bf K}_I$}
        \UnaryInfC{$\Gamma \vdash {\bf pure \: } \: M : {\bf K}A $}
      \end{prooftree}

    \begin{prooftree}
      \AxiomC{$\Gamma \vdash \vec{M} : {\bf K} \vec{A}$}
      \AxiomC{$\vec{x} : \vec{A} \vdash N : B$}
      \RightLabel{$\text{let}_{{\bf K}}$}
      \BinaryInfC{$\Gamma \vdash {\bf let \: pure \:} \vec{x} = \vec{M} {\: \bf in \: } N : {\bf K} B$}
    \end{prooftree}
  \end{defin}

  $\Gamma \vdash \vec{M} : {\bf K} \vec{A}$ -- это синтаксический сахар для $\Gamma \vdash M_1 : {\bf K}A_1,\dots,\Gamma \vdash M_n : {\bf K}A_n$ и $\vec{x} : \vec{A} \vdash N : B$ -- это краткая форма для $x_1 : A_1, \dots, x_n : A_n \vdash N : B$.
  ${\bf let \: pure \:} \vec{x} = \vec{M} {\: \bf in \: } N$ -- это мгновенное локальное связывание в терме $N$.
  Мы будем использовать такую краткую форму вместо ${\bf let \: pure\:} x_1,\dots,x_n = M_1,\dots,M_n {\: \bf in \:} N$.

\end{frame}


\begin{frame}
  \frametitle{Метатеоретические свойства системы}

(Само определение редукции довольно сложное и длинное)

  \begin{theor} Редукция субъекта

  Если $\Gamma \vdash M : A$ и $M \twoheadrightarrow_{\beta \eta} N$, тогда $\Gamma \vdash N : A$

  \end{theor}

  \begin{theor}
  Отношение $\twoheadrightarrow_{\beta}$ сильно нормализуемо;
  \end{theor}

  \begin{theor}
  Отношение $\twoheadrightarrow_{\beta}$ конфлюентно.
  \end{theor}

  \begin{theor}
  Нормальная форма $\lambda_{{\bf K}}$ со стратегией вычисления с вызовом по имени обладает свойством подформульности:
если $M$ в нормальной форме, то всего его подтермы также в нормальной форме.

  \end{theor}
\end{frame}

\begin{frame}
  \frametitle{Семантика: Определение аппликативного функтора.}

  \begin{defin}
    Аппликативный функтор -- это тройка $\langle \mathcal{C}, \mathcal{K}, \eta \rangle$,
  где $\mathcal{C}$ -- это моноидальная категория, $\mathcal{K}$ - это моноидальный эндофунктор и
  $\eta : Id_{\mathcal{C}} \Rightarrow \mathcal{K}$ -- это естественное преобразование с условиями когерентности для него.
  \end{defin}
\end{frame}


\begin{frame}
  \frametitle{Теоретико-категорная семантика.}

  \begin{theor} Корректность
Пусть $\Gamma \vdash M : A$ и $M =_{\beta\eta} N$, тогда $[\![\Gamma \vdash M : A]\!] = [\![\Gamma \vdash N : A]\!]$
  \end{theor}

  Интерпретация модальных правил:

  \begin{prooftree}
  \AxiomC{$[\![\Gamma \vdash M : A]\!] = [\![M]\!] : [\![\Gamma]\!] \rightarrow [\![A]\!]$}
  \UnaryInfC{$[\![\Gamma \vdash {\bf pure \:} M : {\bf K}A]\!] := [\![\Gamma]\!] \xrightarrow{[\![M]\!]}
  [\![A]\!] \xrightarrow{\eta_{[\![A]\!]}} \mathcal{K}[\![A]\!]$}
  \end{prooftree}

\begin{small}
  \begin{prooftree}
    \AxiomC{$[\![\Gamma \vdash \vec{M} : {\bf K} \vec{A}]\!] = \langle [\![M_1]\!],\dots, [\![M_n]\!] \rangle : [\![\Gamma]\!] \rightarrow \prod \limits_{i=1}^n \mathcal{K}[\![A_i]\!]$}
    \AxiomC{$[\![\vec{x} : \vec{A} \vdash N : B]\!] = [\![N]\!] : \prod \limits_{i=1}^n [\![A_i]\!] \rightarrow [\![B]\!]$}
    \BinaryInfC{$[\![\Gamma \vdash {\bf let \: pure \:} \vec{x} = \vec{M} {\: \bf in \: } M : {\bf K} B]\!] = \mathcal{K}([\![N]\!]) \circ \ast_{[\![A_1]\!],\dots,[\![A_n]\!]} \circ \langle [\![M_1]\!],\dots, [\![M_n]\!] \rangle : [\![\Gamma]\!] \rightarrow \mathcal{K}[\![B]\!]$}
  \end{prooftree}
\end{small}
\end{frame}

\begin{frame}
  \frametitle{Теоретико-категорная семантика.}

  \begin{lem} Интерпретация сохраняет подстановку.

  $[\![M [x_1 := M_1,\dots, x_n := M_n]]\!] = [\![M]\!] \circ \langle [\![M_1]\!], \dots, [\![M_n]\!] \rangle$.
  \end{lem}

  \begin{lem} Интерпретация сохраняет редукцию.

      Пусть $\Gamma \vdash M : A$ и $M \twoheadrightarrow_{\beta \eta} N$, тогда $[\![\Gamma \vdash M : A]\!] = [\![\Gamma \vdash N : A]\!]$;
  \end{lem}
\end{frame}


\begin{frame}
  \frametitle{Теоретико-категорная семантика. Полнота.}

  \begin{defin} Эквивалетность на парах вида переменная-терм:
    $ $

    Определим такое бинарное отношение  $\sim_{A, B} \subseteq \mathbb{V} \times \Lambda_{{\bf K}}$, что:

    $(x, M) \sim_{A, B} (y, N) \Leftrightarrow x : A \vdash M : B \:\: \& \:\: y : A \vdash N : A \:\: \& \:\: M =_{\beta \eta} N [y := x]$.
  \end{defin}

Обозначим класс эквивалентности как $[x, M]_{A, B} = \{ (y, N) \: | \: (x, M) \sim_{A, B} (y, N) \}$ (ниже мы будем опускать индексы).

\end{frame}

\begin{frame}
  \frametitle{Теоретико-категорная семантика. Полнота.}

  \begin{defin} Категория $\mathcal{C}(\lambda)$:
  \begin{itemize}
    \item $Ob_{\mathcal{C}} = \{ \hat{A} \: | \: A \in \mathbb{T} \} \cup \{ \mathds{1} \}$;
    \item $Hom_{\mathcal{C}(\lambda)}(\hat{A},\hat{B}) = (\mathbb{V} \times \Lambda_{{\bf K}})/_{\sim_{A, B}}$;
    \item Пусть $[x, M] \in Hom_{\mathcal{C}(\lambda)}(\hat{A},\hat{B})$ и $[y,N] \in Hom_{\mathcal{C}(\lambda)}(\hat{B},\hat{C})$, тогда $[y,M] \circ [x, M] = [x, N [y := M]]$;
    \item Тождественный морфизм $id_{\hat{A}} = [x,x] \in Hom_{\mathcal{C}(\lambda)(\hat{A})}$;
    \item Терминальный объект $\mathds{1}$;
    \item $\widehat{A \times B} = \hat{A} \times \hat{B}$;
    \item Каноническая проекция: $[x, \pi_i x] \in Hom_{\mathcal{C}(\lambda)}(\hat{A_1} \times \hat{A_2},\hat{A_i})$ for $i \in \{ 1, 2 \}$;
    \item $\widehat{A \to B} = \hat{B}^{\hat{A}}$;
    \item Вычисляющая стрелка $\epsilon = [x, (\pi_2 x) (\pi_1 x)] \in Hom_{\mathcal{C}(\lambda)(\hat{B}^{\hat{A}} \times \hat{A}, \hat{B})}$.
  \end{itemize}
  \end{defin}

\end{frame}

\begin{frame}
  \frametitle{Теоретико-категорная семантика. Полнота.}

Необходимо показать, что ${\bf K}$ -- это аппликативный функтор над декартово замкнутой категорией $\mathcal{C}(\lambda)$.

\begin{theor}
  ${\bf K}$ -- это аппликативный функтор над $\mathcal{C}(\lambda)$.
\end{theor}

\end{frame}


\begin{frame}
  \frametitle{Спасибо за внимание!}

  Черепашка ниндзя Донателло пишет представителя класса типов \verb"Applicative":

  \animategraphics[loop,controls,width=0.5515\linewidth]{2}{giphy-}{0}{8}

\end{frame}


\end{document}
