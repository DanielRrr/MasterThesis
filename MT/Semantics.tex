\section{Теоретико-категорная семантика}

\begin{theorem} Корректность

  Пусть $\Gamma \vdash M : A$ и $M =_{\beta\eta} N$, тогда $[\![\Gamma \vdash M : A]\!] = [\![\Gamma \vdash N : A]\!]$
\end{theorem}

\begin{proof}

\begin{defin} Семантическая трансляция из $\lambda_{{\bf K}}$ в аппликативный функтор $\langle \mathcal{C}, \mathcal{K}, \eta \rangle$ над декартово замкнутой категорией $\mathcal{C}$:

\begin{itemize}
\item Интерпретация типлв:
  \begin{itemize}
    \item $[\![A]\!] := \hat{A}, A \in \mathbb{T}$, где $\hat{A}$ -- это объект категории $\mathcal{C}$, полученный в результате некоторого присваивания;
    \item $[\![A \to B]\!] := [\![B]\!]^{[\![A]\!]}$;
    \item $[\![A \times B]\!] := [\![A]\!] \times [\![B]\!]$.
  \end{itemize}
\item Интерпретация для модальных типов:
  \begin{itemize}
    \item $[\![{\bf K}A]\!] = \mathcal{K}[\![A]\!]$;
  \end{itemize}
\item Интерпретация для контекстов:
  \begin{itemize}
    \item $[\![ \quad ]\!] = \mathds{1}$, где $\mathds{1}$ -- это терминальный объект в заданной декартово замкнутой категории;
    \item $[\![\Gamma, x : A]\!] = [\![\Gamma]\!] \times [\![A]\!]$
  \end{itemize}
\item Интерпретация для типовых объявлений:
  \begin{itemize}
    \item $[\![\Gamma \vdash M : A]\!] := [\![M]\!] : [\![\Gamma]\!] \to [\![A]\!]$.
  \end{itemize}
\item Интерпретация для правил типизации:

\begin{prooftree}
\AxiomC{$ $}
\UnaryInfC{$[\![\Gamma, x : A \vdash x : A]\!] = \pi_2 : [\![\Gamma]\!] \times [\![A]\!] \rightarrow
[\![A]\!]$}
\end{prooftree}

\begin{prooftree}
\AxiomC{$[\![\Gamma, x : A \vdash M : B]\!] = [\![M]\!] : [\![\Gamma]\!] \times [\![A]\!] \rightarrow [\![B]\!]$}
\UnaryInfC{$[\![\Gamma \vdash (\lambda x. M) : A \to B]\!] = \Lambda([\![M]\!]) : [\![\Gamma]\!]
\rightarrow[\![B]\!]^{[\![A]\!]}$}
\end{prooftree}

\begin{prooftree}
\AxiomC{$[\![\Gamma \vdash M : A \to B]\!] = [\![M]\!] : [\![\Gamma]\!] \rightarrow [\![B]\!]^{[\![A]\!]}$}
\AxiomC{$[\![\Gamma \vdash N : A]\!] = [\![N]\!] : [\![\Gamma]\!] \rightarrow [\![A]\!]$}
\BinaryInfC{$[\![\Gamma \vdash (M N) : B]\!] = [\![\Gamma]\!] \xrightarrow{\langle [\![M]\!], [\![N]\!]
\rangle} [\![B]\!]^{[\![A]\!]} \times [\![A]\!] \xrightarrow{\epsilon} [\![B]\!] $}
\end{prooftree}

\begin{prooftree}
\AxiomC{$[\![\Gamma \vdash M : A ]\!] = [\![M]\!] : [\![\Gamma]\!] \rightarrow [\![A]\!]$}
\AxiomC{$[\![\Gamma \vdash N : B ]\!] = [\![N]\!] : [\![\Gamma]\!] \rightarrow [\![B]\!]$}
\BinaryInfC{$[\![\Gamma \vdash \langle M, N \rangle : A \times B]\!] = \langle [\![M]\!], [\![N]\!] \rangle : [\![\Gamma]\!] \rightarrow
[\![A]\!] \times [\![B]\!]$}
\end{prooftree}

\begin{prooftree}
\AxiomC{$[\![\Gamma \vdash M : A_1 \times A_2]\!] = [\![M]\!] : [\![\Gamma]\!] \rightarrow [\![A_1]\!] \times
[\![A_2]\!]$}
\RightLabel{$i \in \{1,2\}$}
\UnaryInfC{$[\![\Gamma \vdash \pi_i M : A_i]\!] = [\![\Gamma]\!] \xrightarrow{[\![M]\!]} [\![A_1]\!] \times
[\![A_2]\!] \xrightarrow{\pi_i} [\![A_i]\!]$}
\end{prooftree}

\begin{prooftree}
\AxiomC{$[\![\Gamma \vdash M : A]\!] = [\![M]\!] : [\![\Gamma]\!] \rightarrow [\![A]\!]$}
\UnaryInfC{$[\![\Gamma \vdash {\bf pure \:} M : {\bf K}A]\!] := [\![\Gamma]\!] \xrightarrow{[\![M]\!]}
[\![A]\!] \xrightarrow{\eta_{[\![A]\!]}} \mathcal{K}[\![A]\!]$}
\end{prooftree}

\begin{small}
  \begin{prooftree}
    \AxiomC{$[\![\Gamma \vdash \vec{M} : {\bf K} \vec{A}]\!] = \langle [\![M_1]\!],\dots, [\![M_n]\!] \rangle : [\![\Gamma]\!] \rightarrow \prod \limits_{i=1}^n \mathcal{K}[\![A_i]\!]$}
    \AxiomC{$[\![\vec{x} : \vec{A} \vdash N : B]\!] = [\![N]\!] : \prod \limits_{i=1}^n [\![A_i]\!] \rightarrow [\![B]\!]$}
    \BinaryInfC{$[\![\Gamma \vdash {\bf let \: pure \:} \vec{x} = \vec{M} {\: \bf in \: } M : {\bf K} B]\!] = \mathcal{K}([\![N]\!]) \circ \ast_{[\![A_1]\!],\dots,[\![A_n]\!]} \circ \langle [\![M_1]\!],\dots, [\![M_n]\!] \rangle : [\![\Gamma]\!] \rightarrow \mathcal{K}[\![B]\!]$}
  \end{prooftree}
\end{small}
\end{itemize}
\end{defin}

\begin{defin} Одновременная подстановка

Пусть $\Gamma = \{ x_1 : A_1, ..., x_n : A_n \}$, $\Gamma \vdash M : A$ и для любых $i \in \{ 1,..., n \}$,
$\Gamma \vdash M_i : A_i$.

 Одновременная подстановка $M [ \vec{x} := \vec{M}]$ определяется рекурсивно:

\begin{itemize}
\item $x_i [ \vec{x} := \vec{M}] = M_i $;
\item $(\lambda x. M) [ \vec{x} := \vec{M}] = \lambda x. (M [ \vec{x} := \vec{M}])$;
\item $(M N) [ \vec{x} := \vec{M}] = (M [ \vec{x} = \vec{M}]) (N [ \vec{x} := \vec{M}])$;
\item $\langle M, N \rangle = \langle (M [ \vec{x} = \vec{M}]), (N [ \vec{x} := \vec{M}])\rangle$;
\item $(\pi_i P) [ \vec{x} := \vec{M}] = \pi_i (P [ \vec{x} = \vec{M}])$;
\item $({\bf pure \:} M) [ \vec{x} := \vec{M}] = {\bf pure \:} (M [ \vec{x} = \vec{M}])$;
\item $({\bf let \: pure} \: \vec{x} = \vec{M} {\: \bf in \:} N) [\vec{y} := \vec{P}] =
{\bf let \: pure} \: \vec{x} = (\vec{M} [\vec{y} := \vec{P}]) {\: \bf in \:} N$
\end{itemize}
\end{defin}

\begin{lemma}
$ $

$[\![M [x_1 := M_1,\dots, x_n := M_n]]\!] = [\![M]\!] \circ \langle [\![M_1]\!], \dots, [\![M_n]\!] \rangle$.

\end{lemma}

\begin{proof}

$ $

1)

$\begin{array}{lll}
& [\![\Gamma \vdash ({\bf pure \:} M) [ \vec{x} := \vec{M}] : {\bf K}A]\!] = [\![ \Gamma \vdash {\bf pure \:} (M [ \vec{x} := \vec{M}]) : {\bf K}A]\!] & \\
&\quad\quad\quad\quad\quad\quad\quad \text{Определение подстановки}& \\
&\eta_{[\![A]\!]} \circ [\![(M [ \vec{x} := \vec{M}])]\!]& \\
&\quad\quad\quad\quad\quad\quad\quad \text{интерпретация для {\bf pure}}\\
&\eta_{[\![A]\!]} \circ ([\![M]\!] \circ \langle [\![M_1]\!], \dots, [\![M_n]\!] \rangle) = & \\
&\quad\quad\quad\quad\quad\quad\quad \text{предположение индукции} *\\
&(\eta_{[\![A]\!]} \circ [\![M]\!]) \circ \langle [\![M_1]\!], \dots, [\![M_n]\!] \rangle = & \\
&\quad\quad\quad\quad\quad\quad\quad \text{ассоциативность композиции}&\\
&[\![ \Gamma \vdash {\bf pure \:} M : {\bf K}A]\!] \circ \langle [\![M_1]\!], \dots, [\![M_n]\!] \rangle = & \\
&\quad\quad\quad\quad\quad\quad\quad  \text{интерпретация для {\bf pure}}&
\end{array}$

\vspace{\baselineskip}

2)

\vspace{\baselineskip}

$\begin{array}{lll}
&[\![\Gamma \vdash ({\bf let \: pure} \: \vec{x} = \vec{M} {\: \bf in \:} N) [\vec{y} := \vec{P}] : {\bf K} B]\!] =& \\
&\quad\quad\quad\quad\quad\quad\quad \text{определение одновременной подстановки} &\\
&[\![\Gamma \vdash {\bf let \: pure} \: \vec{x} = (\vec{M} [\vec{y} := \vec{P}]) {\: \bf in \:} N : {\bf K}B]\!] =& \\
&\quad\quad\quad\quad\quad\quad\quad  \text{интерпретация $let_{\bf K}$} \\
&\mathcal{K}([\![N]\!]) \circ \ast_{[\![A_1]\!],\dots,[\![A_n]\!]} \circ [\![\Gamma \vdash (\vec{M} [\vec{y} := \vec{P}]) : {\bf K}\vec{A}]\!] =& \\
&\quad\quad\quad\quad\quad\quad\quad \text{предположение индукция}& \\
&\mathcal{K}([\![N]\!]) \circ \ast_{[\![A_1]\!],\dots,[\![A_n]\!]} \circ ([\![\vec{M}]\!] \circ \langle [\![P_1]\!],\dots,[\![P_n]\!]\rangle) = & \\
&\quad\quad\quad\quad\quad\quad\quad \text{ассоциативность композиции}& \\
&(\mathcal{K}([\![N]\!]) \circ \ast_{[\![A_1]\!],\dots,[\![A_n]\!]} \circ [\![\vec{M}]\!]) \circ \langle [\![P_1]\!],\dots,[\![P_n]\!]\rangle = & \\
&\quad\quad\quad\quad\quad\quad\quad \text{по интерпретации}& \\
&[\![\Gamma \vdash {\bf let \: pure} \: \vec{x} = \vec{M} {\: \bf in \:} N : {\bf K}B]\!] \circ \langle [\![P_1]\!],\dots,[\![P_n]\!]\rangle&
\end{array}$

\end{proof}

\begin{lemma}
  $ $

  Пусть $\Gamma \vdash M : A$ и $M \twoheadrightarrow_{\beta \eta} N$, тогда $[\![\Gamma \vdash M : A]\!] = [\![\Gamma \vdash N : A]\!]$;
\end{lemma}

\begin{proof}
  $ $

Случаи с правилом $\beta$-редукции для $let_{\bf K}$ рассмотрены здесь \cite{ModalK1}. Рассмотрим случаи с ${\bf pure}$.

\vspace{\baselineskip}

1) $[\![\Gamma \vdash {\bf let \: pure \:} \vec{x} = {\bf pure \:} \vec{M} {\: \bf in \:} N : {\bf K}B]\!] = [\![\Gamma \vdash {\bf pure \:} N [\vec{x} := \vec{M}] : {\bf K}B]\!]$

\vspace{\baselineskip}

$\begin{array}{lll}
&[\![\Gamma \vdash {\bf let \: pure \:} \vec{x} = {\bf pure \:} \vec{M} {\: \bf in \:} N : {\bf K}B]\!] = & \\
&\text{\quad\quad\quad\quad\quad\quad интепретация}& \\
&\mathcal{K}([\![N]\!]) \circ \ast_{[\![A_1]\!],\dots,[\![A_n]\!]} \circ \langle \eta_{[\![A_1]\!]} \circ [\![M_1]\!],\dots,\eta_{[\![A_n]\!]} \circ [\![M_n]\!] \rangle = \\
&\text{\quad\quad\quad\quad\quad\quad свойство пары морфизмов}& \\
&\mathcal{K}([\![N]\!]) \circ \ast_{[\![A_1]\!],\dots,[\![A_n]\!]} \circ (\eta_{[\![A_1]\!]} \times \dots \times \eta_{[\![A_n]\!]}) \circ \langle [\![M_1]\!], \dots, [\![M_n]\!]\rangle =& \\
&\text{\quad\quad\quad\quad\quad\quad ассоциативность композиции}& \\
&\mathcal{K}([\![N]\!]) \circ (\ast_{[\![A_1]\!],\dots,[\![A_n]\!]} \circ (\eta_{[\![A_1]\!]} \times \dots \eta_{[\![A_n]\!]})) \circ \langle [\![M_1]\!], \dots, [\![M_n]\!] \rangle =& \\
&\text{\quad\quad\quad\quad\quad\quad по определению аппликативного функтора}& \\
&\mathcal{K}([\![N]\!]) \circ \eta_{[\![A_1]\!] \times \dots \times [\![A_n]\!]} \circ \langle [\![M_1]\!], \dots, [\![M_n]\!] \rangle =& \\
&\text{\quad\quad\quad\quad\quad\quad естественность $\eta$}& \\
&\eta_{[\![B]\!]} \circ [\![N]\!] \circ \langle [\![M_1]\!], \dots, [\![M_n]\!] \rangle =& \\
&\text{\quad\quad\quad\quad\quad\quad ассоциативность композиции}& \\
&\eta_{[\![B]\!]} \circ ([\![N]\!] \circ \langle [\![M_1]\!], \dots, [\![M_n]\!]) \rangle =& \\
&\text{\quad\quad\quad\quad\quad\quad по лемме об одновременной подстановке}& \\
&\eta_{[\![B]\!]} \circ [\![N [\vec{x} := \vec{M}]]\!]& \\
&\text{\quad\quad\quad\quad\quad\quad интерпретация}& \\
&[\![\Gamma \vdash {\bf pure \:} (N [\vec{x} := \vec{M}]) : {\bf K}B]\!]&
\end{array}$

\vspace{\baselineskip}

2) $[\![\vdash {\bf let \: pure \:} \underline{\quad} = \underline {\quad} {\: \bf in \:} M : {\bf K} A]\!] = [\![\vdash {\bf pure \:} M : {\bf K} A]\!]$

$\begin{array}{lll}
&[\![\vdash {\bf let \: pure \:} \underline{\quad} = \underline {\quad} {\: \bf in \:} M : {\bf K} A]\!] = & \\
&\quad\quad\quad\quad\quad\quad \text{интерпретация}& \\
&\mathcal{K}([\![M]\!]) \circ u_{\mathds{1}} = & \\
&\quad\quad\quad\quad\quad\quad \text{определение аппликативного функтора}& \\
&\mathcal{K}([\![M]\!]) \circ \eta_{\mathds{1}} = & \\
&\quad\quad\quad\quad\quad\quad \text{естественность $\eta$} & \\
&\eta_{[\![A]\!]} \circ [\![M]\!] = &\\
&\quad\quad\quad\quad\quad\quad \text{интерпретация} & \\
&[\![\vdash {\bf pure \:} M : {\bf K} A]\!]&
\end{array}$
\end{proof}

\end{proof}

\begin{theorem} Полнота

Пусть $[\![\Gamma \vdash M : A]\!] = [\![\Gamma \vdash N : A]\!]$, тогда $M =_{\beta \eta} N$.
\end{theorem}

\begin{proof}

$ $

Мы будем работать с термовой моделью для простого типизированного $\lambda$-исчисления с $\times$ и $\to$, стандартно описанной здесь \cite{LambekScott}:

\begin{defin} Эквивалетность на парах вида переменная-терм:
  $ $

  Определим такое бинарное отношение  $\sim_{A, B} \subseteq \mathbb{V} \times \Lambda_{{\bf K}}$, что:

  $(x, M) \sim_{A, B} (y, N) \Leftrightarrow x : A \vdash M : B \:\: \& \:\: y : A \vdash N : A \:\: \& \:\: M =_{\beta \eta} N [y := x]$.
\end{defin}

Нетрудно заметить, что данное отношение является отношением эквивалентности.

Обозначим класс эквивалентности как $[x, M]_{A, B} = \{ (y, N) \: | \: (x, M) \sim_{A, B} (y, N) \}$ (ниже мы будем опускать индексы).


\begin{defin} Категория $\mathcal{C}(\lambda)$:
\begin{itemize}
  \item $Ob_{\mathcal{C}} = \{ \hat{A} \: | \: A \in \mathbb{T} \} \cup \{ \mathds{1} \}$;
  \item $Hom_{\mathcal{C}(\lambda)}(\hat{A},\hat{B}) = (\mathbb{V} \times \Lambda_{{\bf K}})/_{\sim_{A, B}}$;
  \item Пусть $[x, M] \in Hom_{\mathcal{C}(\lambda)}(\hat{A},\hat{B})$ и $[y,N] \in Hom_{\mathcal{C}(\lambda)}(\hat{B},\hat{C})$, тогда $[y,M] \circ [x, M] = [x, N [y := M]]$;
  \item Тождественный морфизм $id_{\hat{A}} = [x,x] \in Hom_{\mathcal{C}(\lambda)(\hat{A})}$;
  \item Терминальный объект $\mathds{1}$;
  \item $\widehat{A \times B} = \hat{A} \times \hat{B}$;
  \item Каноническая проекция: $[x, \pi_i x] \in Hom_{\mathcal{C}(\lambda)}(\hat{A_1} \times \hat{A_2},\hat{A_i})$ for $i \in \{ 1, 2 \}$;
  \item $\widehat{A \to B} = \hat{B}^{\hat{A}}$;
  \item Вычисляющая стрелка $\epsilon = [x, (\pi_2 x) (\pi_1 x)] \in Hom_{\mathcal{C}(\lambda)(\hat{B}^{\hat{A}} \times \hat{A}, \hat{B})}$.
\end{itemize}
\end{defin}

Достаточно показать, что ${\bf K}$ -- это аппликативный функтор над $\mathcal{C}(\lambda)$.

\begin{defin}
  Определим эндофунктор $\mathcal{K} : \mathcal{C}(\lambda) \to \mathcal{C}(\lambda)$ таким образом, что
для любых $[x,M] \in Hom_{\mathcal{C}(\lambda)}(\hat{A},\hat{B}), {\bf K}([x,M]) = [y, {\bf let \: pure \:} x = y {\: \bf in \:} M] \in Hom_{\mathcal{C}(\lambda)}({\bf K} \hat{A}, {\bf K} \hat{B})$
(обозначения: $\text{fmap } f$ для произвольной стрелки $f$).

\end{defin}

\begin{lemma} Функториальность

\begin{itemize}
  \item $\text{fmap }(g \circ f) = \text{fmap }(g) \circ \text{fmap }(f)$;
  \item $\text{fmap }(id_{\hat{A}}) = id_{{\bf K}\hat{A}}$.
\end{itemize}
\end{lemma}

\begin{proof}

Простая проверка с использованием правил редукции.

\end{proof}

\begin{defin}

  Определим естественные преобразования:

\begin{itemize}
  \item $\eta:Id \Rightarrow \mathcal{K}$, такое, что $\forall \hat{A} \in Ob_{\mathcal{C}(\lambda)}$, $\eta_{\hat{A}} = [x, {\bf pure \:} x] \in Hom_{\mathcal{C}(\lambda)}(\hat{A}, {\bf K}\hat{A})$;
  \item $\ast_{A,B}:{\bf K}\hat{A} \times {\bf K}\hat{B} \to {\bf K}(\hat{A} \times \hat{B})$, такое, что $\forall \hat{A}, \hat{B} \in Ob_{\mathcal{C}(\lambda)}, \ast_{\hat{A},\hat{B}} = [p, {\bf let \: pure \:} x,y = \pi_1 p, \pi_2 p {\: \bf in \:} \langle x, y \rangle] \in Hom_{\mathcal{C}(\lambda)}({\bf K}A \times {\bf K}B, {\bf K}(A \times B))$.
\end{itemize}
\end{defin}

Реализация $\ast$ в нашей термовой модели -- это частный случай правила ${\text{let}_{\bf K}}$:

\begin{prooftree}
\AxiomC{$p : {\bf K}A \times {\bf K}B \vdash p : {\bf K}A \times {\bf K}B$}
\UnaryInfC{$p : {\bf K}A \times {\bf K}B \vdash \pi_1 p : {\bf K} A$}
\AxiomC{$p : {\bf K}A \times {\bf K}B \vdash p : {\bf K}A \times {\bf K}B$}
\UnaryInfC{$p : {\bf K}A \times {\bf K}B \vdash \pi_2 p : {\bf K} B$}
\AxiomC{$x : A \vdash x : A$}
\AxiomC{$y : B \vdash y : B$}
\BinaryInfC{$x : A, y : B \vdash \langle x, y \rangle : A \times B$}
\TrinaryInfC{$p : {\bf K}A \times {\bf K}B \vdash {\bf let \: pure \:} x,y = \pi_1 p, \pi_2 p {\: \bf in \:} \langle x, y \rangle : {\bf K}(A \times B)$}
\end{prooftree}

\begin{lemma}
  $ $

  {\bf K} нестрогий моноидальный функтор.
\end{lemma}

\begin{proof}
$ $

See \cite{ModalK}
\end{proof}

\begin{lemma} Естественность и когерентность $\eta$:

\begin{itemize}
  \item $\text{fmap } f \circ \eta_A = \eta_B \circ f$;
  \item $\ast_{\hat{A},\hat{B}} \circ (\eta_{A} \times \eta_{B}) = \eta_{\hat{A} \times \hat{B}}$;
\end{itemize}
\end{lemma}

\begin{proof}
  $ $

  i) $\text{fmap } f \circ \eta_{\hat{A}} = \eta_{\hat{B}} \circ f$

\vspace{\baselineskip}

$\begin{array}{lll}
&\eta_{\hat{B}} \circ f = & \\
&\quad\quad\quad\quad\quad\quad \text{по определению}&\\
&[y, {\bf pure \:} y] \circ [x, M] = &\\
&\quad\quad\quad\quad\quad\quad \text{композиция} & \\
&[x, {\bf pure \:} y [y := M]] = &\\
&\quad\quad\quad\quad\quad\quad \text{подстановка}& \\
&[x, {\bf pure \:} M]& \\
& &\\
&\text{С другой стороны:}& \\
&\text{fmap } f \circ \eta_{\hat{A}} = & \\
&\quad\quad\quad\quad\quad\quad \text{по определению}& \\
&[z, {\bf let \: pure \:} x = z {\: \bf in \:} M] \circ [x, {\bf pure \: x}] = &\\
&\quad\quad\quad\quad\quad\quad \text{композиция}& \\
&[x, {\bf let \: pure \:} x = z {\: \bf in \:} M [z := {\bf pure \:} x]] = & \\
&\quad\quad\quad\quad\quad\quad \text{подстановка}& \\
&[x, {\bf let \: pure \:} x = {\bf pure \:} x {\: \bf in \:} M] = & \\
&\quad\quad\quad\quad\quad\quad \text{правило $\beta$-редукции}& \\
&[x, {\bf pure \:} M [x := x]] = &\\
&\quad\quad\quad\quad\quad\quad \text{постановка} & \\
&[x, {\bf pure \:} M]&
\end{array}$

\vspace{\baselineskip}

ii) $\ast_{\hat{A},\hat{B}} \circ (\eta_{\hat{A}} \times \eta_{\hat{B}}) = \eta_{\hat{A} \times \hat{B}}$

\vspace{\baselineskip}

$\begin{array}{lll}
& \ast_{\hat{A},\hat{B}} \circ (\eta_{\hat{A}} \times \eta_{\hat{B}}) = & \\
& \quad\quad\quad\quad\quad\quad \text{раскрытие} & \\
& [q, {\bf let \: pure \:} x, y = \pi_1 q, \pi_2 q {\: \bf in \:} \langle x, y \rangle] \circ [p, \langle {\bf pure \:} (\pi_1 p), {\bf pure \:} (\pi_2 p) \rangle] = & \\
& \quad\quad\quad\quad\quad\quad \text{композиция}& \\
& [p, {\bf let \: pure \:} x, y = \pi_1 q, \pi_2 q {\: \bf in \:} \langle x, y \rangle [q := \langle {\bf pure \:} (\pi_1 p), {\bf pure \:} (\pi_2 p) \rangle]] = & \\
& \quad\quad\quad\quad\quad\quad \text{подстановка}& \\
& [p, {\bf let \: pure \:} x, y = \pi_1 (\langle {\bf pure \:} (\pi_1 p), {\bf pure \:} (\pi_2 p) \rangle), \pi_2 (\langle {\bf pure \:} (\pi_1 p), {\bf pure \:} (\pi_2 p) \rangle) {\: \bf in \:} \langle x, y \rangle] = & \\
& \quad\quad\quad\quad\quad\quad \text{правило $\beta$-редукции}& \\
& [p, {\bf let \: pure \:} x, y = {\bf pure \:} (\pi_1 p), {\bf pure \:} (\pi_2 p) {\: \bf in \:} \langle x, y \rangle] = & \\
& \quad\quad\quad\quad\quad\quad \text{правило $\beta$-редукции}& \\
& [p, {\bf pure \:} (\langle x,y \rangle [x := \pi_1 p, y := \pi_2 p])] = & \\
& \quad\quad\quad\quad\quad\quad \text{подстановка}& \\
& [p, {\bf pure \:} \langle \pi_1 p, \pi_2 p \rangle] = & \\
& \quad\quad\quad\quad\quad\quad \text{правило $\eta$-редукции}& \\
& [p, {\bf pure \:} p] =& \\
& \quad\quad\quad\quad\quad\quad \text{определение}& \\
& \eta_{\hat{A} \times \hat{B}}&
\end{array}$

\end{proof}

\begin{defin}
  $ $

  $u_{\mathds{1}} = [\sqbullet, {\bf let \: pure \:} \underline{\quad} = \underline{\quad} {\: \bf in \:} \sqbullet] \in Hom_{\mathcal{C}(\lambda)}(\mathds{1}, {\bf K}\mathds{1})$.
\end{defin}

\begin{lemma}
  $ $

  $u_{\mathds{1}} = \eta_{\mathds{1}}$

\end{lemma}

\begin{proof}

  Очевидно.
\end{proof}

\begin{defin} Естественное преобразование для тензорно-сильного функтора
  $ $

  Пусть $[p, \langle {\bf pure \:} (\pi_1 p), \pi_2 p \rangle] \in Hom_{\mathcal{C}(\lambda)}(\hat{A} \times {\bf K}\hat{B}, {\bf K}\hat{A} \times {\bf K}\hat{B})$.

  Тогда естественное преобразование для тензорно-сильного функтора $\tau_{\hat{A}, \hat{B}} = \ast_{\hat{A}, \hat{B}} \circ [p, \langle {\bf pure \:} (\pi_1 p), \pi_2 p \rangle]$.
\end{defin}

Ясно, что полученное определение легко упрощается:

$\begin{array}{lll}
&\ast_{\hat{A}, \hat{B}} \circ [p, \langle {\bf pure \:} (\pi_1 p), \pi_2 p \rangle] = & \\
&\quad\quad\quad\quad\quad\quad \text{определение}&\\
&[p^{'}, {\bf let \: pure \:} x, y = \pi_1 p^{'}, \pi_2 p^{'} {\: \bf in \:} \langle x, y \rangle] \circ [p, \langle {\bf pure \:} (\pi_1 p), \pi_2 p \rangle] = & \\
&\quad\quad\quad\quad\quad\quad \text{композиция} & \\
&[p, {\bf let \: pure \:} x, y = \pi_1 p^{'}, \pi_2 p^{'} {\: \bf in \:} \langle x, y \rangle [p^{'} := \langle {\bf pure \:} (\pi_1 p), \pi_2 p \rangle]] = & \\
&\quad\quad\quad\quad\quad\quad \text{подстановка} & \\
&[p, {\bf let \: pure \:} x, y = \pi_1 (\langle {\bf pure \:} (\pi_1 p), \pi_2 p \rangle), \pi_2 (\langle \pi_1 p, {\bf pure \:} (\pi_2 p) \rangle) {\: \bf in \:} \langle x, y \rangle] = & \\
&\quad\quad\quad\quad\quad\quad \text{правило $\beta$-редукции} & \\
&[p, {\bf let \: pure \:} x, y = {\bf pure \:} (\pi_1 p), \pi_2 p {\: \bf in \:} \langle x, y \rangle]&
\end{array}$

\begin{lemma} Когерентность для $\tau$:

  \begin{itemize}
    \item $\text{fmap } \alpha_{\hat{A},\hat{B},\hat{C}} \circ \tau_{\hat{A} \times \hat{B}, \hat{C}} = \tau_{\hat{A}, \hat{B} \times \hat{C}} \circ (id_{\hat{A}} \times \tau_{\hat{B}, \hat{C}}) \circ \alpha_{\hat{A}, \hat{B}, {\bf K}\hat{C}}$;
    \item ${\bf K}(R_{\hat{A}}) \circ \tau_{\mathds{1}, \hat{A}} = R_{{\bf K}\hat{A}}$.
  \end{itemize}
  где $\alpha_{\hat{A},\hat{B},\hat{C}} = [p, \langle \pi_1 (\pi_1 p), \langle \pi_2 (\pi_1 p), \pi_2 p \rangle \rangle]$ и
  $R = \pi_2$.
\end{lemma}

\begin{proof}
  $ $

1) Определим $\tau_{\hat{A} \times \hat{B}, \hat{C}}$ как:

$\tau_{\hat{A} \times \hat{B}, \hat{C}} = [p, {\bf \: let \: pure \:} x,y,z = {\bf pure\:} (\pi_1 (\pi_1 p)), {\bf pure \:} (\pi_2 (\pi_1 p)), \pi_2 p {\: \bf in \:} \langle \langle x,y \rangle, z \rangle]$

\vspace{\baselineskip}

Тогда:

\vspace{\baselineskip}

$\begin{array}{lll}
&\text{fmap } \alpha_{\hat{A},\hat{B},\hat{C}} \circ \tau_{\hat{A} \times \hat{B}, \hat{C}} = & \\
&[q, {\bf let \: pure \:} r = q {\: \bf in \:}  \langle \pi_1 (\pi_1 r), \langle \pi_2 (\pi_1 r), \pi_2 r \rangle \rangle] \: \circ & \\
&\quad\quad\quad\quad \circ \: [p, {\bf \: let \: pure \:} x,y,z = {\bf pure\:} (\pi_1 (\pi_1 p)), {\bf pure \:} (\pi_2 (\pi_1 p)), \pi_2 p {\: \bf in \:} \langle \langle x,y \rangle, z \rangle] = & \\
&\quad\quad\quad\quad\quad\quad \text{композиция}& \\
&[p, {\bf let \: pure \:} r = q {\: \bf in \:}  \langle \pi_1 (\pi_1 r), \langle \pi_2 (\pi_1 r), \pi_2 r \rangle \rangle & \\
&\quad\quad\quad\quad [q := {\bf \: let \: pure \:} x,y,z = {\bf pure\:} (\pi_1 (\pi_1 p)), {\bf pure \:} (\pi_2 (\pi_1 p)), \pi_2 p {\: \bf in \:} \langle \langle x,y \rangle, z \rangle]] = & \\
&\quad\quad\quad\quad\quad\quad \text{подстановка и правило $\beta$-редукции}& \\
&[p, {\bf let \: pure \:} r = ({\bf let \: pure \:} x,y,z = {\bf pure\:} (\pi_1 (\pi_1 p)), {\bf pure \:} (\pi_2 (\pi_1 p)), \pi_2 p {\: \bf in \:} \langle \langle x,y \rangle, z \rangle) & \\
&\quad\quad\quad\quad {\: \bf in \:} \langle \pi_1 (\pi_1 r), \langle \pi_2 (\pi_1 r), \pi_2 r \rangle \rangle] = & \\
&\quad\quad\quad\quad\quad\quad \text{правило $\beta$-редукции}& \\
&[p, {\bf let \: pure \:} x,y,z = {\bf pure\:} (\pi_1 (\pi_1 p)), {\bf pure \:} (\pi_2 (\pi_1 p)), \pi_2 p {\: \bf in \:}  \langle \pi_1 (\pi_1 r), \langle \pi_2 (\pi_1 r), \pi_2 r \rangle \rangle & \\
&\quad\quad\quad\quad [r := \langle \langle x,y \rangle, z \rangle]] = & \\
&\quad\quad\quad\quad\quad\quad \text{правило $\beta$-редукции}& \\
&[p, {\bf let \: pure \:} x,y,z = {\bf pure\:} (\pi_1 (\pi_1 p)), {\bf pure \:} (\pi_2 (\pi_1 p)), \pi_2 p {\: \bf in \:} \langle x, \langle y, z \rangle \rangle]&
\end{array}$

\vspace{\baselineskip}

С другой стороны,

\vspace{\baselineskip}

Определим $\tau_{\hat{A}, \hat{B} \times \hat{C}}$:

$\begin{array}{lll}
&\tau_{\hat{A}, \hat{B} \times \hat{C}} = [r, {\bf let \: pure \:} x, y, z = ({\bf pure \:} \pi_1 r, {\bf let \: pure \:} q^{'} = \pi_2 r {\: \bf in \:} \pi_1 q^{'}, {\bf let \: pure \:} q^{'} = \pi_2 r {\: \bf in \:} \pi_2 q^{'}) & \\
&\quad\quad\quad\quad {\: \bf in \:} \langle x, \langle y, z \rangle\rangle]& \\
\end{array}$

\vspace{\baselineskip}

Упростим данный частный случай естественного преобразования для тензорно-сильного функтора:

$\begin{array}{lll}
&[p, {\bf let \: pure \:} x, y, z = ({\bf pure \:} \pi_1 p, {\bf let \: pure \:} p^{'} = \pi_2 p {\: \bf in \:} \pi_1 p^{'}, {\bf let \: pure \:} p^{'} = \pi_2 p {\: \bf in \:} \pi_2 p^{'}) & \\
&\quad\quad\quad\quad {\: \bf in \:} \langle x, \langle y, z \rangle\rangle] = & \\
&\quad\quad\quad\quad\quad\quad \text{подстановка и правила редукции}& \\
&[p, {\bf let \: pure \:} x, p^{'}, z = ({\bf pure \:} \pi_1 p, \pi_2 p, {\bf let \: pure \:} p^{'} = \pi_2 p {\: \bf in \:} \pi_2 p^{'}) {\: \bf in \:} \langle x, \langle \pi_1 p^{'}, z \rangle\rangle] = & \\
&\quad\quad\quad\quad\quad\quad \text{подстановка и правила редукции}& \\
&[p, {\bf let \: pure \:} x, p^{'} = {\bf pure \:} \pi_1 p, \pi_2 p {\: \bf in \:} \langle x, \langle \pi_1 p^{'}, \pi_2 p^{'} \rangle\rangle]&
\end{array}$

\vspace{\baselineskip}

Тогда:

\vspace{\baselineskip}

$\begin{array}{lll}
&\tau_{\hat{A}, \hat{B} \times \hat{C}} \circ (id_{\hat{A}} \times \tau_{\hat{B}, \hat{C}}) \circ \alpha_{\hat{A}, \hat{B}, {\bf K}\hat{C}} = & \\
&\quad\quad\quad\quad\quad\quad \text{раскрытие}& \\
&\tau_{\hat{A}, \hat{B} \times \hat{C}} \circ [q, \langle \pi_1 q, {\bf let \: pure \:} y, z = {\bf pure \:} (\pi_1 (\pi_2 q)), \pi_2 (\pi_2 q) {\: \bf in \:} \langle y, z \rangle \rangle] \: \circ & \\
&\quad\quad\quad\quad \circ \: [p, \langle \pi_1 (\pi_1 p), \langle \pi_2 (\pi_1 p), \pi_2 p \rangle \rangle] = & \\
&\quad\quad\quad\quad\quad\quad \text{композиция}& \\
&\tau_{\hat{A}, \hat{B} \times \hat{C}} \circ [p, \langle \pi_1 q, {\bf let \: pure \:} y, z = {\bf pure \:} (\pi_1 (\pi_2 q)), \pi_2 (\pi_2 q) {\: \bf in \:} \langle y, z \rangle \rangle & \\
&\quad\quad\quad\quad [q := \langle \pi_1 (\pi_1 p), \langle \pi_2 (\pi_1 p), \pi_2 p \rangle \rangle]] = & \\
&\quad\quad\quad\quad\quad\quad \text{подстановка и редукция}& \\
&\tau_{\hat{A}, \hat{B} \times \hat{C}} \circ [p, \langle \pi_1 (\pi_1 p), {\bf let \: pure \:} y, z = {\bf pure \:} \pi_2 (\pi_1 p), \pi_2 p {\: \bf in \:} \langle y, z \rangle \rangle] = & \\
&\quad\quad\quad\quad\quad\quad \text{подстановка}& \\
&[r, {\bf let \: pure \:} x, p^{'} = {\bf pure \:} \pi_1 r, \pi_2 r {\: \bf in \:} \langle x, \langle \pi_1 p^{'}, \pi_2 p^{'} \rangle\rangle] \: \circ & \\
&\quad\quad\quad\quad \circ \: [p, \langle \pi_1 (\pi_1 p), {\bf let \: pure \:} y, z = {\bf pure \:} \pi_2 (\pi_1 p), \pi_2 p {\: \bf in \:} \langle y, z \rangle \rangle] = & \\
&\quad\quad\quad\quad\quad\quad \text{подстановка и редукция}& \\
&[p, {\bf let \: pure \:} x, p^{'} = ({\bf pure \:} (\pi_1 (\pi_1 p)), {\bf let \: pure \:} y, z = ({\bf pure \:} \pi_2 (\pi_1 p), \pi_2 p {\: \bf in \:} \langle y, z \rangle \rangle) & \\
&\quad\quad\quad\quad {\: \bf in \:} \langle x, \langle \pi_1 p^{'}, \pi_2 p^{'} \rangle\rangle ] = & \\
&\quad\quad\quad\quad\quad\quad \text{редукция}& \\
&[p, {\bf let \: pure \:} x,y,z = {\bf pure \:} (\pi_1 (\pi_1 p)), {\bf pure \:} \pi_2 (\pi_1 p), \pi_2 p {\: \bf in \:} \langle x, \langle \pi_1 p^{'}, \pi_2 p^{'} \rangle\rangle [p^{'} := \langle y, z \rangle]] = & \\
&\quad\quad\quad\quad\quad\quad \text{подстановка и редукция}& \\
&[p, {\bf let \: pure \:} x,y,z = {\bf pure \:} (\pi_1 (\pi_1 p)), {\bf pure \:} \pi_2 (\pi_1 p), \pi_2 p {\: \bf in \:} \langle x, \langle y, z \rangle\rangle]&
\end{array}$

\vspace{\baselineskip}

2) Очевидно.

\end{proof}

\begin{lemma}

  {\bf K} -- это аппликативный функтор.
\end{lemma}

\begin{proof}
  Непосредственно следует из предыдущих лемм.
\end{proof}

Аналогично \cite{Abramsky}, мы применяем трансляцию из $\lambda_{{\bf K}}$ к произвольной декартово замкнутой категории с аппликативным функтором $\mathcal{K}$, тогда
мы имеем $[\![\Gamma \vdash M : A]\!] = [x, M [x_i := \pi_i x]]$, so $M =_{\beta \eta} N \Leftrightarrow [\![\Gamma \vdash M : A]\!] = [\![\Gamma \vdash N : A]\!]$.

\end{proof}
