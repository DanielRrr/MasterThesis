\addcontentsline{toc}{section}{Список использованной литературы}

\begin{thebibliography}{}

\bibitem{Artemov} Artemov S. and Protopopescu T., \/ ``Intuitionistic Epistemic Logic'', \textit{The
Review of Symbolic Logic}, 2016, vol. 9, no 2. pp. 266-298.\parskip=1mm

\bibitem{Krupski} Krupski V. N. and Yatmanov A., \/ ``Sequent Calculus for Intuitionistic Epistemic Logic
IEL'', \textit{Logical Foundations of Computer Science: International Symposium, LFCS 2016, Deerfield
Beach, FL, USA, January 4-7, 2016. Proceedings}, 2016, pp. 187-201.\parskip=1mm

\bibitem{Haskell} Haskell Language. // URL: https://www.haskell.org. (Date: 1.08.2017) \parskip=1mm

\bibitem{Idris} Idris. A Language with Dependent Types.// URL:https://www.idris-lang.org. (Date:
1.08.2017) \parskip=1mm

\bibitem{Purs} Purescript. A strongly-typed functional programming language that compiles to JavaScript.
URL: http://www.purescript.org. (Date: 1.08.2017) \parskip=1mm

\bibitem{Elm} Elm. A delightful language for reliable webapps. // URL: http://elm-lang.org. (Date:
1.08.2017) \parskip=1mm

\bibitem{Base} Hackage, \/ ``The base package'' // URL: https://hackage.haskell.org/package/base-4.10.0.0
(Date: 1.08.2017) \parskip=1mm

\bibitem{Miran} Lipovaca M, \/ ``Learn you a Haskell for Great Good!''. //URL:
http://learnyouahaskell.com/chapters (Date: 1.08.2017) \parskip=1mm

\bibitem{McP} McBride C. and Paterson R., ``Applicative programming with effects", \textit{Journal of
Functional Programming}, 2008, vol. 18, no 01. pp 1-13. \parskip=1mm

\bibitem{McP2} McBride C. and Paterson R, ``Functional Pearl. Idioms: applicative programming with
effects'', \textit{Journal of Functional Programming}, 2005. vol. 18, no 01. pp 1-20. \parskip=1mm

\bibitem{Neder} R. Nederpelt and H. Geuvers, ``Type Theory and Formal Proof: An Introduction''.
\textit{Cambridge University Press}, New York, NY, USA, 2014. pp. 436. \parskip=1mm

\bibitem{Morten} Sorensen M. H. and Urzyczyn P, ``Lectures on the Curry-Howard isomorphism'',
\textit{Studies in Logic and the Foundations of Mathematics}, vol. 149, \textit{Elsevier Science}, 1998.
pp 261. \parskip=1mm

\bibitem{Pierce} Pierce B. C., ``Types and Programming Languages''. \textit{Cambridge, Mass: The MIT
Press}, 2002. pp. 605. \parskip=1mm

\bibitem{Girard} Girard J.-Y., Taylor P. and  Lafont Y, ``Proofs and Types'', \textit{Cambridge University
Press}, New York, NY, USA, 1989. pp. 175. \parskip=1mm

\bibitem{Baren} Barendregt. H. P., ``Lambda calculi with types" // Abramsky S., Gabbay Dov M., and S. E.
Maibaum, ``Handbook of logic in computer science (vol. 2), Osborne Handbooks Of Logic In Computer
Science'', Vol. 2. \textit{Oxford University Press, Inc.}, New York, NY, USA, 1993. pp 117-309.
\parskip=1mm

\bibitem{Hindley} Hindley J. Roger, ``Basic Simple Type Theory''. \textit{Cambridge University Press}, New
York, NY, USA, 1997. pp. 185. \parskip=1mm

\bibitem{Lax} Pfenning F. and Davies R., ``A judgmental reconstruction of modal logic'',
\textit{Mathematical Structures in Computer Science}, vol. 11, no 4, 2001, pp. 511-540. \parskip=1mm

\bibitem{Baren2} H.P. Barendregt. The Lambda Calculus --- Its Syntax and Semantics. Studies in Logic and
the Foundations of Mathematics, vol. 103. Amsterdam: North-Holland, 1985.

\bibitem{ModalK} Yoshihiko KAKUTANI, A Curry-Howard Correspondence for Intuitionistic Normal Modal Logic, Computer Software, Released February 29, 2008, Online ISSN , Print ISSN 0289-6540.

\bibitem{ModalK1} Kakutani Y. (2007) Call-by-Name and Call-by-Value in Normal Modal Logic. In: Shao Z. (eds) Programming Languages and Systems. APLAS 2007. Lecture Notes in Computer Science, vol 4807. Springer, Berlin, Heidelberg

\bibitem{Abe} T. Abe. Completeness of modal proofs in first-order predicate logic. Computer Software, JSSST Journal, 24:165 -- 177, 2007.

\bibitem{LambekScott} Lambek, J. and Scott P.J. (1986) Introduction to Higher Order Categorical Logic, Cambridge Studies in Advanced Mathematics 7, Cambridge: Cambridge University Press.

\bibitem{ElKelly} Samuel Eilenberg and Max Kelly, Closed categories. Proc. Conf. Categorical Algebra (La Jolla, Calif., 1965).

\bibitem{Abramsky} Samson Abramsky and Nikos Tzevelekos, Introduction to Categories and Categorical Logic

\bibitem{ModalLa} G. A. Kavvos. The Many Worlds of Modal $\Lambda$--calculi: I. Curry-Howard for Necessity, Possibility and Time

\bibitem{Cons} Ross Paterson. in Mathematics of Program Construction, Madrid, 2012, Lecture Notes in Computer Science, vol. 7342, pp. 300--323, Springer, 2012.
\end{thebibliography}
