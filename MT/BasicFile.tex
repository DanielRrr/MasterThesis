\documentclass[a4paper, 14pt]{article}
\usepackage{setspace}
\usepackage{amsmath}
\usepackage{amsthm}
\usepackage{dsfont}
\usepackage{pgfplots}
\usepackage[utf8]{inputenc}
\usepackage{tikz-cd}
\usepackage[all, 2cell]{xy}
\usepackage{amssymb}
\usepackage{verbatim}
\usepackage[all]{xy}
\usepackage{tikz}
\usepackage{hyperref}
\usepackage{mathabx}
\usepackage{bussproofs}
\usepackage[english,russian]{babel}
\linespread{1.4}
\usepackage{geometry}
\geometry{left=3cm}
\geometry{right=1.5cm}
\geometry{top=2.4cm}
\geometry{bottom=2.4cm}
\newtheorem{theorem}{Теорема}
\newtheorem{prop}{Предложение}
\newtheorem{lemma}{Лемма}
\newtheorem{defin}{Определение}
\newtheorem{ex}{Пример}
\newtheorem{col}{Следствие}
\usepackage{listings} 		% for source code
\date{}
\title{Теоретико-категорная семантика модальной теории типов, основанной на интуиционистской эпистемической логике}
\begin{document}
\maketitle

\section{Предварительные замечания и определения}

\subsection{Глоссарий по основным конструкциям функционального языка программирования Haskell: функторы и аппликативные функторы}

\begin{defin} Класс типов

  Классом типов в языке Haskell -- это реализация некоторого общего интерфейса для некоторой совокупности типов.

  Представителем класса типов называется реализация данного класса для конкретного типа.
\end{defin}

\begin{defin} Функтор

  Функтор -- это однопараметрический класс типов, позволяющий пронести действие одноместной функции через значения,
  полученные в результате применения к их типу одноместного типового оператора.
\end{defin}

Определение в стандартной библиотеке выглядит следующим образом:

\begin{lstlisting}[language=Haskell]
  class Functor f where
    fmap :: (a -> b) -> f a -> f b
\end{lstlisting}

Рассмотрим примеры:

\begin{itemize}
  \item Список (неограниченная в длине последовательность) является функтором:

  \begin{lstlisting}[language=Haskell]
    instance Functor [] where
      fmap :: (a -> b) -> [a] -> [b]
      fmap f [] = []
      fmap f (x:xs) = (f x) : (fmap f xs)
  \end{lstlisting}

  Данный пример достаточно прост: реализация функтора для списка -- это функция высшего порядка, которая,
  принимая на входе одноместную функцию из типа $a$ в тип $b$ и список элементов типа $a$, возвращает список элементов типа $b$, который
  получен применением функции к каждому элементу списка, полученного на вход.

  Пример использования:

\begin{lstlisting}[language=Haskell]
  > fmap succ ''bg`lo`fmd\USrtodqmnu`''
  ''champagne supernova''
\end{lstlisting}

В данном примере мы отобразили функцию ``следующий за''
\footnote{В языке Haskell функция ``следующий за'' определения для
любого типа, между элементами которого есть линейный порядок,
в частности, лексиграфический порядок на символах, вводимых с клавиатуры компьютера}
на строчку, указанную выше. Строка в языка Haskell -- это частный случай списка, списка символов.
Тогда \verb"fmap" применяет функцию \verb"succ" к каждой букве строки, возвращая в результате
строку со словосочетанием ``champagne supernova''.

  \item Пара (тип декартова произведения типов) также функтор:
  \begin{lstlisting}[language=Haskell]
  instance Functor (b,) where
    fmap :: (a -> c) -> (b,a) -> (b,c)
    fmap f (x,y) = (x, f y)
  \end{lstlisting}

  Конструктор пары является двухпараметрическим типовым оператором, но мы сделали из него однопараметрический оператор фиксацией первого параметра.

  Данная реализация также довольно проста: на вход принимается функция из типа $a$ в тип $c$ и кортеж, в котором первая координата имеет тип $b$, а вторая -- тип $a$.
  На выходе мы получаем кортеж типа $(b,c)$, применяя полученную на вход функцию ко второй координате пары.

  \item Тип $Maybe$ -- это однопараметрический типовой оператор, для обработки неопределенных значений, иными словами тип $Maybe$ доопределяет
  частично-определенную функцию до тотальной:
  \begin{lstlisting}[language=Haskell]
    data Maybe a = Nothing | Just a
  \end{lstlisting}

  Реализация функтора для типа $Maybe$:
  \begin{lstlisting}[language=Haskell]
    instance Functor Maybe where
      fmap :: (a -> b) -> Maybe a -> Maybe b
      fmap f Nothing = Nothing
      fmap f (Just x) = Just (f x)
  \end{lstlisting}

  Если второй аргумент является неопределенным значением (на вход передан $Nothing$), то и возвращается $Nothing$. Если же значение определено, то есть оно имеет вид $Just \: x$, тогда
  мы применяем функцию функцию к $x$, а результат вычисления оборачиваем в конструктор $Just$.
\end{itemize}

\begin{defin} Аппликативный функтор

  Аппликативным функтором называется класс типов, обобщающий функтор для функций произвольной арности.
\end{defin}

Определение класса в языке Haskell:

\begin{lstlisting}[language=Haskell]
  class Functor f => Applicative f where
    pure :: a -> f a
    (<*>) :: f (a -> b) -> f a -> f b
\end{lstlisting}

Обобщение действия функтора с помощью методой класса \verb"Applicative" для произвольной функции:
\begin{lstlisting}[language=Haskell]
  liftA2 :: (a -> b -> c) -> f a -> f b -> f c
  liftA2 g x y = pure g <*> x <*> y
\end{lstlisting}

Данный комбинатор берет функцию, которая по объектам из типов $a$ и $b$ сопоставляет объект типа $c$ и,
по аргументам $x$ и $y$ типов соответственно $f \: a$ и $f \: b$, полученных в результате применения к данным типов функтора $f$,
возвращает объект типа $f \: c$, тип которого также получен в результате применения функтора к типу $f \: c$.

Рассмотрим реализацию данной функции детальнее. Имея функцию $g$ типа \verb"a -> b -> c", мы
применяет к ней \verb"pure", получая в результате объект \verb"pure g" типа \verb"f (a -> b -> c)",
иными словами, мы подняли функцию $g$ на уровень функтора $f$.

Далее мы применяем поднятую функцию $f$ к аргументу $x$ типа $f \: a$ с использованием \verb"(<*>)"
и получаем объект \verb"pure g <*> x" типа \verb"f (b -> c)", получив одноместную функцию из \verb"b" в \verb"c".
Затем мы \verb"pure g <*> x" применяем к аргументу $y$ типа $f \: b$ опять же с использованием
\verb"(<*>)" и получаем объект типа $f \: c$.

Рассмотрим примеры представителей класса типов \verb"Applicative":

\begin{itemize}
  \item Списки
  \begin{lstlisting}[language=Haskell]
    instance Applicative [] where
      pure x = [x]
      fs <*> xs = [ f x | f <- fs , x <- xs]
  \end{lstlisting}

  Метод \verb"pure" для списков переводит объект $x$ произвольного типа $a$ в
  одноэлементный список $[x]$. Метод \verb"(<*>)" в качестве левого операдна
  принимает список функций и список аргументов -- в качестве правого и возващает
  список всевозможных применений элементов первого списка к элементам второго списка.
  Нотация здесь является калькой с теоретико-множественной нотации, которая могла бы иметь
  следующий вид $\{ f(x) \: | \: f \in B^A \land x \in A \}$.
  \item Пары
  \begin{lstlisting}[language=Haskell]
  instance Monoid a => Applicative ((,) a) where
      pure x = (mempty, x)
      (u, f) <*> (v, x) = (u <> v, f x)
  \end{lstlisting}

  Чтобы пару сделать аппликативным функтором, необходимо соблюсти следующее ограничение:
  тип первой координаты должен быть моноидом \footnote{То есть тип должен содержать нейтральный и бинарную ассоциативную операцию.
  Моноид в языке Haskell также является классом типов, который, как видно из названия, является калькой с одноименной структуры в алгебре.}.

  Метод \verb"pure" переводит объект $x$ произвольного типа $b$ в упорядоченную пару, первый элемент которой единица моноида $a$,
  а второй -- $x$, полученный при входе. Метод \verb"(<*>)" принимает две пары: первая пара -- это пара элемента $u$ моноида и некоторой функции $f$,
  вторая пара состоит из другого элемента $v$ моноида и аргумента $x$. Возвращаемый результат:
  соединение элементво $u$ и $v$ с помощью моноидной операции в первой координате, а во второй координате -- применение функции $f$ к аргументу $x$.

  Пример использования:
  \begin{lstlisting}[language=Haskell]
  > (''(what's the story) '' , succ) <*> (''morning glory?'' , 1994)
  (''(what's the story) morning glory?'',1995)
  \end{lstlisting}
  где \verb"succ" -- функция ``следующий за''.

  \item Тип Maybe
  \begin{lstlisting}[language=Haskell]
    instance Applicative Maybe where
      pure = Just
      Just f <*> m = fmap f m
      Nothing <*> _m = Nothing
  \end{lstlisting}

  В данной реализации представителя, \verb"pure" просто оборачивает значение типа произольно типа $a$ в $Just$,
  создавая таким образом объект типа \verb"Maybe a". Метод \verb"(<*>)" принимает функцию типа \verb"Maybe (a -> b)" и
  аргумент типа \verb"Maybe a". Если аргументы, переданные \verb"(<*>)", имеют вид $\verb"Just f"$ и $\verb"Just m"$,
  где $f$ и $m$ -- это объекты типов \verb"a -> b" и \verb"a" соответственно, тогда происходит применение функции $f$
  к аргументу $m$ внутри \verb"Just" и возвращаемое значение в таком случае имеет вид \verb"Just (f m)".
  В остальных случаях возвращается \verb"Nothing".

  Пример использования:
  \begin{lstlisting}[language=Haskell]
  > liftA2 (++) (Just "Definitely") Nothing
  Nothing

  > liftA2 (++) (Just ''Definitely '') (Just "Maybe")
  Just ''Definitely Maybe''
  \end{lstlisting}

Данный пример как раз является примером того, как действие функтора обобщается на функций многих аргументов,
в данном случае, двух аргументов. Здесь функцией двух аргументов является функция $(++)$, конкатенация строк.

В первом случае определен только первый аргумент, строка \verb"Definitely", обернутая в \verb"Just",
второй же аргумент является неопределенным (так как в качестве второго аргумента передан \verb"Nothing"),
поэтому данная попытка конкатенации строк также вернет \verb"Nothing".

Во втором примере, второй аргумент определен, это строка \verb"Maybe", обернутая в \verb"Just".
Тогда конкатенация пройдет успешно и \verb"liftA2" пронесет комбинатор $(++)$ и совершит конкатенацию строк
\verb"Definitely" и \verb"Maybe" внутри \verb"Just", и на выходе будет возвращена строка \verb"Definitely Maybe",
к которой применен \verb"Just".
\end{itemize}


\newpage

\section{Приложение А. Глоссарий по теории категорий.}

\begin{defin}

  Категория $\mathcal{C}$ состоит из:
  \begin{itemize}
    \item Класса объектов $Ob_{\mathcal{C}}$;
    \item Для любых объекта $A, B \in Ob_{\mathcal{C}}$ определено множество стрелок (или морфизмов) из $A$ в $B$ $Hom_{\mathcal{C}}(A, B)$;
    \item Если $f \in Hom_{\mathcal{C}}(A, B)$ и $g \in Hom_{\mathcal{C}}(B,C)$, то $g \circ f \in Hom_{\mathcal{C}}(A, C)$;
    \item Для любого объекта $A \in Ob_{\mathcal{C}}$, определен тождественный морфизм $id_A \in Hom_{\mathcal{C}}(A, A)$;
    \item Для любой стрелки $f \in Hom_{\mathcal{C}}(A, B)$, для любой стрелки $g \in Hom_{\mathcal{C}}(B,C)$ и для любой стрелки $h \in Hom_{\mathcal{C}}(C,D)$, $h \circ (g \circ f) = (h \circ g) \circ f$.
    \item Для любой стрелки $f \in Hom_{\mathcal{C}}(A, B)$, $f \circ id_A = f$ и $id_B \circ f = f$.
  \end{itemize}
\end{defin}

\begin{defin} Функтор

  Пусть $\mathcal{C}, \mathcal{D}$ -- категории. Функтором называется отображение $F : \mathcal{C} \to \mathcal{D}$, такое, что:
  \begin{itemize}
    \item $F : A \mapsto F A$, где $A \in Ob_{\mathcal{C}}$;
    \item $F (g \circ f) = F(g) \circ F(f)$;
    \item $F (id_A) = id_{F A}$.
  \end{itemize}
\end{defin}

\begin{defin} Естественное преобразование
  Пусть $\mathcal{F}, \mathcal{G} : \mathcal{C} \to \mathcal{D}$ -- функторы.
  Естественным преобразованием $\alpha : \mathcal{F} \Rightarrow \mathcal{G}$ называется такое индексированное семейство стрелок $(\alpha_{X})_{X \in Ob_{\mathcal{C}}}$,
что для любых $A, B \in Ob_{\mathcal{C}}$, для любой стрелки $f \in Hom_{\mathcal{C}}(A, B)$, диаграмма коммутирует:

\xymatrix{
&&&& \mathcal{F} A \ar[dd]_{\alpha_A} \ar[rr]^{\mathcal{F}(f)} && \mathcal{F} B \ar[dd]^{\alpha_B}\\
\\
&&&& \mathcal{G}A \ar[rr]_{\mathcal{G}(f)} && \mathcal{G} B
}
\end{defin}

\begin{defin} Моноидальная категория

  Моноидальная категория -- это категория $\mathcal{C}$ с дополнительной структурой:
  \begin{itemize}
    \item Бифунктор $\otimes : \mathcal{C} \times \mathcal{C} \to C$, который мы будем называть тензором;
    \item Единица $\mathds{1}$;
    \item Изоморфизм, который мы будем называть ассоциатором: $\alpha_{A,B,C}: (A \otimes B) \otimes C \cong A \otimes (B \otimes C)$;
    \item Изоморфизм  $L_A : \mathds{1} \otimes A \cong A$;
    \item Изоморфизм $R_A : A \otimes \mathds{1} \cong A$;
    \item Первое условие когерентности (пятиугольник Маклейна) (данная диаграмма коммутирует):

    \xymatrix{
      & (A \otimes (B \otimes C)) \otimes D \ar[dr]^{\alpha_{A,B \otimes C,D}}\\
      ((A \otimes B) \otimes C) \otimes D \ar[d]_{\alpha_{A \otimes B, C, D}} \ar[ur]^{\alpha_{A,B,C} \otimes id_D \quad} && A \otimes ((B \otimes C) \otimes D) \ar[d]^{id_A \otimes \alpha_{B,C,D}}\\
      (A \otimes B) \otimes (C \otimes D) \ar[rr]_{\alpha_{A,B,C \otimes D}}&& A \otimes (B \otimes (C \otimes D))
    }
    \item Второе условие когерентности (тождество треугольника):

    \xymatrix{
    && (A \otimes \mathds{1}) \otimes B \ar[rr]^{\alpha_{A, \mathds{1}, B}} \ar[dr]_{R_A \otimes id_B} && A \otimes (\mathds{1} \otimes B) \ar[dl]^{id_A \otimes L_B} \\
    &&& A \otimes B
    }

    \item Моноидальная категория $\mathcal{C}$ называется симметрической, если для любых $A,B \in Ob_{\mathcal{C}}$, имеет место изоморфизм $\sigma_{A,B} : A \otimes B \cong B \otimes A$.
  \end{itemize}
\end{defin}

\begin{defin} Декартово замкнутная категория

  Декартово замкнутная категория -- это категория с терминальным объектом, конечными произведениями и экспоненцированием.
\end{defin}

Легко видеть, что декартово замкнутая категория -- это частный случай (симметрической) моноидальной категории, в котором тензор -- это произведения, а единица -- это терминальный объект.

\begin{defin} Нестрогий моноидальный функтор

  Пусть $\langle \mathcal{C}, \otimes_1, \mathds{1}_{\mathcal{C}} \rangle$ и $\langle \mathcal{D}, \otimes_2, \mathds{1}_{\mathcal{D}} \rangle$ моноидальные категории.

  Нестрогий моноидальный функтор $\mathcal{F} : \langle \mathcal{C}, \otimes_1, \mathds{1} \rangle \to \langle \mathcal{D}, \otimes_2, \mathds{1}' \rangle$ это функтор
  $\mathcal{F} : \mathcal{C} \to \mathcal{D}$ с дополнительными естественными преобразованиями:

  \begin{itemize}
  \item $u : \mathds{1}_{\mathcal{D}} \to \mathcal{F}\mathds{1}_{\mathcal{C}}$;
  \item $\ast_{A, B} : \mathcal{F}A \otimes_{\mathcal{D}} \mathcal{F}B \to \mathcal{F}(A \otimes_{\mathcal{C}} B)$.
  \end{itemize}

  и условиями когерентности:

  \begin{itemize}
    \item Ассоциативность:

  \xymatrix{
    && (\mathcal{F}A \otimes_{\mathcal{D}} \mathcal{F}B) \otimes_{\mathcal{D}} \mathcal{F}C \ar[d]_{\ast_{A,B} \otimes_{\mathcal{D}} id_{\mathcal{F}B}}
    \ar[rr]^{\alpha^{\mathcal{D}}_{{\mathcal{F}A, \mathcal{F}B, \mathcal{F}C}}} && \mathcal{F}A \otimes_{\mathcal{D}} (\mathcal{F}B \otimes_{\mathcal{D}} \mathcal{F}C) \ar[d]^{id_{\mathcal{F}A} \otimes_{\mathcal{D}} \ast_{B,C}}\\
    && \mathcal{F}(A \otimes_{\mathcal{C}} B) \otimes_{\mathcal{D}} \mathcal{C} \ar[d]_{\ast_{A \otimes_{\mathcal{C}} B, C}} && \mathcal{F}A \otimes_{\mathcal{D}} \mathcal{F}(B \otimes_{\mathcal{C}} C) \ar[d]^{\ast_{A, B \otimes_{\mathcal{C}} C}}\\
    && \mathcal{F}((A \otimes_{\mathcal{C}} B) \otimes_{\mathcal{C}} C) \ar[rr]_{\mathcal{F}(\alpha^{\mathcal{C}}_{A,B,C})}&& \mathcal{F}(A \otimes_{\mathcal{C}} (B \otimes_{\mathcal{C}} C))
  }

    \item Свойство левой единицы:

    \xymatrix{
    &&  \mathds{1}_{\mathcal{D}} \otimes_{\mathcal{D}} \mathcal{F}A \ar[d]_{L^{\mathcal{D}}_{\mathcal{F}A}}\ar[rr]^{u \otimes_{\mathcal{D}} id_{\mathcal{F}A}} && \mathcal{F}\mathds{1}_{\mathcal{C}} \otimes_{\mathcal{D}} \mathcal{F}A \ar[d]^{\ast_{\mathds{1}_{\mathcal{C}}, A}} \\
    &&  \mathcal{F}A && \mathcal{F}(\mathds{1}_{\mathcal{C}} \otimes_{\mathcal{C}} A) \ar[ll]_{\mathcal{F}(L^{\mathcal{C}}_A)}
    }

    \item Свойство правой единицы:

    \xymatrix{
    &&  \mathcal{F}A \otimes_{\mathcal{D}} \mathds{1}_{\mathcal{D}} \ar[d]_{R^{\mathcal{D}}_{\mathcal{F}A}}\ar[rr]^{id_{\mathcal{F}A} \otimes_{\mathcal{D}} u} && \mathcal{F}A \otimes_{\mathcal{D}} \mathcal{F}\mathds{1}_{\mathcal{C}} \ar[d]^{\ast_{A, \mathds{1}_{\mathcal{C}}}} \\
    &&  \mathcal{F}A && \mathcal{F}(A \otimes_{\mathcal{C}} \mathds{1}_{\mathcal{C}} ) \ar[ll]^{\mathcal{F}(R^{\mathcal{C}}_A)}
    }
  \end{itemize}
\end{defin}

\begin{defin}

Тензорно-сильный функтор -- это эндофунктор над моноидальной категорией с дополнительным естественным преобразованием и условиями когерентности для него (ниже соответствующие коммутирующие диаграмы):

\begin{center}
$\begin{array}{lll}
  \tau_{A, B} : A \otimes \mathcal{K}B \rightarrow \mathcal{K}(A \otimes B)
\end{array}$
\end{center}

\xymatrix{
(A \otimes B) \otimes \mathcal{K}C \ar[d]_{\alpha_{A, B, \mathcal{K}C}}\ar[rrrr]^{\tau_{A \otimes B, C}} &&&& \mathcal{K}((A \otimes B) \otimes C) \ar[d]^{\mathcal{K}(\alpha_{A,B,C})}\\
A \otimes (B \otimes \mathcal{K}C) \ar[rr]_{id_A \otimes \tau_{B,C}} && A \otimes \mathcal{K}(B \otimes C) \ar[rr]_{\quad \tau_{A, (B \otimes C)}} && \mathcal{K}(A \otimes (B \otimes C)) \\
}

\xymatrix{
&&& \mathds{1} \otimes \mathcal{K}A \ar[drr]_{R_{\mathcal{K}A}} \ar[rr]^{\mu_{\mathds{1}, A}} && \mathcal{K}(\mathds{1} \otimes A) \ar[d]^{\mathcal{K}(R_A)}\\
&&&&& \mathcal{K}A
}
\end{defin}

\begin{defin} Аппликативный функтор

  Аппликативный функтор -- это тройка $\langle \mathcal{C}, \mathcal{K}, \eta \rangle$,
где $\mathcal{C}$ -- это моноидальная категория, $\mathcal{K}$ - это тензорно-сильный нестрогий моноидальный эндофунктор и
$\eta : Id_{\mathcal{C}} \Rightarrow \mathcal{K}$ -- это естественное преобразование, такое, что:

\begin{itemize}
\item $u = \eta_{\mathds{1}}$;
\item $\ast_{A,B} \circ (\eta_A \otimes \eta_B) = \eta_{A \otimes B}$, то есть диаграмма коммутирует:

\xymatrix
{
&&& A \otimes B \ar[rr]^{\eta_A \otimes \eta_B} \ar[drr]_{\eta_{A \otimes B}} && \mathcal{K}A \otimes \mathcal{K}B \ar[d]^{\ast_{A,B}} \\
&&&&& \mathcal{K}(A \otimes B)
}
\item $\tau_{A, B} = \ast_{A, B} \circ \eta_{A} \otimes id_{\mathcal{K}B}$.
\end{itemize}
\end{defin}

По умолчанию мы будем рассматривать ниже аппликативный функтор над декартово замкнутой категорией.


\newpage

\input{Intro}

\newpage

\section{Модальное $\lambda$-исчисление, основанное на исчислении IEL$^{-}$}

\subsection{Натуральный вывод для IEL$^{-}$}
Определим натуральное исчисление для IEL$^{-}$ :

\begin{defin} Hатуральное исчисление NIEL$^{-}$ для интуиционистской эпистемической логики IEL$^{-}$ -- это
расширение натурального исчисления для интуиционистской логики высказываний с добавлением следующих правил вывода для модальности:

\begin{minipage}{0.5\textwidth}
  \begin{flushleft}
  \begin{prooftree}
    \AxiomC{$\Gamma \vdash A$}
    \RightLabel{$\Box_I$}
    \UnaryInfC{$\Gamma \vdash \Box A$}
\end{prooftree}
  \end{flushleft}
\end{minipage}
\begin{minipage}{0.5\textwidth}
  \begin{flushright}
  \begin{prooftree}
  \AxiomC{$\Gamma \vdash \Box A_1, \dots, \Gamma \vdash \Box A_n $}
  \AxiomC{$A_1,\dots,A_n \vdash B$}
  \BinaryInfC{$\Gamma \vdash \Box B$}
  \end{prooftree}
  \end{flushright}
\end{minipage}
\end{defin}

Первое правило позволяет выводить ко-рефлексию. Второе модальное правило -- это аналог для правила $\Box_I$
в натуральном исчислении для конструктивной K (see \cite{ModalLa}) без $\Diamond$.

Мы будем обозначать $\Gamma \vdash \Box A_1, \dots, \Gamma \vdash \Box A_n$ и $A_1,\dots,A_n \vdash B$ соответственно как $\Gamma \vdash {\bf K} \vec{A}$ и $\vec{A} \vdash B$ для краткости.

\vspace{\baselineskip}

\begin{lemma}
  $\Gamma \vdash_{\text{NIEL}^{-}} A \Rightarrow$ IEL$^{-} \vdash \bigwedge \Gamma \rightarrow A$.
\end{lemma}

\begin{proof}
Индукция по построению вывода. Рассмотрим модальные случаи.

\vspace{\baselineskip}

1) Если $\Gamma \vdash_{\text{NIEL}^{-}} A$, тогда $\text{IEL}^{-} \vdash \bigwedge \Gamma \rightarrow \Box A$.

$\begin{array}{lll}
(1) & \bigwedge \Gamma \rightarrow A & \text{предположение индукции}\\
(2) & A \rightarrow \Box A &\text{ко-рефлексия}\\
(3) & (\bigwedge \Gamma \rightarrow A) \rightarrow ((A \rightarrow \Box A) \rightarrow (\bigwedge \Gamma \rightarrow \Box A))&\text{теорема IEL$^{-}$}\\
(4) & (A \rightarrow \Box A) \rightarrow (\bigwedge \Gamma \rightarrow \Box A) &\text{из (1), (3) и MP}\\
(5) & \bigwedge \Gamma \rightarrow \Box A &\text{из (2), (4) и MP}\\
\end{array}$

\vspace{\baselineskip}

2) Если $\Gamma \vdash_{\text{NIEL}^{-}} \Box \vec{A}$ и $\vec{A} \vdash B$, то $\text{IEL}^{-} \vdash \bigwedge \Gamma \rightarrow \Box B$.

$\begin{array}{lll}
(1) &\bigwedge \Gamma \rightarrow \bigwedge \limits_{i = 1}^{n} \Box A_i & \text{предположение индукции} \\
(2) &\bigwedge \limits_{i = 1}^{n} \Box A_i \rightarrow \Box \bigwedge \limits_{i = 1}^{n} A_i& \text{теорема IEL$^{-}$} \\
(3) &\bigwedge \Gamma \rightarrow \Box \bigwedge \limits_{i = 1}^{n} A_i & \text{по (1), (2) и правилу силлогизма} \\
(4) &\bigwedge \limits_{i = 1}^{n} A_i \rightarrow B& \text{предположение индукции} \\
(5) &(\bigwedge \limits_{i = 1}^{n} A_i \rightarrow B) \rightarrow \Box (\bigwedge \limits_{i = 1}^{n} A_i \rightarrow B)& \text{ко-рефлексия}\\
(6) &\Box (\bigwedge \limits_{i = 1}^{n} A_i \rightarrow B)& \text{из (4), (5) и MP} \\
(7) &\Box \bigwedge \limits_{i = 1}^{n} A_i \rightarrow \Box B & \text{по (6) и по нормальности} \\
(8) &\bigwedge \Gamma \rightarrow \Box B & \text{по (3), (7) и правилу силлогизма}
\end{array}$

\end{proof}

\begin{lemma}
$ $
Если $\text{IEL}^{-} \vdash A$, то $\text{NIEL}^{-} \vdash A$.
\end{lemma}

\begin{proof}
Построение выводов для модальных аксиом в $\text{NIEL}^{-}$. Мы рассмотрим эти выводы ниже с использованием термов.
\end{proof}

\vspace{\baselineskip}

\subsection{Модальное $\lambda$-исчисление $\lambda_{\bf K}$}

Далее мы построим типизированное $\lambda$-исчисление по фрагменту NIEL$^{-}$ с правилами для импликации, конъюнкции и модальности.
Данный фрагмент экивалентен IEL$^{-}$ без аксиом для отрицания и дизъюнкции, что элементарно проверяется аналогично.

Определим термы и типы:

\vspace{\baselineskip}

\begin{defin} Множество термов:

Пусть $\mathbb{V}$ счетное множество переменных. Термы $\Lambda_{{\bf K}}$ порождается следующей грамматикой:

$\begin{array}{lll}
& \Lambda_{{\bf K}} ::= \mathbb{V} \: | \:  (\lambda \mathbb{V}.\Lambda_{{\bf K}}) \: | \: (\Lambda_{{\bf K}}\Lambda_{{\bf K}}) \: | \: (\Lambda_{{\bf K}} , \Lambda_{{\bf K}}) \: | \: (\pi_1 \Lambda_{{\bf K}}) \: | \: (\pi_2 \Lambda_{{\bf K}}) \: | & \\
& \quad\quad\quad\quad\quad\quad\quad\quad\quad\quad\quad\quad\quad\quad\quad\quad ({\bf pure \: } \: \Lambda_{{\bf K}}) \: | \: ({\bf let \: pure \:} \mathbb{V}^{*} = \Lambda_{{\bf K}}^{*} {\: \bf in \:} \Lambda_{{\bf K}})
\end{array}$

\end{defin}

Где $\mathbb{V}^{*}$ и $\Lambda_{{\bf K}}^{*}$ обозначают множество всех конечных последовательностей переменных $\bigcup \limits_{i=0}^{\infty} \mathbb{V}^i$
и множество всех конечных последовательностей термов $\bigcup \limits_{i = 0}^{\infty} {\Lambda_{{\bf K}}}^i $. Последовательность переменных $\vec{x}$ и последовательность термов $\vec{M}$ в терме вида ${\bf let \: pure \:}$ должны иметь одинаковую длину.
Иначе терм не будет правильно построенным.

\begin{defin} Множество типов:

Пусть $\mathbb{T} = \{ A_1, A_2, A_3, \dots \}$ -- это счетное множество атормарных типов. Типы $\mathbb{T}_{{\bf K}}$ с типовым оператором $\Box$ порождаются следующей грамматикой:
\begin{equation}
  \mathbb{T}_{{\bf K}} ::= \mathbb{T} \: | \: (\mathbb{T}_{{\bf K}} \to \mathbb{T}_{{\bf K}}) \: |
  \: (\mathbb{T}_{{\bf K}} \times \mathbb{T}_{{\bf K}}) \: | \: (\Box \mathbb{T}_{{\bf K}})
\end{equation}
\end{defin}

Контекст, его домен и кодомен определены стандартно \cite{Neder}\cite{Morten}.

Наша система состоит из следующих правил типизации в стиле Карри:

\begin{defin} Модальное $\lambda$-исчисление, основанное на исчислении IEL$^{-}$:

  \begin{center}
  \begin{prooftree}
  \AxiomC{$ $}
  \RightLabel{\scriptsize{ax}}
  \UnaryInfC{$\Gamma , x : A \vdash x : A$}
  \end{prooftree}
  \end{center}

  \begin{minipage}{0.45\textwidth}
    \begin{prooftree}
    \AxiomC{$\Gamma, x : A \vdash M : B$}
    \RightLabel{$\rightarrow_i$}
    \UnaryInfC{$\Gamma \vdash \lambda x. M : A \to B$}
    \end{prooftree}

    \begin{prooftree}
    \AxiomC{ $\Gamma \vdash M : A$ }
    \AxiomC{ $\Gamma \vdash N : B$ }
    \RightLabel{$\times_i$}
    \BinaryInfC{$\Gamma \vdash \langle M, N \rangle : A \times B$}
    \end{prooftree}

    \begin{prooftree}
      \AxiomC{$\Gamma \vdash M : A$}
      \RightLabel{$\Box_I$}
      \UnaryInfC{$\Gamma \vdash {\bf pure \: } \: M : {\bf K}A $}
    \end{prooftree}
\end{minipage}%
\hfill
\begin{minipage}{0.45\textwidth}
\begin{tabular}{p{\textwidth}}
  \begin{prooftree}
  \AxiomC{$\Gamma \vdash M : A \to B$}
  \AxiomC{$\Gamma \vdash N : A$}
  \RightLabel{$\rightarrow_e$}
  \BinaryInfC{$\Gamma \vdash MN : B$}
  \end{prooftree}

  \begin{prooftree}
  \AxiomC{ $\Gamma \vdash M : A_1 \times A_2$ }
  \RightLabel{$\times_e$, $i \in \{ 1, 2 \}$}
  \UnaryInfC{$\Gamma \vdash \pi_i M : A_i$}
  \end{prooftree}

  \begin{prooftree}
    \AxiomC{$\Gamma \vdash \vec{M} : \Box \vec{A}$}
    \AxiomC{$\vec{x} : \vec{A} \vdash N : B$}
    \RightLabel{$\text{let}_{\Box}$}
    \BinaryInfC{$\Gamma \vdash {\bf let \: pure \:} \vec{x} = \vec{M} {\: \bf in \: } N : \Box B$}
  \end{prooftree}
\end{tabular}
\end{minipage}%

\end{defin}

Правило типизации $\Box$ аналогично правилу $\bigcirc_I$ в монадическом метаязыке \cite{Lax}.

$\Box_I$ позволяет вкладывать объект типа $A$ в текущий вычислительный контекст, изменяя его тип на $\Box A$.

Правило типизации $\text{let}_{\Box}$ аналогично правилу $\Box$-правилу в модальном $\lambda$-исчислении
для интуционистской минимальной нормальной модальной логики {\bf IK} \cite{ModalK}.

$\Gamma \vdash \vec{M} : \Box \vec{A}$ -- это синтаксический сахар для $\Gamma \vdash M_1 : \Box A_1,\dots,\Gamma \vdash M_n : \Box A_n$ и
$\vec{x} : \vec{A} \vdash N : B$ -- это краткая форма для $x_1 : A_1, \dots, x_n : A_n \vdash N : B$.
${\bf let \: pure \:} \vec{x} = \vec{M} {\: \bf in \: } N$ -- это мгновенное локальное связывание в терме $N$.
Мы будем использовать такую краткую форму вместо ${\bf let \: pure\:} x_1,\dots,x_n = M_1,\dots,M_n {\: \bf in \:} N$.

\vspace{\baselineskip}

Примеры выводов:

\begin{prooftree}
\AxiomC{$x : A \vdash x : A$}
\UnaryInfC{$x : A \vdash {\bf pure \:} x : \Box A$}
\UnaryInfC{$\vdash (\lambda x. {\bf pure \: } x) : A \to \Box A$}
\end{prooftree}

\begin{prooftree}
\AxiomC{$f : \Box (A \to B) \vdash f : \Box (A \to B)$}
\AxiomC{$x : \Box A \vdash x : \Box A $}
\AxiomC{$g : A \to B \vdash g : A \to B$}
\AxiomC{$y : A \vdash y : A$}
\RightLabel{$\to_e$}
\BinaryInfC{$g : A \to B, y : A \vdash g y : B$}
\RightLabel{$\text{let}_{{\bf K}}$}
\TrinaryInfC{$f : \Box (A \to B), x : \Box A \vdash {\bf let \: pure \:} g, y  = f, x {\: \bf in \:} g y : \Box B$}
\UnaryInfC{$f : \Box (A \to B) \vdash \lambda x. {\bf let \: pure\:} g, y = f, x {\: \bf in \:} g y : \Box A \to \Box B$}
\UnaryInfC{$\vdash \lambda f. \lambda x. {\bf let \:pure \:} g, y = f, x {\: \bf in \:} g y : \Box(A \to B) \to \Box A \to \Box B$}
\end{prooftree}

Нетрудно видеть, что данные примеры деревьев вывода являются в точности размеченными термами выводами модальных аксиом IEL$^{-}$.

\vspace{\baselineskip}

Определим свободные переменные, подставновку, $\beta$-редукцию и $\eta$-редукцию. Многошаговая $\beta$-редукция и $\beta \eta$-эквивалентность определены стандартно:

\begin{defin} Множество свободных переменных $FV(M)$ для произвольного терма $M$:

1) $FV(x) = \{ x \}$;

2) $FV(\lambda x. M) = FV(M) \setminus \{ x\}$;

3) $FV(M N) = FV(M) \cup FV(N);$

4) $FV(\langle M,N \rangle) = FV(M) \cup FV(N)$;

5) $FV(\pi_i M) \subseteq FV(M)$, $i \in \{ 1, 2\}$;

6) $FV(\text{pure } M) = FV(M)$;

7) $FV({\bf let \: pure} \: \vec{x} = \vec{M} \:\: {\bf in} \:\: N) = \bigcup \limits_{i = 1}^n FV(M), \text{где $n = |\vec{M}|$}$.
\end{defin}

Во избежание лишних коллизий мы будем полагать, что если терм вида ${\bf let \: pure \:} x = ({\bf let \: pure \:} \vec{y} = \vec{N} {\: \bf in \:} P) {\: \bf in \:} M$
типизируется, то $x$ не содержится в $\vec{y}$ (как следствие, свободные переменные термов $M$ и $\vec{N}$ не должны пересекаться),
то есть последовательное применение локальных связываний требует на каждом шаге различныx переменныx.

\begin{defin} Подстановка:

1) $x [x := N] = N$, $x [y := N] = x$;

2) $(M N) [x := N] = M[x := N] N [x := N]$;

3) $(\lambda x. M) [x := N] = \lambda x. M [y := N]$, $y \in FV(M)$;

4) $(M, N)[x := P] = (M[x := P], N [x := P])$;

5) $(\pi_i M) [x := P] = \pi_i (M[x := P])$, $i \in \{ 1, 2\}$;

6) $({\bf pure \: } M) [x := P] = {\bf pure \: } (M [x := P])$;

7) $({\bf let \: pure \:}\vec{x} = \vec{M} {\: \bf in \:} N) [y := P] = {\bf let \: pure \:} \vec{x} = (\vec{M} [y := P]) {\: \bf in \:} N$.
\end{defin}


\begin{defin} Правила $\beta$-редукции и $\eta$-редукции:

1) $(\lambda x. M) N \rightarrow_{\beta} M [x := N]$;

2) $\pi_1 \langle M, N \rangle \rightarrow_{\beta} M$;

3) $\pi_2 \langle M, N \rangle \rightarrow_{\beta} N$;

4) $\begin{array}{lll}
& {\bf let \: pure \:} \vec{x}, y, \vec{z} = \vec{M}, {\bf let \: pure \: } \vec{w} = \vec{N} {\: \bf in \: } Q, \vec{P} {\: in \:} R \rightarrow_{\beta \Box} & \\
& {\bf let \: pure \:} \vec{x}, \vec{w}, \vec{z} = \vec{M}, \vec{N}, \vec{P} {\: \bf in \: } R [y := Q]
\end{array}$

5) ${\bf let \: pure \:} \vec{x} = {\bf pure \:} \vec{M} {\: \bf in \:} N \rightarrow_{\beta \Box {\bf pure}} {\bf pure \:} N [\vec{x} := \vec{M}]$

6) ${\bf let \: pure \:} \underline{\quad} = \underline{\quad} {\: \bf in \:} M \rightarrow_{\beta {\bf nec}} {\bf pure \:} M$, где \underline{\quad} -- это пустая последовательность термов.

7) $\lambda x. f x \rightarrow_{\eta} f$;

8) $\langle \pi_1 P, \pi_2 P \rangle \rightarrow_{\eta} P$;

9) ${\bf let \: pure \:} x = M {\: \bf in \: } x \rightarrow_{\Box id} M$;

\end{defin}

Мы будет писать $M \rightarrow_{r} N$, если терм $M$ редуцируется к терму $N$ за один шаг по одному из перечисленных выше правил.

\begin{defin} Многошаговая редукция $\twoheadrightarrow_{r}$.

  Многошаговой редукцией $\twoheadrightarrow_{r}$ является рефлексивно-транзитивное замыкание одношаговой редукции $M \rightarrow_{r} N$.
\end{defin}

По умолчанию мы используем стратегию вычисления с вызовом по имени.

\subsection{Леммы о контекстах}

Докажем стандартные леммы о контекстах \footnote{Мы не будем рассматривать случаи для стандартных связок, так как они уже доказаны для просто типизированного $\lambda$-исчисления \cite{Neder} \cite{Morten}. Мы будем рассматривать только модальные случаи}:

\begin{lemma} Инверсия отношения типизации ${\Box}_I$.

Пусть $\Gamma \vdash {\bf pure \:} M : \Box A$, тогда $\Gamma \vdash M : A$;
\end{lemma}

\begin{proof}
Очевидно.
\end{proof}

\begin{lemma} Базовые леммы.

\begin{itemize}
\item Если $\Gamma \vdash M : A$ и $\Gamma \subseteq \Delta$, тогда $\Delta \vdash M : A$;
\item Если $\Gamma \vdash M : A$, тогда $\Delta \vdash M : A$, где $\Delta = \{ x_i : A_i \: | \: (x_i : A_i) \in \Gamma \: \& \: x_i \in FV(M) \}$
\item Если $\Gamma, x : A \vdash M : B$ и $\Gamma \vdash N : A$, где $\Gamma \vdash M [x := N] : B$.
\end{itemize}
\end{lemma}

Рассмотрим случаи для правила $\text{let}_{\Box}$.

\begin{proof}
$ $

1) Пусть вывод заканчивается следующим правилом:

\begin{prooftree}
\AxiomC{$\Gamma \vdash \vec{M} : \Box \vec{A}$}
\AxiomC{$\vec{x} : \vec{A} \vdash N : B$}
\RightLabel{$\text{let}_{\Box}$}
\BinaryInfC{$\Gamma \vdash {\bf let \: pure \:} \vec{x} = \vec{M} {\: \bf in \: } N : \Box B$}
\end{prooftree}

По предположению индукции $\Delta \vdash \vec{M} : {\bf K} \vec{A}$, тогда $\Delta \vdash {\bf let \: pure \:} \vec{x} = \vec{M} {\: \bf in \: } N : \Box B$.

\vspace{\baselineskip}

Случаи 2)--3) рассматриваются аналогично.

\end{proof}

\subsection{Метатеоретические свойства системы}

\begin{theorem} Редукция субъекта

Если $\Gamma \vdash M : A$ и $M \twoheadrightarrow_{\beta \eta} N$, тогда $\Gamma \vdash N : A$

\end{theorem}

\begin{proof}

Индукция по выводу $\Gamma \vdash M : A$ и по порождению $\rightarrow_{\beta \eta}$.

Случаи с функцией и парами рассмотрены здесь \cite{Morten} \cite{Pierce}.

1) Если $\Gamma \vdash {\bf let \: pure \:} \vec{x}, y, \vec{z} = \vec{M}, {\bf let \: pure \: } \vec{w} = \vec{N} {\bf \: in \: } Q, \vec{P}  {\: \bf in \:} R : \Box B$,
тогда $\Gamma \vdash {\bf let \: pure \:} \vec{x}, \vec{w}, \vec{z} = \vec{M}, \vec{N}, \vec{P} \: { \bf in } \: R [y := Q] : \Box B$ по правилу 4).

2) Если $\Gamma \vdash {\bf let \: pure \:} x = M {\: \bf in \: } x : \Box A$, тогда $\Gamma \vdash M : {\bf K}A$ по правилу 9).

Рассмотрено здесь \cite{ModalK}.

3) Пусть вывод заканчивается применением следующего правила

\begin{prooftree}
\AxiomC{$\Gamma \vdash {\bf pure \:} \vec{M} : \Box \vec{A}$}
\AxiomC{$\vec{x} : \vec{A} \vdash N : B$}
\BinaryInfC{$\Gamma \vdash {\bf let \: pure \:} \vec{x} = {\bf pure \:} \vec{M} {\: \bf in \:} N : \Box B$}
\end{prooftree}

Тогда $\Gamma \vdash \vec{M} : \vec{A}$ по инверсии отношения типизации для $\Box_I$ и $\Gamma \vdash N [\vec{x} := \vec{M}] : B$ по лемме 4, часть 3.

Тогда мы можем преобразовать данный вывод в следующий:

\begin{prooftree}
\AxiomC{$\Gamma \vdash N [\vec{x} := \vec{M}] : B$}
\RightLabel{$\Box_I$}
\UnaryInfC{$\Gamma \vdash {\bf pure \:} N [\vec{x} := \vec{M}] : \Box B$}
\end{prooftree}

4) Пусть вывод заканчивается применением правила ${\text let}_{\Box}$ для типового объявления, выводимого из пустого контекста:

\begin{prooftree}
\AxiomC{$\vdash M : A$}
\UnaryInfC{$\vdash {\bf let \: pure \:} \underline{\quad} = \underline {\quad} {\: \bf in \:} M : \Box A$}
\end{prooftree}

Тогда, если $\vdash M : A$, тогда $\vdash {\bf pure \:} M : \Box A$.

Данное рассуждение действует также и в обратную сторону.
\end{proof}

\begin{theorem}
$ $

$\twoheadrightarrow_{r}$ сильно нормализуемо;
\end{theorem}

\begin{proof}
$ $

Построим отображение из $\lambda_{\bf K}$
в просто типизированное $\lambda$-исчисление с типами $\to$, $\times$ и выделенным типом натуральных чисел $\mathbb{N}$, для которого есть
следующие правила типизации и редукции:

\begin{prooftree}
\AxiomC{$ $}
\UnaryInfC{$\Gamma \vdash 0 : \mathbb{N}$}
\end{prooftree}

\begin{prooftree}
\AxiomC{$\Gamma \vdash n : \mathbb{N}$}
\UnaryInfC{$\Gamma \vdash {\bf succ \:} n : \mathbb{N}$}
\end{prooftree}

\begin{prooftree}
\AxiomC{$\Gamma \vdash n : \mathbb{N}$}
\AxiomC{$\Gamma \vdash m : \mathbb{N}$}
\BinaryInfC{$\Gamma \vdash n + m : \mathbb{N}$}
\end{prooftree}

\begin{itemize}
  \item $n + 0 \rightarrow_{\beta} n$;
  \item $(n + {\bf succ \:} m) \rightarrow_{\beta} {\bf succ \:} (n + m)$
\end{itemize}

Определим перевод $|.|$ между данными исчислениями отдельно на типах, и на термах

\begin{defin} Интерпретация типов

  \begin{itemize}
    \item $A \in \mathbb{T} \Rightarrow |A| = A$;
    \item $|A \to B| = |A| \to |B|$;
    \item $|A \times B| = |A| \times |B|$;
    \item $|\Box A| = \mathbb{N} \times |A|$.
  \end{itemize}
\end{defin}

\begin{defin} Интерпретация термов
  \begin{itemize}
    \item $x \in \mathbb{V} \Rightarrow |x| = x$;
    \item $|\lambda x. M| = \lambda x. |M|$;
    \item $|(M N)| = |M| |N|$;
    \item $|\langle M, N \rangle| = \langle |M|, |N| \rangle$;
    \item $|\pi_i M| = \pi_i |M|$, $i \in \{ 1, 2\}$;
    \item $|{\bf pure \:} M| = \langle 0, |M| \rangle$;
    \item $|{\bf let \: pure \:} \underline{\quad} = \underline{\quad} {\: \bf in \:} M| = \langle 0, M \rangle$
    \item $|{\bf let \: pure \:} \vec{x} = \vec{N} {\: \bf in \:} M| = \langle \sum \limits_{i = 1}^n \pi_1 |N|, |M| [\vec{x} := \pi_2 \vec{N}] \rangle$
  \end{itemize}
\end{defin}

Рассмотрим интерпретацию последнего терма с помощью интерпретации правила типизации:

\begin{prooftree}
  \AxiomC{$|\Gamma \vdash \vec{N} : \Box \vec{A}| = |\Gamma| \vdash \vec{|N|} : \mathbb{A} \times \vec{|A|}$}
  \AxiomC{$|\vec{x} : \vec{A} \vdash M : B| = \vec{x} : \vec{|A|} \vdash |M| : |B|$}
  \RightLabel{$\text{let}_{\Box}$}
  \BinaryInfC{$|\Gamma \vdash {\bf let \: pure \:} \vec{x} = \vec{N} {\: \bf in \: } M : \Box B| = |\Gamma| \vdash \langle \sum \limits_{i = 1}^n \pi_1 |N|, |M| [\vec{x} := \pi_2 \vec{N}] \rangle : \mathbb{N} \times |B|$}
\end{prooftree}

\begin{lemma} Интерпретация сохраняет подстановку:

  $|M [x := N]| = |M| [x := |N|]$ для произвольного терма $M$.
\end{lemma}

\begin{proof}

  Несложная индукция по длине $M$.
\end{proof}

\begin{lemma}

  $M \twoheadrightarrow_{r} N \Rightarrow |M| \twoheadrightarrow_{\beta\eta} |N|$
\end{lemma}

\begin{proof}

  Рассмотрим случаи c $\beta \Box$, $\beta \Box {\bf pure}$ и $\Box id$.

1)

$\begin{array}{lll}
&|{\bf let \: pure \:} x = ({\bf let \: pure \:} y = N {\: \bf in \:} P) {\: \bf in \:} M| = & \\
&\quad\quad\quad\quad\quad\quad\quad \text{По интерпретации}& \\
&\langle \pi_1 |N|, |M| [x := |P| [y := \pi_2 |N|]] \rangle& \\
&|{\bf let \: pure \:} y = N {\: \bf in \:} M [x := P]| = & \\
&\langle \pi_1 |N|, |M| [x := |P|] [y := \pi_2 |N|]\rangle& \equiv \\
&\quad\quad\quad\quad\quad\quad\quad \text{По лемме Барендрегта по подстановке}& \\
&\langle \pi_1 |N|, |M| [y := \pi_2 |N|] [x := |P| [y := \pi_2 |N|]] \rangle& \equiv \\
&\quad\quad\quad\quad\quad\quad\quad \text{Поскольку $y \notin FV(M)$} & \\
&\langle \pi_1 |N|, |M| [x := |P| [y := \pi_2 |N|]]&
\end{array}$

2)

$\begin{array}{lll}
&|{\bf let \: pure \:} \vec{x} = {\bf pure \:} \vec{N} {\: \bf in \:} M| = & \\
&\quad\quad\quad\quad\quad\quad\quad \text{По интерпретации}& \\
&\langle 0 + \dots + 0, |M| [\vec{x} := \vec{|N|}]\rangle \twoheadrightarrow_{\beta}& \\
&\quad\quad\quad\quad\quad\quad\quad \text{Многошаговая редукция для натуральных чисел}& \\
&\langle 0, |M| [\vec{x} := \vec{|N|}] \rangle = & \\
&\quad\quad\quad\quad\quad\quad\quad \text{По интерпретации}& \\
&|{\bf pure \:} M [\vec{x} := \vec{N}]|&
\end{array}$

3)

$\begin{array}{lll}
&|{\bf let \: pure \:} x = M {\: \bf in \:} x| = & \\
&\quad\quad\quad\quad\quad\quad\quad \text{По интерпретации}& \\
&\langle \pi_1 |M|, x [x := \pi_2 |M|] \rangle = & \\
&\quad\quad\quad\quad\quad\quad\quad \text{Подстановка}& \\
&\langle \pi_1 |M|, \pi_2 |M| \rangle \rightarrow_{\eta}& \\
&\quad\quad\quad\quad\quad\quad\quad \text{Правило $\eta$-редукции для пары}& \\
&|M|&
\end{array}$
\end{proof}

Таким образом, мы показали, что $\lambda_{\bf K}$ корректно относительно $\lambda_{\to, \times, \mathbb{N}}$,
тогда $\lambda_{\bf K}$ сильно нормализуемо, поскольку $\lambda_{\to, \times, \mathbb{N}}$ сильно нормализуемо.

\end{proof}

\begin{theorem} Свойство Черча-Россера
$ $

$\twoheadrightarrow_{r}$ конфлюентно.
\end{theorem}

\begin{proof}

  По лемме Ньюмана, если отношение сильно нормализуемо и локально конфлюентно, то отношение конфлюентно.

Достаточно показать локальную конфлюентность.

\begin{lemma} Локальная конфлюентность.

Если $M \rightarrow_{r} N$ и $M \rightarrow_{r} Q$, тогда найдется такой терм $P$,
что $N \twoheadrightarrow_{r} P$ и $Q \twoheadrightarrow_{r} P$.

\end{lemma}

\begin{proof}

Рассмотрим данную критическую пару и покажем, что оба терма из данной пары редуцируются к одному и тому же терму:

\xymatrix{
{\bf let \: pure \:} x = ({\bf let \: pure \:} \vec{y} = {\bf pure \:} \vec{N} {\: \bf in \:} P) {\: \bf in \:} M \ar[d]_{\beta \Box} \ar[dr]^{\beta \Box {\bf pure}} \\
{\bf let \: pure \:} \vec{y} = {\bf pure \:} \vec{N} {\: \bf in \:} M [x := P] & {\bf let \: pure \:} x = {\bf pure \:} P [\vec{y} := \vec{N}] {\: \bf in \:} M
}

$\begin{array}{lll}
&{\bf let \: pure \:} \vec{y} = {\bf pure \:} \vec{N} {\: \bf in \:} M [x := P] \rightarrow_{\beta \Box {\bf pure}}& \\
&\quad\quad\quad\quad\quad\quad\quad {\bf pure \:} M [x := P] [\vec{y} = \vec{N}]& \\
&{\bf let \: pure \:} x = {\bf pure \:} P [\vec{y} := \vec{N}] {\: \bf in \:} M \rightarrow_{\beta \Box {\bf pure}}& \\
&\quad\quad\quad\quad\quad\quad\quad {\bf pure \:} M [x := P[\vec{y} := \vec{N}]]& \\
&\text{ }&\\
&\text{По лемме о подстановке}& \\
&{\bf pure \:} M [x := P] [\vec{y} = \vec{N}] \equiv {\bf pure \:} M [\vec{y} = \vec{N}] [x := P[\vec{y} := \vec{N}]]& \\
&\text{По нашему соглашению, $x \notin \vec{y}$, тогда}& \\
&M[\vec{y} = \vec{N}] [x := P[\vec{y} := \vec{N}]] \equiv M [x := P[\vec{y} := \vec{N}]]& \\
\end{array}$
\end{proof}


\end{proof}

\begin{theorem}
$ $

Нормальная форма $\lambda_{{\bf K}}$ со стратегией вычисления с вызовом по имени обладает свойством подформульности: если $M$ в нормальной форме, то всего его подтермы также в нормальной форме.

\end{theorem}

\begin{proof}
Индукция по структуре $M$.

Случай ${\bf let \: pure\:} \vec{x} = \vec{M} {\: \bf in \:} N$ рассмотрен Какутани \cite{ModalK} \cite{ModalK1}.

Пусть ${\bf pure \:} M$ в нормальной форме, тогда $M$ в нормальной форме и все его подтермы также в нормальной форме по предположению индукции.

Тогда, если ${\bf pure \:} M$ в нормальной форме, то и все его подтермы также в нормальной форме.
\end{proof}


\newpage

\section{Теоретико-категорная семантика системы типов $\lambda_{\bf K}$}

\subsection{Корректность}

\begin{theorem} Корректность

  Пусть $\Gamma \vdash M : A$ и $M =_{\beta\eta} N$, тогда $[\![\Gamma \vdash M : A]\!] = [\![\Gamma \vdash N : A]\!]$
\end{theorem}

\begin{proof}

\begin{defin} Семантическая трансляция из $\lambda_{{\bf K}}$ в аппликативный функтор $\langle \mathcal{C}, \boxdot, \eta \rangle$ над декартово замкнутой категорией $\mathcal{C}$,
где $\boxdot$ -- это моноидальный эндофунктор и $\eta$ -- это естественное преобразование $Id_{\mathcal{C}} \Rightarrow \boxdot$:

\begin{itemize}
\item Интерпретация типов:
  \begin{itemize}
    \item $[\![A]\!] := \hat{A}, A \in \mathbb{T}$, где $\hat{A}$ -- это объект категории $\mathcal{C}$, полученный в результате некоторого присваивания;
    \item $[\![A \to B]\!] := [\![B]\!]^{[\![A]\!]}$;
    \item $[\![A \times B]\!] := [\![A]\!] \times [\![B]\!]$.
  \end{itemize}
\item Интерпретация для модальных типов:
  \begin{itemize}
    \item $[\![\Box A]\!] = \boxdot[\![A]\!]$;
  \end{itemize}
\item Интерпретация для контекстов:
  \begin{itemize}
    \item $[\![ \quad ]\!] = \mathds{1}$, где $\mathds{1}$ -- это терминальный объект в заданной декартово замкнутой категории;
    \item $[\![\Gamma, x : A]\!] = [\![\Gamma]\!] \times [\![A]\!]$
  \end{itemize}
\item Интерпретация для типовых объявлений:
  \begin{itemize}
    \item $[\![\Gamma \vdash M : A]\!] := [\![M]\!] : [\![\Gamma]\!] \to [\![A]\!]$.
  \end{itemize}
\item Интерпретация для правил типизации:

\begin{prooftree}
\AxiomC{$ $}
\UnaryInfC{$[\![\Gamma, x : A \vdash x : A]\!] = \pi_2 : [\![\Gamma]\!] \times [\![A]\!] \rightarrow
[\![A]\!]$}
\end{prooftree}

\begin{prooftree}
\AxiomC{$[\![\Gamma, x : A \vdash M : B]\!] = [\![M]\!] : [\![\Gamma]\!] \times [\![A]\!] \rightarrow [\![B]\!]$}
\UnaryInfC{$[\![\Gamma \vdash (\lambda x. M) : A \to B]\!] = \Lambda([\![M]\!]) : [\![\Gamma]\!]
\rightarrow[\![B]\!]^{[\![A]\!]}$}
\end{prooftree}

\begin{prooftree}
\AxiomC{$[\![\Gamma \vdash M : A \to B]\!] = [\![M]\!] : [\![\Gamma]\!] \rightarrow [\![B]\!]^{[\![A]\!]}$}
\AxiomC{$[\![\Gamma \vdash N : A]\!] = [\![N]\!] : [\![\Gamma]\!] \rightarrow [\![A]\!]$}
\BinaryInfC{$[\![\Gamma \vdash (M N) : B]\!] = [\![\Gamma]\!] \xrightarrow{\langle [\![M]\!], [\![N]\!]
\rangle} [\![B]\!]^{[\![A]\!]} \times [\![A]\!] \xrightarrow{\epsilon} [\![B]\!] $}
\end{prooftree}

\begin{prooftree}
\AxiomC{$[\![\Gamma \vdash M : A ]\!] = [\![M]\!] : [\![\Gamma]\!] \rightarrow [\![A]\!]$}
\AxiomC{$[\![\Gamma \vdash N : B ]\!] = [\![N]\!] : [\![\Gamma]\!] \rightarrow [\![B]\!]$}
\BinaryInfC{$[\![\Gamma \vdash \langle M, N \rangle : A \times B]\!] = \langle [\![M]\!], [\![N]\!] \rangle : [\![\Gamma]\!] \rightarrow
[\![A]\!] \times [\![B]\!]$}
\end{prooftree}

\begin{prooftree}
\AxiomC{$[\![\Gamma \vdash M : A_1 \times A_2]\!] = [\![M]\!] : [\![\Gamma]\!] \rightarrow [\![A_1]\!] \times
[\![A_2]\!]$}
\RightLabel{$i \in \{1,2\}$}
\UnaryInfC{$[\![\Gamma \vdash \pi_i M : A_i]\!] = [\![\Gamma]\!] \xrightarrow{[\![M]\!]} [\![A_1]\!] \times
[\![A_2]\!] \xrightarrow{\pi_i} [\![A_i]\!]$}
\end{prooftree}

\begin{prooftree}
\AxiomC{$[\![\Gamma \vdash M : A]\!] = [\![M]\!] : [\![\Gamma]\!] \rightarrow [\![A]\!]$}
\UnaryInfC{$[\![\Gamma \vdash {\bf pure \:} M : \Box A]\!] := [\![\Gamma]\!] \xrightarrow{[\![M]\!]}
[\![A]\!] \xrightarrow{\eta_{[\![A]\!]}} \boxdot[\![A]\!]$}
\end{prooftree}

\begin{small}
  \begin{prooftree}
    \AxiomC{$[\![\Gamma \vdash \vec{M} : \Box \vec{A}]\!] = \langle [\![M_1]\!],\dots, [\![M_n]\!] \rangle : [\![\Gamma]\!] \rightarrow \prod \limits_{i=1}^n \boxdot [\![A_i]\!]$}
    \AxiomC{$[\![\vec{x} : \vec{A} \vdash N : B]\!] = [\![N]\!] : \prod \limits_{i=1}^n [\![A_i]\!] \rightarrow [\![B]\!]$}
    \BinaryInfC{$[\![\Gamma \vdash {\bf let \: pure \:} \vec{x} = \vec{M} {\: \bf in \: } M : \Box B]\!] = \boxdot ([\![N]\!]) \circ ([\![A_1]\!] \ast \dots \ast [\![A_n]\!]) \circ \langle [\![M_1]\!],\dots, [\![M_n]\!] \rangle : [\![\Gamma]\!] \rightarrow \boxdot [\![B]\!]$}
  \end{prooftree}
\end{small}
\end{itemize}
\end{defin}

Для большей ясности проиллюстрируем интерпретацию последнего правила коммутативной диаграммой:

\xymatrix{
&&& [\![\Gamma]\!] \ar[dd]_{\langle [\![M_1]\!],\dots, [\![M_n]\!] \rangle} \ar[rrrrr]^{\boxdot ([\![N]\!]) \circ ([\![A_1]\!] \ast \dots \ast [\![A_n]\!]) \circ \langle [\![M_1]\!],\dots, [\![M_n]\!] \rangle} &&&&& \boxdot [\![B]\!]\\
\\
&&& \prod \limits_{i=1}^n \boxdot [\![A_i]\!] \ar[rrrrr]_{[\![A_1]\!] \ast \dots \ast [\![A_n]\!]} &&&&& \boxdot \prod \limits_{i=1}^n [\![A_i]\!] \ar[uu]_{\boxdot ([\![N]\!])}
}

\begin{defin} Одновременная подстановка

Пусть $\Gamma = \{ x_1 : A_1, ..., x_n : A_n \}$, $\Gamma \vdash M : A$ и для любых $i \in \{ 1,..., n \}$,
$\Gamma \vdash M_i : A_i$.

 Одновременная подстановка $M [ \vec{x} := \vec{M}]$ определяется рекурсивно:

\begin{itemize}
\item $x_i [ \vec{x} := \vec{M}] = M_i $;
\item $(\lambda x. M) [ \vec{x} := \vec{M}] = \lambda x. (M [ \vec{x} := \vec{M}])$;
\item $(M N) [ \vec{x} := \vec{M}] = (M [ \vec{x} = \vec{M}]) (N [ \vec{x} := \vec{M}])$;
\item $\langle M, N \rangle = \langle (M [ \vec{x} = \vec{M}]), (N [ \vec{x} := \vec{M}])\rangle$;
\item $(\pi_i P) [ \vec{x} := \vec{M}] = \pi_i (P [ \vec{x} = \vec{M}])$;
\item $({\bf pure \:} M) [ \vec{x} := \vec{M}] = {\bf pure \:} (M [ \vec{x} = \vec{M}])$;
\item $({\bf let \: pure} \: \vec{x} = \vec{M} {\: \bf in \:} N) [\vec{y} := \vec{P}] =
{\bf let \: pure} \: \vec{x} = (\vec{M} [\vec{y} := \vec{P}]) {\: \bf in \:} N$
\end{itemize}
\end{defin}

\begin{lemma}
$ $

$[\![M [x_1 := M_1,\dots, x_n := M_n]]\!] = [\![M]\!] \circ \langle [\![M_1]\!], \dots, [\![M_n]\!] \rangle$.

\end{lemma}

\begin{proof}

$ $

1)

$\begin{array}{lll}
& [\![\Gamma \vdash ({\bf pure \:} M) [ \vec{x} := \vec{M}] : \Box A]\!] = & \\
&\quad\quad\quad\quad\quad\quad\quad \text{Определение мгновенной подстановки}& \\
&[\![ \Gamma \vdash {\bf pure \:} (M [ \vec{x} := \vec{M}]) : \Box A]\!]& \\
&\quad\quad\quad\quad\quad\quad\quad \text{Интерпретация для {\bf pure}}\\
&\eta_{[\![A]\!]} \circ [\![(M [ \vec{x} := \vec{M}])]\!]& \\
&\quad\quad\quad\quad\quad\quad\quad \text{Предположение индукции} *\\
&\eta_{[\![A]\!]} \circ ([\![M]\!] \circ \langle [\![M_1]\!], \dots, [\![M_n]\!] \rangle) = & \\
&\quad\quad\quad\quad\quad\quad\quad \text{Ассоциативность композиции}&\\
&(\eta_{[\![A]\!]} \circ [\![M]\!]) \circ \langle [\![M_1]\!], \dots, [\![M_n]\!] \rangle = & \\
&\quad\quad\quad\quad\quad\quad\quad  \text{Интерпретация для {\bf pure}}& \\
&[\![ \Gamma \vdash {\bf pure \:} M : \Box A]\!] \circ \langle [\![M_1]\!], \dots, [\![M_n]\!] \rangle = & \\
\end{array}$

\vspace{\baselineskip}

2)

\vspace{\baselineskip}

$\begin{array}{lll}
&[\![\Gamma \vdash ({\bf let \: pure} \: \vec{x} = \vec{M} {\: \bf in \:} N) [\vec{y} := \vec{P}] : \Box B]\!] =& \\
&\quad\quad\quad\quad\quad\quad\quad \text{Определение одновременной подстановки} &\\
&[\![\Gamma \vdash {\bf let \: pure} \: \vec{x} = (\vec{M} [\vec{y} := \vec{P}]) {\: \bf in \:} N : \Box B]\!] =& \\
&\quad\quad\quad\quad\quad\quad\quad  \text{Интерпретация $let_{\Box}$} \\
&\boxdot ([\![N]\!]) \circ ([\![A_1]\!] \ast \dots \ast [\![A_n]\!]) \circ [\![\Gamma \vdash (\vec{M} [\vec{y} := \vec{P}]) : \Box \vec{A}]\!] =& \\
&\quad\quad\quad\quad\quad\quad\quad \text{Предположение индукции}& \\
&\boxdot ([\![N]\!]) \circ ([\![A_1]\!] \ast \dots \ast [\![A_n]\!]) \circ ([\![\vec{M}]\!] \circ \langle [\![P_1]\!],\dots,[\![P_n]\!]\rangle) = & \\
&\quad\quad\quad\quad\quad\quad\quad \text{Ассоциативность композиции}& \\
&(\boxdot ([\![N]\!]) \circ ([\![A_1]\!] \ast \dots \ast [\![A_n]\!]) \circ [\![\vec{M}]\!]) \circ \langle [\![P_1]\!],\dots,[\![P_n]\!]\rangle = & \\
&\quad\quad\quad\quad\quad\quad\quad \text{По интерпретации}& \\
&[\![\Gamma \vdash {\bf let \: pure} \: \vec{x} = \vec{M} {\: \bf in \:} N : \Box B]\!] \circ \langle [\![P_1]\!],\dots,[\![P_n]\!]\rangle&
\end{array}$

\end{proof}

\begin{lemma}
  $ $

  Пусть $\Gamma \vdash M : A$ и $M \twoheadrightarrow_{\beta \eta} N$, тогда $[\![\Gamma \vdash M : A]\!] = [\![\Gamma \vdash N : A]\!]$;
\end{lemma}

\begin{proof}
  $ $

Случаи с правилом $\beta$-редукции для $let_{\Box}$ рассмотрены здесь \cite{ModalK1}. Рассмотрим случаи с ${\bf pure}$.

\vspace{\baselineskip}

1) $[\![\Gamma \vdash {\bf let \: pure \:} \vec{x} = {\bf pure \:} \vec{M} {\: \bf in \:} N : \Box B]\!] = [\![\Gamma \vdash {\bf pure \:} N [\vec{x} := \vec{M}] : \Box B]\!]$

\vspace{\baselineskip}

$\begin{array}{lll}
&[\![\Gamma \vdash {\bf let \: pure \:} \vec{x} = {\bf pure \:} \vec{M} {\: \bf in \:} N : \Box B]\!] = & \\
&\text{\quad\quad\quad\quad\quad\quad Интепретация}& \\
&\boxdot ([\![N]\!]) \circ ([\![A_1]\!] \ast \dots \ast [\![A_n]\!]) \circ \langle \eta_{[\![A_1]\!]} \circ [\![M_1]\!],\dots,\eta_{[\![A_n]\!]} \circ [\![M_n]\!] \rangle = \\
&\text{\quad\quad\quad\quad\quad\quad Свойство произведения морфизмов}& \\
&\boxdot ([\![N]\!]) \circ ([\![A_1]\!] \ast \dots \ast [\![A_n]\!]) \circ (\eta_{[\![A_1]\!]} \times \dots \times \eta_{[\![A_n]\!]}) \circ \langle [\![M_1]\!], \dots, [\![M_n]\!]\rangle =& \\
&\text{\quad\quad\quad\quad\quad\quad Ассоциативность композиции}& \\
&\boxdot ([\![N]\!]) \circ (([\![A_1]\!] \ast \dots \ast [\![A_n]\!]) \circ (\eta_{[\![A_1]\!]} \times \dots \eta_{[\![A_n]\!]})) \circ \langle [\![M_1]\!], \dots, [\![M_n]\!] \rangle =& \\
&\text{\quad\quad\quad\quad\quad\quad По определению аппликативного функтора}& \\
&\boxdot ([\![N]\!]) \circ \eta_{[\![A_1]\!] \times \dots \times [\![A_n]\!]} \circ \langle [\![M_1]\!], \dots, [\![M_n]\!] \rangle =& \\
&\text{\quad\quad\quad\quad\quad\quad Естественность $\eta$}& \\
&\eta_{[\![B]\!]} \circ [\![N]\!] \circ \langle [\![M_1]\!], \dots, [\![M_n]\!] \rangle =& \\
&\text{\quad\quad\quad\quad\quad\quad Ассоциативность композиции}& \\
&\eta_{[\![B]\!]} \circ ([\![N]\!] \circ \langle [\![M_1]\!], \dots, [\![M_n]\!]) \rangle =& \\
&\text{\quad\quad\quad\quad\quad\quad По лемме об одновременной подстановке}& \\
&\eta_{[\![B]\!]} \circ [\![N [\vec{x} := \vec{M}]]\!]& \\
&\text{\quad\quad\quad\quad\quad\quad Интерпретация}& \\
&[\![\Gamma \vdash {\bf pure \:} (N [\vec{x} := \vec{M}]) : \Box B]\!]&
\end{array}$

\vspace{\baselineskip}

2) $[\![\vdash {\bf let \: pure \:} \underline{\quad} = \underline {\quad} {\: \bf in \:} M : \Box A]\!] = [\![\vdash {\bf pure \:} M : \Box A]\!]$

$\begin{array}{lll}
&[\![\vdash {\bf let \: pure \:} \underline{\quad} = \underline {\quad} {\: \bf in \:} M : \Box A]\!] = & \\
&\quad\quad\quad\quad\quad\quad \text{Интерпретация}& \\
&\boxdot ([\![M]\!]) \circ u_{\mathds{1}} = & \\
&\quad\quad\quad\quad\quad\quad \text{Определение аппликативного функтора}& \\
&\boxdot ([\![M]\!]) \circ \eta_{\mathds{1}} = & \\
&\quad\quad\quad\quad\quad\quad \text{Естественность $\eta$} & \\
&\eta_{[\![A]\!]} \circ [\![M]\!] = &\\
&\quad\quad\quad\quad\quad\quad \text{Интерпретация} & \\
&[\![\vdash {\bf pure \:} M : \Box A]\!]&
\end{array}$
\end{proof}

\end{proof}

\subsection{Полнота}

\begin{theorem} Полнота

Пусть $[\![\Gamma \vdash M : A]\!] = [\![\Gamma \vdash N : A]\!]$, тогда $M =_{\beta \eta} N$.
\end{theorem}

\begin{proof}

$ $

Мы будем работать с термовой моделью для простого типизированного $\lambda$-исчисления с $\times$ и $\to$, стандартно описанной здесь \cite{LambekScott}:

\begin{defin} Эквивалетность на парах вида переменная-терм:
  $ $

  Определим такое бинарное отношение  $\sim_{A, B} \subseteq \mathbb{V} \times \Lambda_{{\bf K}}$, что:

  $(x, M) \sim_{A, B} (y, N) \Leftrightarrow x : A \vdash M : B \:\: \& \:\: y : A \vdash N : A \:\: \& \:\: M =_{\beta \eta} N [y := x]$.
\end{defin}

Нетрудно заметить, что данное отношение является отношением эквивалентности.

Обозначим класс эквивалентности как $[x, M]_{A, B} = \{ (y, N) \: | \: (x, M) \sim_{A, B} (y, N) \}$
(ниже мы будем опускать индексы).


\begin{defin} Категория $\mathcal{C}(\lambda)$:
\begin{itemize}
  \item $Ob_{\mathcal{C}} = \{ \hat{A} \: | \: A \in \mathbb{T} \} \cup \{ \mathds{1} \}$;
  \item $Hom_{\mathcal{C}(\lambda)}(\hat{A},\hat{B}) = (\mathbb{V} \times \Lambda_{{\bf K}})/_{\sim_{A, B}}$, где $(\mathbb{V} \times \Lambda_{{\bf K}})/_{\sim_{A, B}}$ -- это фактор-множество по отношению $A, B$ эквивалетности.
  Иными словами, множество стрелок из $\hat{A}$ в $\hat{B}$ определены выводимостями $x : A \vdash M : B$ с точностью до $A, B$ эквивалентности;
  \item Пусть $[x, M] \in Hom_{\mathcal{C}(\lambda)}(\hat{A},\hat{B})$ и $[y,N] \in Hom_{\mathcal{C}(\lambda)}(\hat{B},\hat{C})$, тогда $[y,M] \circ [x, M] = [x, N [y := M]]$;
  \item Тождественный морфизм $id_{\hat{A}} = [x,x] \in Hom_{\mathcal{C}(\lambda)(\hat{A})}$;
  \item Терминальный объект $\mathds{1}$;
  \item $\widehat{A \times B} = \hat{A} \times \hat{B}$;
  \item Каноническая проекция: $[x, \pi_i x] \in Hom_{\mathcal{C}(\lambda)}(\hat{A_1} \times \hat{A_2},\hat{A_i})$ for $i \in \{ 1, 2 \}$;
  \item $\widehat{A \to B} = \hat{B}^{\hat{A}}$;
  \item Вычисляющая стрелка $\epsilon = [x, (\pi_2 x) (\pi_1 x)] \in Hom_{\mathcal{C}(\lambda)(\hat{B}^{\hat{A}} \times \hat{A}, \hat{B})}$.
\end{itemize}
\end{defin}

\begin{defin}
  Определим эндофунктор $\boxdot : \mathcal{C}(\lambda) \to \mathcal{C}(\lambda)$ таким образом, что
для любых $[x,M] \in Hom_{\mathcal{C}(\lambda)}(\hat{A},\hat{B}), \boxdot ([x,M]) = [y, {\bf let \: pure \:} x = y {\: \bf in \:} M] \in Hom_{\mathcal{C}(\lambda)}(\boxdot \hat{A}, \boxdot \hat{B})$
(обозначения: $\text{fmap } f$ для произвольной стрелки $f$).

\end{defin}

Достаточно показать, что $\boxdot$ -- это аппликативный функтор над $\mathcal{C}(\lambda)$.

\begin{lemma} Функториальность

\begin{itemize}
  \item $\text{fmap }(g \circ f) = \text{fmap }(g) \circ \text{fmap }(f)$;
  \item $\text{fmap }(id_{\hat{A}}) = id_{\boxdot \hat{A}}$.
\end{itemize}
\end{lemma}

\begin{proof}

$ $

1)

$\begin{array}{lll}
&\text{fmap }(g \circ f) = \text{fmap}([y, N] \circ [x, M]) = & \\
&\quad\quad\quad\quad\quad\quad \text{По определению композиции}&\\
&\text{fmap }([x, N [y := M]]) =& \\
&\quad\quad\quad\quad\quad\quad \text{По определению fmap}& \\
&[z, {\bf let \: pure \:} x = z {\: \bf in \:} N [y := M]]& \\
& & \\
&\text{fmap }(g) \circ \text{fmap }(f) = \text{fmap }([y, N]) \circ \text{fmap }([x, M]) =& \\
&\quad\quad\quad\quad\quad\quad \text{По определению fmap}& \\
&[y_1, {\bf let \: pure \:} y = y_1 {\: \bf in \:} N] \circ [z, {\bf let \: pure \:} x = z {\: \bf in \:} M] =& \\
&\quad\quad\quad\quad\quad\quad \text{По определению композиции}&\\
&[z, {\bf let \: pure \:} y = y_1 {\: \bf in \:} N [y_1 := {\bf let \: pure \:} x = z {\: \bf in \:} M]] =& \\
&\quad\quad\quad\quad\quad\quad \text{Подстановка}& \\
&[z, {\bf let \: pure \:} y = ({\bf let \: pure \:} x = z {\: \bf in \:} M) {\: \bf in \:} N] =& \\
&\quad\quad\quad\quad\quad\quad \text{Правило $\beta \Box$}& \\
&[z, {\bf let \: pure \:} x = z {\: \bf in \:} N [y := M]]&
\end{array}$

\vspace{\baselineskip}

2)

$\begin{array}{lll}
&\text{fmap }(id_{\hat{A}}) = & \\
&\quad\quad\quad\quad\quad\quad \text{Определение тождественного морфизма}& \\
&\text{fmap }[x, x] = & \\
&\quad\quad\quad\quad\quad\quad \text{По определению fmap}& \\
&[z, {\bf let \: pure \:} x = z {\: \bf in \:} x]& \\
&\quad\quad\quad\quad\quad\quad \text{Правило $\Box id$}& \\
&[z, z] = id_{\boxdot \hat{A}}&
\end{array}$

\end{proof}

\begin{defin}

  Определим естественные преобразования:

\begin{itemize}
  \item $\eta:Id \Rightarrow \boxdot$, такое, что $\forall \hat{A} \in Ob_{\mathcal{C}(\lambda)}$, $\eta_{\hat{A}} = [x, {\bf pure \:} x] \in Hom_{\mathcal{C}(\lambda)}(\hat{A}, \boxdot \hat{A})$;
  \item $\ast_{A,B}:\boxdot \hat{A} \times \boxdot \hat{B} \to \boxdot (\hat{A} \times \hat{B})$, такое, что $\forall \hat{A}, \hat{B} \in Ob_{\mathcal{C}(\lambda)}, \ast_{\hat{A},\hat{B}} = [p, {\bf let \: pure \:} x,y = \pi_1 p, \pi_2 p {\: \bf in \:} \langle x, y \rangle] \in Hom_{\mathcal{C}(\lambda)}(\boxdot A \times \boxdot B, \boxdot (A \times B))$.
\end{itemize}
\end{defin}

Реализация $\ast$ в нашей термовой модели -- это частный случай правила $\text{let}_{\Box}$:

\begin{prooftree}
\AxiomC{$p : \Box A \times \Box B \vdash p : \Box A \times \Box B$}
\UnaryInfC{$p : \Box A \times \Box B \vdash \pi_1 p : \Box A$}
\AxiomC{$p : \Box A \times \Box B \vdash p : \Box A \times \Box B$}
\UnaryInfC{$p : \Box A \times \Box B \vdash \pi_2 p : \Box B$}
\AxiomC{$x : A \vdash x : A$}
\AxiomC{$y : B \vdash y : B$}
\BinaryInfC{$x : A, y : B \vdash \langle x, y \rangle : A \times B$}
\TrinaryInfC{$p : \Box A \times \Box B \vdash {\bf let \: pure \:} x,y = \pi_1 p, \pi_2 p {\: \bf in \:} \langle x, y \rangle : \Box (A \times B)$}
\end{prooftree}

\begin{lemma}
  $ $

  $\boxdot$ -- моноидальный эндофунктор.

\end{lemma}

\begin{proof}
$ $

Показывается аналогично \cite{ModalK}.
\end{proof}

\begin{lemma} Естественность и когерентность $\eta$:

\begin{itemize}
  \item $\text{fmap } f \circ \eta_A = \eta_B \circ f$;
  \item $\ast_{\hat{A},\hat{B}} \circ (\eta_{A} \times \eta_{B}) = \eta_{\hat{A} \times \hat{B}}$;
\end{itemize}
\end{lemma}

\begin{proof}
  $ $

  i) $\text{fmap } f \circ \eta_{\hat{A}} = \eta_{\hat{B}} \circ f$

\vspace{\baselineskip}

$\begin{array}{lll}
&\eta_{\hat{B}} \circ f = & \\
&\quad\quad\quad\quad\quad\quad \text{По определению}&\\
&[y, {\bf pure \:} y] \circ [x, M] = &\\
&\quad\quad\quad\quad\quad\quad \text{Композиция} & \\
&[x, {\bf pure \:} y [y := M]] = &\\
&\quad\quad\quad\quad\quad\quad \text{Подстановка}& \\
&[x, {\bf pure \:} M]& \\
& &\\
&\text{С другой стороны:}& \\
&\text{fmap } f \circ \eta_{\hat{A}} = & \\
&\quad\quad\quad\quad\quad\quad \text{По определению}& \\
&[z, {\bf let \: pure \:} x = z {\: \bf in \:} M] \circ [x, {\bf pure \: x}] = &\\
&\quad\quad\quad\quad\quad\quad \text{Композиция}& \\
&[x, {\bf let \: pure \:} x = z {\: \bf in \:} M [z := {\bf pure \:} x]] = & \\
&\quad\quad\quad\quad\quad\quad \text{Подстановка}& \\
&[x, {\bf let \: pure \:} x = {\bf pure \:} x {\: \bf in \:} M] = & \\
&\quad\quad\quad\quad\quad\quad \text{Правило $\beta \Box {\bf pure}$}& \\
&[x, {\bf pure \:} M [x := x]] = &\\
&\quad\quad\quad\quad\quad\quad \text{Тождественная постановка} & \\
&[x, {\bf pure \:} M]&
\end{array}$

\vspace{\baselineskip}

ii) $\ast_{\hat{A},\hat{B}} \circ (\eta_{\hat{A}} \times \eta_{\hat{B}}) = \eta_{\hat{A} \times \hat{B}}$

\vspace{\baselineskip}

$\begin{array}{lll}
& \ast_{\hat{A},\hat{B}} \circ (\eta_{\hat{A}} \times \eta_{\hat{B}}) = & \\
& [q, {\bf let \: pure \:} x, y = \pi_1 q, \pi_2 q {\: \bf in \:} \langle x, y \rangle] \circ [p, \langle {\bf pure \:} (\pi_1 p), {\bf pure \:} (\pi_2 p) \rangle] = & \\
& \quad\quad\quad\quad\quad\quad \text{Композиция}& \\
& [p, {\bf let \: pure \:} x, y = \pi_1 q, \pi_2 q {\: \bf in \:} \langle x, y \rangle [q := \langle {\bf pure \:} (\pi_1 p), {\bf pure \:} (\pi_2 p) \rangle]] = & \\
& \quad\quad\quad\quad\quad\quad \text{Подстановка}& \\
& [p, {\bf let \: pure \:} x, y = \pi_1 (\langle {\bf pure \:} (\pi_1 p), {\bf pure \:} (\pi_2 p) \rangle), \pi_2 (\langle {\bf pure \:} (\pi_1 p), {\bf pure \:} (\pi_2 p) \rangle) {\: \bf in \:} \langle x, y \rangle] = & \\
& \quad\quad\quad\quad\quad\quad \text{Правило $\beta$-редукции для пары}& \\
& [p, {\bf let \: pure \:} x, y = {\bf pure \:} (\pi_1 p), {\bf pure \:} (\pi_2 p) {\: \bf in \:} \langle x, y \rangle] = & \\
& \quad\quad\quad\quad\quad\quad \text{Правило $\beta \Box {\bf pure}$}& \\
& [p, {\bf pure \:} (\langle x,y \rangle [x := \pi_1 p, y := \pi_2 p])] = & \\
& \quad\quad\quad\quad\quad\quad \text{подстановка}& \\
& [p, {\bf pure \:} \langle \pi_1 p, \pi_2 p \rangle] = & \\
& \quad\quad\quad\quad\quad\quad \text{Правило $\eta$-редукции для пары}& \\
& [p, {\bf pure \:} p] =& \\
& \quad\quad\quad\quad\quad\quad \text{По определению}& \\
& \eta_{\hat{A} \times \hat{B}}&
\end{array}$

\end{proof}

\begin{defin}
  $ $

  $u_{\mathds{1}} = [\sqbullet, {\bf let \: pure \:} \underline{\quad} = \underline{\quad} {\: \bf in \:} \sqbullet] \in Hom_{\mathcal{C}(\lambda)}(\mathds{1}, \boxdot \mathds{1})$.
\end{defin}

\begin{lemma}
  $ $

  $u_{\mathds{1}} = \eta_{\mathds{1}}$

\end{lemma}

\begin{proof}

  Следует напрямую из правила $\beta {\bf nec}$.
\end{proof}

\begin{lemma}

  $\boxdot$ -- это аппликативный функтор.
\end{lemma}

\begin{proof}
  Непосредственно следует из предыдущих лемм.
\end{proof}

Аналогично \cite{Abramsky}, мы применяем трансляцию из $\lambda_{{\bf K}}$ к произвольной декартово замкнутой категории с аппликативным функтором $\boxdot$, тогда
мы имеем $[\![\Gamma \vdash M : A]\!] = [x, M [x_i := \pi_i x]]$, so $M =_{\beta \eta} N \Leftrightarrow [\![\Gamma \vdash M : A]\!] = [\![\Gamma \vdash N : A]\!]$.

\end{proof}


\newpage

\addcontentsline{toc}{section}{Список использованной литературы}

\begin{thebibliography}{}

\bibitem{Artemov} Artemov S. and Protopopescu T., \/ ``Intuitionistic Epistemic Logic'', \textit{The
Review of Symbolic Logic}, 2016, vol. 9, no 2. pp. 266-298.\parskip=1mm

\bibitem{Krupski} Krupski V. N. and Yatmanov A., \/ ``Sequent Calculus for Intuitionistic Epistemic Logic
IEL'', \textit{Logical Foundations of Computer Science: International Symposium, LFCS 2016, Deerfield
Beach, FL, USA, January 4-7, 2016. Proceedings}, 2016, pp. 187-201.\parskip=1mm

\bibitem{Haskell} Haskell Language. // URL: https://www.haskell.org. (Date: 1.08.2017) \parskip=1mm

\bibitem{Idris} Idris. A Language with Dependent Types.// URL:https://www.idris-lang.org. (Date:
1.08.2017) \parskip=1mm

\bibitem{Purs} Purescript. A strongly-typed functional programming language that compiles to JavaScript.
URL: http://www.purescript.org. (Date: 1.08.2017) \parskip=1mm

\bibitem{Elm} Elm. A delightful language for reliable webapps. // URL: http://elm-lang.org. (Date:
1.08.2017) \parskip=1mm

\bibitem{Base} Hackage, \/ ``The base package'' // URL: https://hackage.haskell.org/package/base-4.10.0.0
(Date: 1.08.2017) \parskip=1mm

\bibitem{Miran} Lipovaca M, \/ ``Learn you a Haskell for Great Good!''. //URL:
http://learnyouahaskell.com/chapters (Date: 1.08.2017) \parskip=1mm

\bibitem{McP} McBride C. and Paterson R., ``Applicative programming with effects", \textit{Journal of
Functional Programming}, 2008, vol. 18, no 01. pp 1-13. \parskip=1mm

\bibitem{McP2} McBride C. and Paterson R, ``Functional Pearl. Idioms: applicative programming with
effects'', \textit{Journal of Functional Programming}, 2005. vol. 18, no 01. pp 1-20. \parskip=1mm

\bibitem{Neder} R. Nederpelt and H. Geuvers, ``Type Theory and Formal Proof: An Introduction''.
\textit{Cambridge University Press}, New York, NY, USA, 2014. pp. 436. \parskip=1mm

\bibitem{Morten} Sorensen M. H. and Urzyczyn P, ``Lectures on the Curry-Howard isomorphism'',
\textit{Studies in Logic and the Foundations of Mathematics}, vol. 149, \textit{Elsevier Science}, 1998.
pp 261. \parskip=1mm

\bibitem{Pierce} Pierce B. C., ``Types and Programming Languages''. \textit{Cambridge, Mass: The MIT
Press}, 2002. pp. 605. \parskip=1mm

\bibitem{Girard} Girard J.-Y., Taylor P. and  Lafont Y, ``Proofs and Types'', \textit{Cambridge University
Press}, New York, NY, USA, 1989. pp. 175. \parskip=1mm

\bibitem{Baren} Barendregt. H. P., ``Lambda calculi with types" // Abramsky S., Gabbay Dov M., and S. E.
Maibaum, ``Handbook of logic in computer science (vol. 2), Osborne Handbooks Of Logic In Computer
Science'', Vol. 2. \textit{Oxford University Press, Inc.}, New York, NY, USA, 1993. pp 117-309.
\parskip=1mm

\bibitem{Hindley} Hindley J. Roger, ``Basic Simple Type Theory''. \textit{Cambridge University Press}, New
York, NY, USA, 1997. pp. 185. \parskip=1mm

\bibitem{Lax} Pfenning F. and Davies R., ``A judgmental reconstruction of modal logic'',
\textit{Mathematical Structures in Computer Science}, vol. 11, no 4, 2001, pp. 511-540. \parskip=1mm

\bibitem{Baren2} H.P. Barendregt. The Lambda Calculus --- Its Syntax and Semantics. Studies in Logic and
the Foundations of Mathematics, vol. 103. Amsterdam: North-Holland, 1985.

\bibitem{ModalK} Yoshihiko KAKUTANI, A Curry-Howard Correspondence for Intuitionistic Normal Modal Logic, Computer Software, Released February 29, 2008, Online ISSN , Print ISSN 0289-6540.

\bibitem{ModalK1} Kakutani Y. (2007) Call-by-Name and Call-by-Value in Normal Modal Logic. In: Shao Z. (eds) Programming Languages and Systems. APLAS 2007. Lecture Notes in Computer Science, vol 4807. Springer, Berlin, Heidelberg

\bibitem{Abe} T. Abe. Completeness of modal proofs in first-order predicate logic. Computer Software, JSSST Journal, 24:165 -- 177, 2007.

\bibitem{LambekScott} Lambek, J. and Scott P.J. (1986) Introduction to Higher Order Categorical Logic, Cambridge Studies in Advanced Mathematics 7, Cambridge: Cambridge University Press.

\bibitem{ElKelly} Samuel Eilenberg and Max Kelly, Closed categories. Proc. Conf. Categorical Algebra (La Jolla, Calif., 1965).

\bibitem{Abramsky} Samson Abramsky and Nikos Tzevelekos, Introduction to Categories and Categorical Logic

\bibitem{ModalLa} G. A. Kavvos. The Many Worlds of Modal $\Lambda$--calculi: I. Curry-Howard for Necessity, Possibility and Time

\bibitem{Cons} Ross Paterson. in Mathematics of Program Construction, Madrid, 2012, Lecture Notes in Computer Science, vol. 7342, pp. 300--323, Springer, 2012.
\end{thebibliography}


\end{document}
