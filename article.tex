\documentclass[a4paper]{article}
\usepackage{amsmath}
\usepackage{amsthm}
\usepackage{amsfonts}
\usepackage{amssymb}
\usepackage{bussproofs}
\usepackage{mathtools}
\usepackage{verbatim}
\usepackage{dsfont}
\usepackage[all, 2cell]{xy}
\usepackage[all]{xy}
\newtheorem{theorem}{Theorem}
\newtheorem{lemma}{Lemma}
\newtheorem{defin}{Definition}
\newtheorem{ex}{Example}
\newtheorem{col}{Corollary}
\newtheorem{con}{Convention}
\usepackage{listings} 		% for source code
\date{}
\title{Modal type theory based on the intuitionistic epistemic logic}

\begin{document}
\maketitle

\begin{abstract}
Modal intuitionistic epistemic logic IEL$^{-}$ was proposed by S.Artemov and T. Protopopescu as the formal
foundation for the intuitionistic theory of knowledge. We construct a modal simply typed lambda-calculus
which is Curry-Howard isomorphic to IEL$^{-}$ as formal theory of calculations with applicative functors in
functional programming languages like Haskell or Idris. We prove that this typed lambda-calculus has the
strong normalization and Church-Rosser properties.
\end{abstract}


  % This is a comment, not shown in final output.
  % The following shows typesetting  power of LaTeX:

  \section{Introduction}

  Modal intutionistic epistemic logic IEL was proposed by S. Artemov and T. Proropopescu \cite{Artemov}. IEL
  provides the epistimology and the theory of knowledge as based on BHK-semantics of intuitionistic logic.
  $IEL^{-}$  is a variant of IEL, that corresponds to intuitionistic belief.
  Informally, $\textbf{K}A$ denotes that $A$ is verified intuitionistically.

  Intuitionistic epistemic logic IEL$^{-}$ is defined with by following axioms and derivation rules:

  \begin{defin} Intuitionistic epistemic logic IEL:

    1) IPC axioms;

    2) $\textbf{K}(A \to B) \to (\textbf{K}A \to \textbf{K}B)$ (normality);

    3) $A \to \textbf{K}A$ (co-reflection);

    Rule: MP.

  \end{defin}

  We have the deduction theorem and necessitation rule which is derivable.

  V. Krupski and  A. Yatmanov provided the sequential calculus for IEL and proved that this calculus is
  PSPACE-complete \cite{Krupski}.

  It's not difficult to see that modal axioms in $IEL^{-}$ and types of the methods of Applicative class in
  Haskell-like languages (which is described below) are syntactically similar and we are going to show that
  this coincidence has a
  non-trivial computational meaning.

  Functional programming languages such as Haskell \cite{Haskell}, Idris \cite{Idris}, Purescript \cite{Purs}
  or Elm \cite{Elm} have special type classes\footnote{Type class in Haskell is a general interface for
  special group of datatypes.} for calculations with container types like \verb"Functor" and
  \verb"Applicative" \footnote{Reader may read more about container types in the Haskell standard library documentation\cite{Base} or in the next one textbook
  \cite{Miran}}:

  \begin{lstlisting}[language=Haskell]
    class Functor f where
      fmap :: (a -> b) -> f a -> f b

    class Functor f => Applicative f where
      pure :: a -> f a
      (<*>) :: f (a -> b) -> f a -> f b
  \end{lstlisting}

  By \emph{container} (or \emph{computational context}) type we mean some type-operator $f$, where $f$ is a
  ``function'' from $*$ to $*$: type operator takes a simple type (which has kind $*$) and returns another
  simple type type with kind \verb"*". For more detailed description of the type system with kinds used in
  Haskell see \cite{Morten}.


  The main goal of our research is a relationship between intuitionistic epistemic logic $IEL^{-}$ and
  functional programming with effects. We show that relationship by building the type system (which is called $\lambda_{\textbf{K}}$) which is Curry-Howard isomorphic to $IEL^{-}$. So we will consider $\textbf{K}$-modality as an arbitrary applicative functor.

  $\lambda K$ consists of the rules for simply typed lambda-calculus and special typing rules for
  lifting types into the applicative functor $\textbf{K}$. We assume that our type system will
  axiomatize the simplest case of computation with effects with one container. We provide proof-theoretical
  view on
  this kind of computations in functional programming and prove strong normalization and confluence.

  \section{Typed lambda-calculus based on IEL$^{-}$}

  At first we define the natural deduction for IEL$^{-}$ with $\textbf{K}$-modality and binary connectives
  $\to$ and $\land$ (we call that calculus NIEL$^{-}_{\land, \to}$):

  \begin{defin} Natural deduction NIEL$^{-}_{\land, \to}$ for IEL$^{-}$ with $\to$ and $\wedge$:

  \begin{center}
  \begin{prooftree}
  \AxiomC{}
  \RightLabel{\scriptsize{ax}}
  \UnaryInfC{$\Gamma , \alpha \vdash A$}
  \end{prooftree}
  \end{center}

  \begin{minipage}{0.5\textwidth}
    \begin{flushleft}
  	\begin{prooftree}
  	\AxiomC{$\Gamma, A \vdash B$}
  	\RightLabel{$\rightarrow_i$}
  	\UnaryInfC{$\Gamma \vdash A \to B$}
    	\end{prooftree}

  	\begin{prooftree}
  	\AxiomC{ $\Gamma \vdash A$ }
  	\AxiomC{ $\Gamma \vdash B$ }
  	\RightLabel{$\land_i$}
  	\BinaryInfC{$\Gamma \vdash A \land B$}
  	\end{prooftree}

  	\begin{prooftree}
    	\AxiomC{$\Gamma \vdash A$}
    	\RightLabel{$\textbf{K}_I$}
    	\UnaryInfC{$\Gamma \vdash \textbf{K}A$}
  \end{prooftree}
    \end{flushleft}
  \end{minipage}
  \begin{minipage}{0.5\textwidth}
    \begin{flushright}
    	\begin{prooftree}
  	\AxiomC{$\Gamma \vdash A \to B$}
  	\AxiomC{$\Gamma \vdash A$}
  	\RightLabel{$\rightarrow_i$}
  	\BinaryInfC{$\Gamma \vdash B$}
    	\end{prooftree}

  	\begin{prooftree}
  	\AxiomC{ $\Gamma \vdash A_1 \land A_2$ }
  	\RightLabel{$\land_e, i \in \{ 1, 2 \}$}
  	\UnaryInfC{$\Gamma \vdash A_i$}
  	\end{prooftree}

  	\begin{prooftree}
    \AxiomC{$\Gamma \vdash {\bf K} \vec{A}$}
    \AxiomC{$\vec{A} \vdash B$}
    \BinaryInfC{$\Gamma \vdash {\bf K} B$}
  	\end{prooftree}
    \end{flushright}
  \end{minipage}
  \end{defin}

Where $\Gamma \vdash {\bf K} \vec{A}$ is a syntax sugar for $\Gamma \vdash {\bf K} A_1 ,\dots, \Gamma \vdash {\bf K} A_n$.

  \vspace{\baselineskip}

  \begin{lemma}
    $\Gamma \vdash_{\text{NIEL}^{-}_{\land, \to}} A \Rightarrow$ IEL$^{-} \vdash \bigwedge \Gamma \rightarrow A$.
  \end{lemma}

  \begin{proof}
Induction on the derivation.

  \vspace{\baselineskip}

Let us consider cases with modality.

\vspace{\baselineskip}

1) If $\Gamma \vdash_{NIEL^{-}_{\land, \to}} A$, then $IEL^{-} \vdash \bigwedge \Gamma \rightarrow \textbf{K}A$.

$\begin{array}{lll}
(1) & \bigwedge \Gamma \rightarrow A & \text{assumption}\\
(2) & A \rightarrow \textbf{K}A &\text{co-reflection}\\
(3) & (\bigwedge \Gamma \rightarrow A) \rightarrow ((A \rightarrow \textbf{K}A) \rightarrow (\bigwedge \Gamma \rightarrow \textbf{K}A))&\text{IPC theorem}\\
(4) & (A \rightarrow \textbf{K}A) \rightarrow (\bigwedge \Gamma \rightarrow \textbf{K}A) &\text{from (1), (3) and MP}\\
(5) & \bigwedge \Gamma \rightarrow \textbf{K}A &\text{from (2), (4) and MP}\\
\end{array}$

\vspace{\baselineskip}

2) If $\Gamma \vdash_{NIEL^{-}_{\land, \to}} {\bf K} \vec{A}$ and $\vec{A} \vdash B$, then $IEL^{-} \vdash \bigwedge \Gamma \rightarrow \textbf{K}B$.

$\begin{array}{lll}
(1) &\bigwedge \Gamma \rightarrow \bigwedge \limits_{i = 1}^{n} {\bf K} A_i & \text{assumption} \\
(2) &\bigwedge \limits_{i = 1}^{n} {\bf K} A_i \rightarrow {\bf K} \bigwedge \limits_{i = 1}^{n} A_i& \text{IEL theorem} \\
(3) &\bigwedge \Gamma \rightarrow {\bf K} \bigwedge \limits_{i = 1}^{n} A_i & \text{from (1), (2) and transitivity} \\
(4) &\bigwedge \limits_{i = 1}^{n} A_i \rightarrow B& \text{assumption} \\
(5) &(\bigwedge \limits_{i = 1}^{n} A_i \rightarrow B) \rightarrow {\bf K} (\bigwedge \limits_{i = 1}^{n} A_i \rightarrow B)& \text{co-reflection}\\
(6) &{\bf K} (\bigwedge \limits_{i = 1}^{n} A_i \rightarrow B)& \text{from (2), (3) and MP} \\
(7) &{\bf K} \bigwedge \limits_{i = 1}^{n} A_i \rightarrow {\bf K} B & \text{from (6) and normality} \\
(8) &\bigwedge \Gamma \rightarrow {\bf K} B & \text{from (3), (7) and transitivity}
\end{array}$

  \end{proof}

\begin{lemma}
  $ $
  If $\text{IEL}^{-} \vdash A$, then $\text{NIEL}^{-} \vdash A$.
\end{lemma}

\begin{proof}
  Straightforward derivation of modal axioms in $\text{NIEL}^{-}$. We consider this derivation below using terms.
\end{proof}

  \vspace{\baselineskip}

  At the next step we build the typed lambda-calculus based on NIEL$^{-}_{\land, \to}$ by proof-assingment in
  rules.

  At first, we define lambda-terms and types for this lambda-calculus.

  \vspace{\baselineskip}

  \begin{defin} The set of terms:

  Let $\mathbb{V}$ be the set of variables. The set $\Lambda_{\textbf{K}}$ of terms is defined by the grammar:

$\begin{array}{lll}
& \Lambda_{\textbf{K}} ::= \mathbb{V} \: | \:  (\lambda \Lambda.\Lambda_{\textbf{K}}) \: | \: (\Lambda_{\textbf{K}}\Lambda_{\textbf{K}}) \: | \: (\Lambda_{\textbf{K}} , \Lambda_{\textbf{K}}) \: | \: (\pi_1 \Lambda_{\textbf{K}}) \: | \: (\pi_2 \Lambda_{\textbf{K}}) \: | & \\
& \quad\quad\quad\quad\quad\quad\quad\quad\quad\quad\quad\quad\quad\quad\quad\quad ({\bf pure \: } \: \Lambda_{\textbf{K}}) \: | \: ({\bf let \:\: pure} \:\: \Lambda_{\textbf{K}} = \Lambda_{\textbf{K}} \:\: {\bf in} \:\: \Lambda_{\textbf{K}})
\end{array}$

  \end{defin}

  \begin{defin} The set of types:

  Let $\mathbb{T}$ be the set of atomic types. The set $\mathbb{T}_{\textbf{K}}$ of types with applicative
  functor
  $\textbf{K}$ is generated by the grammar:
  \begin{equation}
    \mathbb{T}_{\textbf{K}} ::= \mathbb{T} \: | \: (\mathbb{T}_{\textbf{K}} \to \mathbb{T}_{\textbf{K}}) \: |
    \:
    (\mathbb{T}_{\textbf{K}} \times \mathbb{T}_{\textbf{K}}) \: | \: (\textbf{K}\mathbb{T}_{\textbf{K}})
  \end{equation}
  \end{defin}

  Context, domain of context and range of context are defined standardly \cite{Neder}\cite{Morten}.

  Our type system is based on the Curry-style typing rules:

  \begin{defin} Modal typed lambda calculus $\lambda \textbf{K}$ based on NIEL$^{-}_{\land, \to}$:

    \begin{center}
    \begin{prooftree}
    \AxiomC{$ $}
    \RightLabel{\scriptsize{ax}}
    \UnaryInfC{$\Gamma , x : A \vdash x : A$}
    \end{prooftree}
    \end{center}

    \begin{minipage}{0.45\textwidth}
      \begin{prooftree}
      \AxiomC{$\Gamma, x : A \vdash M : B$}
      \RightLabel{$\rightarrow_i$}
      \UnaryInfC{$\Gamma \vdash \lambda x. M : A \to B$}
      \end{prooftree}

      \begin{prooftree}
      \AxiomC{ $\Gamma \vdash M : A$ }
      \AxiomC{ $\Gamma \vdash N : B$ }
      \RightLabel{$\times_i$}
      \BinaryInfC{$\Gamma \vdash \langle x, y \rangle : A \times B$}
      \end{prooftree}

      \begin{prooftree}
        \AxiomC{$\Gamma \vdash x : A$}
        \RightLabel{$\textbf{K}_I$}
        \UnaryInfC{$\Gamma \vdash {\bf pure \: } \: x : {\bf K}A $}
      \end{prooftree}
  \end{minipage}%
  \hfill
  \begin{minipage}{0.45\textwidth}
  \begin{tabular}{p{\textwidth}}
    \begin{prooftree}
    \AxiomC{$\Gamma \vdash f : A \to B$}
    \AxiomC{$\Gamma \vdash x : A$}
    \RightLabel{$\rightarrow_e$}
    \BinaryInfC{$\Gamma \vdash fx : B$}
    \end{prooftree}

    \begin{prooftree}
    \AxiomC{ $\Gamma \vdash M : A_1 \times A_2$ }
    \RightLabel{$\times_e$, $i \in \{ 1, 2 \}$}
    \UnaryInfC{$\Gamma \vdash \pi_i M : A_i$}
    \end{prooftree}

    \begin{prooftree}
      \AxiomC{$\Gamma \vdash \vec{M} : {\bf K} \vec{A}$}
      \AxiomC{$\vec{x} : \vec{A} \vdash M : B$}
      \RightLabel{$\text{let}_{\bf K}$}
      \BinaryInfC{$\Gamma \vdash {\bf let \: pure \:} \vec{x} = \vec{M} {\: \bf in \: } M : {\bf K} B$}
    \end{prooftree}
  \end{tabular}
  \end{minipage}%

  \end{defin}

  $\textbf{K}_I$-typing rule is the same as $\bigcirc$-introduction in lax logic (also known as monadic
  metalanguage \cite{Lax}) and in typed lambda-calculus which is derived by proof-assignment for lax-logic
  proofs.
  $\textbf{K}_I$ allows to inject an object of type $\alpha$ into the functor. $\textbf{K}_I$  reflects the
  Haskell method {\bf pure} for Applicative class. It plays the same role as the {\bf return} method in Monad class.

  $\text{let}_{\bf K}$ is similar to $\Box_I$-rule in typed lambda calculus for intuitionistic normal modal logic {\bf IK}, which is described in \cite{ModalK}.

  \vspace{\baselineskip}

Here are some examples of derivation trees.

  \begin{prooftree}
  \AxiomC{$x : A \vdash x : A$}
  \RightLabel{$\textbf{K}_I$}
  \UnaryInfC{$x : A \vdash {\bf pure \: } x : \textbf{K} A$}
  \RightLabel{$\rightarrow_i$}
  \UnaryInfC{$\vdash (\lambda x. {\bf pure \: } x) : A \to \textbf{K}A$}
  \end{prooftree}


  \begin{prooftree}
    \AxiomC{$f : {\bf K} (A \to B) \vdash f : {\bf K} (A \to B)$}
    \AxiomC{$x : {\bf K} A \vdash x : {\bf K} A $}
    \AxiomC{$g : A \to B$}
    \AxiomC{$y : A$}
    \BinaryInfC{$g : A \to B, y : A \vdash g y : B$}
    \TrinaryInfC{$f : {\bf K} (A \to B), x : {\bf K} A \vdash {\bf let \:\: pure} \:\: \langle g, y \rangle = \langle f, x \rangle \:\: {\bf in } \:\: g y : {\bf K} B$}
    \UnaryInfC{$f : {\bf K} (A \to B) \vdash \lambda x. {\bf let \:\: pure} \:\: \langle g, y \rangle = \langle f, x \rangle \:\: {\bf in } \:\: g y : {\bf K} A \to {\bf K} B$}
    \UnaryInfC{$\vdash \lambda f. \lambda x. {\bf let \:\: pure} \:\: \langle g, y \rangle = \langle f, x \rangle \:\: {\bf in } \:\: g y : {\bf K}(A \to B) \to {\bf K} A \to {\bf K} B$}
  \end{prooftree}

\begin{prooftree}
  \AxiomC{$f : A \to B \vdash f : A \to B$}
  \UnaryInfC{$f : A \to B \vdash {\bf pure \:\:} f : {\bf K }(A \to B)$}
  \AxiomC{$x : {\bf K} A \vdash x : {\bf K} A $}
  \AxiomC{$g : A \to B$}
  \AxiomC{$y : A$}
  \BinaryInfC{$g : A \to B, y : A \vdash g y : B$}
  \TrinaryInfC{$f : A \to B, x : {\bf K} A \vdash {\bf let \:\: pure} \:\: \langle g, y \rangle = \langle {\bf pure \:\:} f, x \rangle \:\: {\bf in } \:\: g y : {\bf K} B$}
  \UnaryInfC{$f : A \to B \vdash \lambda x. {\bf let \:\: pure} \:\: \langle g, y \rangle = \langle {\bf pure \:\:} f, x \rangle \:\: {\bf in } \:\: g y : {\bf K} A \to {\bf K} B$}
  \UnaryInfC{$\lambda f. \lambda x. {\bf let \:\: pure} \:\: \langle g, y \rangle = \langle {\bf pure \:\:} f, x \rangle \:\: {\bf in } \:\: g y : (A \to B) \to {\bf K} A \to {\bf K} B$}
\end{prooftree}




  \vspace{\baselineskip}

  Now we define free variables and substitutions. $\beta$-reduction, multi-step $\beta$-reduction and $\beta$-equality are defined standardly:

  \begin{defin} Set $FV(M)$ of free variables for arbitrary term $M$:

  1) $FV(x) = \{ x \}$;

  2) $FV(\lambda x. M) = FV(M) \setminus \{ x\}$;

  3) $FV(M N) = FV(M) \cup FV(N);$

  4) $FV((M,N)) = FV(M) \cup FV(N)$;

  5) $FV(\pi_i p) \subseteq FV(p)$, $i \in \{ 1, 2\}$;

  6) $FV(\text{pure } M) = FV(M)$;

  7) $FV({\bf let \:\: pure} \:\: \vec{N} = \vec{M} \:\: {\bf in} \:\: M) = \bigcup \limits_{i = 1}^n FV(M), \text{where $n = |\vec{M}|$}$.
  \end{defin}

  \begin{defin} Substitution:

  1) $x [x := N] = N$, $x [y := N] = x$;

  2) $(M N) [x := N] = M[x := N] N [x := N]$;

  3) $(\lambda x. M) [x := N] = \lambda x. M [x := N]$;

  4) $(M, N)[x := P] = (M[x := P], N [x := P])$;

  5) $(\pi_i M) [x := P] = \pi_i (M[x := P])$, $i \in \{ 1, 2\}$;

  6) $({\bf pure \: } M) [x := P] = {\bf pure \: } (M [x := P])$;

  7) $({\bf let \:\: pure} \:\: \vec{N} = \vec{M} \:\: {\bf in} \:\: M) [x := P] = {\bf let \:\: pure} \:\: \vec{N} = (\vec{M} [x := P]) \:\: {\bf in} \:\: M$.
  \end{defin}

  \begin{defin} $\beta$-reduction and $\eta$-reduction rules for \emph{$\lambda \textbf{K}$}.

  1) $(\lambda x. M) N \rightarrow_{\beta} M [x := N]$;

  2) $\pi_1 \langle M, N \rangle \rightarrow_{\beta} M$;

  3) $\pi_2 \langle M, N \rangle \rightarrow_{\beta} N$;

  4) $\begin{array}{llll}
  & {\bf let \: pure \:} \langle \vec{x}, y, \vec{z} \rangle = \langle \vec{M}, {\bf let \: pure \: } \vec{w} = \vec{N} {\bf \: in \: } Q, \vec{P} \rangle {\: in \:} R \rightarrow_{\beta} & \\
  & {\bf let \: pure \:} \langle \vec{x}, \vec{w}, \vec{z} \rangle = \langle \vec{M}, \vec{N}, \vec{P} \rangle \: { \bf in } \: R [y := Q]
  \end{array}$

  5) ${\bf pure \:} ((\lambda x. M) N) \rightarrow_{\beta} {\bf pure \: } (M [x := N])$;

  6) ${\bf pure \:} (\pi_i \langle M_1, M_2 \rangle) \rightarrow_{\beta} {\bf pure \:} M_i$, where $i \in \{ 1, 2 \}$.

  7) $\begin{array}{llll}
  & {\bf pure \:} ({\bf let \: pure \:} \langle \vec{x}, y, \vec{z} \rangle = \langle \vec{M}, {\bf let \: pure \: } \vec{w} = \vec{N} {\bf \: in \: } Q, \vec{P} \rangle {\: in \:} R) \rightarrow_{\beta} & \\
  & {\bf pure \:} ({\bf let \: pure \:} \langle \vec{x}, \vec{w}, \vec{z} \rangle = \langle \vec{M}, \vec{N}, \vec{P} \rangle \: { \bf in } \: R [y := Q])
  \end{array}$

  8) $\lambda x. f x \rightarrow_{\eta} f$;

  9) $\langle \pi_1 P, \pi_2 P \rangle \rightarrow_{\eta} P$;

  10) ${\bf let \: pure \:} \underline{\quad} = \underline{\quad} {\: \bf in \:} N \rightarrow_{\eta} {\bf pure \:} N$;

  11) ${\bf let \: pure \:} x = M {\: \bf in \: } x \rightarrow_{\eta} M$;

  12) ${\bf pure \:} (\lambda x. f x) \rightarrow_{\eta} {\bf pure \:} f$;

  13) ${\bf pure \:} (\langle \pi_1 P, \pi_2 P \rangle) \rightarrow_{\eta} {\bf pure \:} P$;

  14) ${\bf pure \:} ({\bf let \: pure \:} x = M {\: \bf in \: } x) \rightarrow_{\eta} {\bf pure \:} M$;

  15) ${\bf pure \:} ({\bf let \: pure \:} \underline{\quad} = \underline{\quad} {\: \bf in \:} N) \rightarrow { \bf pure \:} ({\bf pure \:} N)$.
  \end{defin}



  \section{Basic lemmas}

  Now we will prove standard lemmas for contexts in type systems\footnote{We will not prove cases with
  $\to$-constructor, they are proved standardly in the same lemmas for simply typed lambda calculus, for
  example see \cite{Neder}\cite{Morten}\cite{Girard}. We will consider only modal cases}:

  \begin{defin} The domain of a context $\Gamma$:

  Let $\Gamma = \{ x_1 : A_1, ..., x_n : A_n \}$. Then the domain of $\Gamma$, or $dom(\Gamma)$, is a set $\{  x_1, ..., x_n\}$.

  \end{defin}

  \begin{lemma}

  If $\Gamma \vdash M : A$, then $FV(M) \subseteq dom(\Gamma)$

  \end{lemma}

  \begin{proof}

  Induction on the derivation of $\Gamma \vdash M : A$.

  \end{proof}

  \vspace{\baselineskip}

  \begin{lemma} Generation for $\lambda \textbf{K}$.

  1) $\Gamma \vdash {\bf pure \: } M : \textbf{K}\alpha$ implies that $\Gamma \vdash M : \alpha$;

  2) $\Gamma \vdash {\bf let \: pure} \: \vec{N} = \vec{M} \: {\bf in} \: M : {\bf K} B$ implies that $\Gamma \vdash \vec{M} : {\bf K} \vec{A}$ and
$\vec{N} : \vec{A} \vdash M : B$.
  \end{lemma}

  \begin{proof}

  $ $

Induction on the derivation of $\Gamma \vdash {\bf pure \: } M : \textbf{K}\alpha$ and $\Gamma \vdash {\bf let \: pure} \: \vec{N} = \vec{M} \: {\bf in} \: M : {\bf K} B$ respectively.
  \end{proof}

  \vspace{\baselineskip}

  The next one lemma allows that weakening structural rule is admissable.

  \begin{lemma} Weakening for $\lambda \textbf{K}$.

  Let $\Gamma \vdash M : A$ and $\Gamma \subseteq \Delta$, then $\Delta \vdash M : A$.

  \end{lemma}

  \begin{proof}

  $ $

  Induction on derivation of $\Gamma \vdash M : A$. Let us assume $\Gamma \subseteq \Delta$.

  1) Let $\Gamma \vdash x : A$, such that $\Gamma = \Delta, x : A$ and $\Theta \subseteq \Gamma$.
  Let $\Sigma = \Theta \setminus \Gamma$, or, which is the same, $\Sigma = \Theta \setminus \Delta, x :
  A$, then $\Sigma, \Delta, x : A \vdash x : A$, or, $\Theta \vdash x : A$.

  \vspace{\baselineskip}

  2) Let $\Gamma \vdash {\bf pure \: } M : \textbf{K}A$ and $\Gamma \subseteq \Theta$.

  By generation $\Gamma \vdash M : A$

  By hypothesis, $\Theta \vdash M : A$, so $\Theta \vdash {\bf pure \: } M : \textbf{K}A$ by applying
  $\textbf{K}_I$-rule.

    \vspace{\baselineskip}

  3) Let $\Gamma \vdash {\bf let \: pure \:} \vec{x} = \vec{M} {\: \bf in \:} N : {\bf K} B$ and $\Gamma \subseteq \Theta$.

  By generation $\Gamma \vdash \vec{M} : {\bf K} \vec{A}$ and $\vec{x} : \vec{A} \vdash N : B$.

  By assumption $\Theta \vdash \vec{M} : {\bf K} \vec{A}$.

  Hence $\Theta \vdash {\bf let \: pure \:} \vec{x} = \vec{M} {\: \bf in \:} N : {\bf K} B$.

  \end{proof}

  \begin{lemma} Considering for $\lambda \textbf{K}$.

  If $\Gamma \vdash M : \alpha$, then $\Gamma \uparrow FV(M) \vdash M : \alpha$, where $\Gamma \uparrow
  FV(M)$ is a subcontext of $\Gamma$, such that $dom(\Gamma \uparrow FV(M)) = dom(\Gamma) \cap FV(M)$.

  \end{lemma}

  \begin{proof}

  $ $

  1) Let $\Gamma \vdash x : A$, where $\Gamma = \Delta, x : A$, $x \in \mathbb{V}$.

  $FV(x) = \{ x\}$, then $dom(\Gamma) \cap \{ x\} = \{ x\}. $ So $(\Delta, x : A) \uparrow FV(x) = \{ x
  : A\}$, then $x : A \vdash x : A$ by axiom.

  \vspace{\baselineskip}

  2) Let $\Gamma \vdash {\bf pure \: } M : {\bf K} A$.

  By generation, $\Gamma \vdash M : A$ and $\Gamma \uparrow FV(M) \vdash M : A$ by hypothesis.

  So $\Gamma \uparrow FV(M) \vdash {\bf pure \: } M : {\bf K} A$ by ${\bf K}_I$.

  \vspace{\baselineskip}

  3) Let $\Gamma \vdash {\bf let \: pure \:} \vec{x} = \vec{M} {\: \bf in \:} N : {\bf K} B$.

By generation, $\Gamma \vdash \vec{M} : {\bf K} \vec{A}$ and $\vec{x} : \vec{A} \vdash N : B$.

By assumption, $\Gamma \uparrow FV(\vec{M}) \vdash \vec{M} : {\bf } \vec{A}$.

By $\text{let}_{\bf K}$, $\Gamma \uparrow FV(\vec{M}) \vdash {\bf let \: pure \:} \vec{x} = \vec{M} {\: \bf in \:} N : {\bf K} B$

  \end{proof}

  \vspace{\baselineskip}

  \begin{lemma}

  If $\Gamma, x : A \vdash M : B$  and $\Gamma \vdash N : A$, then $\Gamma \vdash (M [x := N]) : B$

  \end{lemma}

  \begin{proof}

  $ $

  1) Let $\Gamma, x : A \vdash {\bf pure \: } M : \textbf{K}B$ and $\Gamma \vdash N : A$.

  By generation, $\Gamma, x : A \vdash M : B$.

  By assumption, $\Gamma \vdash (M [x := N]) : B$

  Then $\Gamma \vdash {\bf pure \: } (M [x := N]) : \textbf{K}B$ by $\textbf{K}_I$.

  But ${\bf pure \: } (M [x := N]) = ({\bf pure \: } M [x := N]) $ by substitution definition, so
  $\Gamma \vdash ({\bf pure \: } M [x := N]) : \textbf{K}B$

  \vspace{\baselineskip}

  2) Let $\Gamma, y : A \vdash {\bf let \: pure \:} \vec{x} = \vec{M} {\: \bf in \:} N : {\bf K} B$ and $\Gamma \vdash N : A$.

  By generation, $\Gamma, y : A \vdash \vec{M} : {\bf K} \vec{A}$ and $\vec{x} : \vec{A} \vdash N : B$.

  By hypothesis, $\Gamma \vdash \vec{M}[x := N] : {\bf K} \vec{A}$.

  Hence $\Gamma \vdash {\bf let \: pure \:} \vec{x} = \vec{M}[x := N] {\: \bf in \:} N : {\bf K} B$.

  \end{proof}

  \vspace{\baselineskip}

  \begin{theorem} Subject reduction

  $ $

  Let $\Gamma \vdash M : \alpha$ and $M \twoheadrightarrow_{\beta} N$, then $\Gamma \vdash N : \alpha$

  \end{theorem}

We consider only modal $\beta$-reduction rules. The general statement for $\twoheadrightarrow_{\beta}$ follows from transitivity of multi-step $\beta$-reduction.

  \begin{proof}

  $ $

1) Let $\Gamma \vdash {\bf let \: pure \:} \langle \vec{x}, y, \vec{z} \rangle = \langle \vec{M}, {\bf let \: pure \: } \vec{w} = \vec{N} { \: \bf in \: } Q, \vec{P} \rangle { \: \bf in \: } R : {\bf K }B$

By generation we have $\Gamma \vdash \vec{M} : {\bf K}\vec{A_1}$, $\Gamma \vdash {\bf let \: pure \: } \vec{w} = \vec{N} {\: \bf in \:} Q : {\bf K}\vec{A_2}$, $\Gamma \vdash \vec{P} : {\bf K}\vec{A_3}$
and $\vec{x} : \vec{A_1}, y : A_2, \vec{z} : \vec{A_3} \vdash R : B$.

If $\Gamma \vdash {\bf let \: pure \: } \vec{w} = \vec{N} {\bf \: in \: } Q : {\bf K}\vec{A_2}$, then

$\Gamma \vdash \vec{N} : {\bf K}\vec{A_4}$ and $\vec{w} : \vec{A_4} \vdash Q : A_2$.

Then $\vec{x} : \vec{A_1}, \vec{w} : \vec{A_4}, \vec{z} : \vec{A_3} \vdash R [y := Q] : B$ by substitution lemma and weakening.

Hence $\Gamma \vdash {\bf let \: pure \:} \langle \vec{x}, \vec{w}, \vec{z} \rangle = \langle \vec{M}, \vec{N}, \vec{P} \rangle \: { \bf in } \: R [y := Q] : {\bf K} B$
by $let_{{\bf K}}$.

  \vspace{\baselineskip}

2) Let $\Gamma \vdash {\bf pure } ((\lambda x. M) N) : {\bf K} B$.

By generation $\Gamma \vdash (\lambda x. M) N : B$, but $\Gamma \vdash M [x := N] : B$, then, by ${\bf K}_I$, $\Gamma \vdash {\bf pure \:} (M [x := N]) : {\bf K} B$.

  \vspace{\baselineskip}

3) Let $\Gamma \vdash {\bf pure } (\pi_i \langle M_1, M_2 \rangle) : {\bf K} A_i$, where $i \in \{ 1,2 \}$.

By generation $\Gamma \vdash \pi_i \langle M_1, M_2 \rangle : A_i$ and $\Gamma \vdash M_i : A_i$.

Hence $\Gamma \vdash {\bf pure \: } M_i : { \bf K } A_i$ by ${\bf K}_I$.

\vspace{\baselineskip}

4) Let $\Gamma \vdash {\bf let \: pure \:} (\langle \vec{x}, y, \vec{z} \rangle = \langle \vec{M}, {\bf let \: pure \: } \vec{w} = \vec{N} {\bf \: in \: } Q, \vec{P} \rangle { \: \bf in \: } R) : {\bf K }^2 B$.

By generation $\Gamma \vdash {\bf let \: pure \:} \langle \vec{x}, y, \vec{z} \rangle = \langle \vec{M}, {\bf let \: pure \: } \vec{w} = \vec{N} {\bf \: in \: } Q, \vec{P} \rangle { \: \bf in \: } R : {\bf K }B$,

hence $\Gamma \vdash {\bf let \: pure \:} \langle \vec{x}, \vec{w}, \vec{z} \rangle = \langle \vec{M}, \vec{N}, \vec{P} \rangle \: { \bf in } \: R [y := Q] : {\bf K} B$ by the first case.

So $\Gamma \vdash {\bf pure \:} ({\bf let \: pure \:} \langle \vec{x}, \vec{w}, \vec{z} \rangle = \langle \vec{M}, \vec{N}, \vec{P} \rangle \: { \bf in } \: R [y := Q]) : {\bf K}^2 B$
by ${\bf K}_I$.



  \end{proof}


  \vspace{\baselineskip}


\section{Strong normalization}

  We modify and apply Tait's technique of logical relation for modalities. Strong normalization proof with Tait's method for simply typed lambda calculus is described here \cite{Pierce}.


  \begin{theorem}

    Let $M \in \Lambda_{\textbf{K}}$, then any sequence of reduction $M \rightarrow_{\beta} M_1 \dots$ terminates.

  \end{theorem}

  \begin{proof}

  We build the smallest of subset of strongly normalazing terms of modal types and show that an arbitrary
  term belongs to this subset.

  \begin{defin} The set of strongly computable terms of type $\phi \in \mathbb{T}_{\textbf{K}}$, $SC_{\phi}$:

  \begin{itemize}
  \item Let $\phi = \textbf{K}\alpha$ and $\alpha \in \mathbb{T}$ , then:
  \begin{equation}
  SC_{\textbf{K}\alpha} = \{ M : \textbf{K}\alpha \: | \: \text{$M$ is strongly normalizing} \}
  \end{equation}

  \item Let $\phi = \textbf{K}(\tau \to \psi)$ and $\tau, \psi \in \mathbb{T}_{\textbf{K}}$, then:
  \begin{equation}
  SC_{\textbf{K}(\tau \to \psi)} = \{ M : \textbf{K}(\tau \to \psi) \: | \: \forall N \in SC_{\textbf{K}\tau}, M \star N \in SC_{\textbf{K}\psi}\}
  \end{equation}

  \item Let $\phi = \textbf{K}(\tau_1 \times \tau_2)$ and $\tau_1, \tau_2 \in \mathbb{T}_{\textbf{K}}$, then:
  \begin{equation}
  SC_{\textbf{K}(\tau_1 \times \tau_2)} = \{ P : \textbf{K}(\tau_1 \times \tau_2) \: |
  \: {\bf pure \:} (\lambda x. \pi_i x) \star P \in SC_{\textbf{K}\tau_i}, i \in \{1,2\} \}
  \end{equation}
  \end{itemize}
  \end{defin}

  \begin{lemma}

  $ $

  If $M \in SC_{\alpha}$, then $M$ is strongly normalizing.

  \end{lemma}

  \begin{proof}

$ $

  1) If $M \in SC_{{\bf K} \alpha}$ and $\alpha \in \mathbb{T}$, then $M$ is strongly normalizing by the
  definition of $SC_{{\bf K} \alpha}$.

  2) Let $M \in SC_{\textbf{K}(\tau \to \psi)}$, so by every $N \in SC_{\textbf{K}\tau}$,
  $M \star N \in SC_{\textbf{K}\psi}$, which is strongly normalizing by hypothesis. So $M$ is
  strongly normalizing.

  3) Let $M \in SC_{\textbf{K}(\tau_1 \times \tau_2)}$, so ${\bf pure \:} (\lambda x. \pi_i x) \star M \in
  SC_{\textbf{K}\tau_i}$, $i \in \{ 1,2 \}$, which are strongly normalizing. So $M$ is
  strongly normalizing.

  \end{proof}


  \begin{lemma}

  $ $

  Let $M \rightarrow_{\beta} M'$ and $M \in SC_{\alpha}$, then $M' \in SC_{\alpha}$.

  \end{lemma}


  \begin{proof}

$ $

  1) Let $M \rightarrow_{\beta} M'$ and $M \in SC_{\textbf{K}\alpha}$, where $\alpha \in \mathbb{T}$.

  $M$ has the longest reduction path (which we denote as $p(M)$). So $p(M') < p(M)$, then $M' \in SC_{\textbf{K}\alpha}$.

  2) Let $M \in SC_{\textbf{K}(\alpha \to \beta)}$ and $M \rightarrow_{\beta} M'$. Let $N \in
  SC_{\textbf{K}\alpha}$. So $M \star N \in SC_{\textbf{K}\beta}$.

  If $M \rightarrow_{\beta} M'$, then $M \star N \rightarrow_{\beta} M' \star N$ by reduction
  rule, so $M' \star N \in SC_{\textbf{K}\beta}$ and $M' \in SC_{\textbf{K}(\alpha \to \beta)}$ by
  hypothesis.

  3) Let $M \in SC_{\textbf{K}(\tau_1 \times \tau_2)}$ and $M \rightarrow_{\beta} M'$.

  So ${\bf pure \: } (\lambda x. \pi_i x) \star M \rightarrow_{\beta} {\bf pure \: } (\lambda x. \pi_i x)
  \star M'$, $i \in \{ 1, 2 \}$ by reduction rule. So ${\bf pure \: } (\lambda x. \pi_i x) \star M' \in
  SC_{\textbf{K} \tau_i }$ and $M' \in SC_{\textbf{K}(\tau_1 \times \tau_2)}$.

  \end{proof}

  \begin{defin} Neutral term:

  We define a term $M$ to be neutral if it has of the next forms:

  1) $M = x$, where $x \in \mathbb{V}$;

  2) $M = (P Q)$;

  3) $M = \pi_i M$, $i \in \{ 1,2 \}$;

  4) $M = P \star Q$;

  5) If $M$ is a neutral, then ${\bf pure \: } M$ is a neutral.

  \end{defin}

  \begin{lemma}

  Let $M \rightarrow_{\beta} M'$ and $M' \in SC_{\alpha}$ for every one-step reduction. So if $M'$ is a
  neutral, then $M \in SC_{\alpha}$.

  \end{lemma}

  \begin{proof}

  $ $

  Simple induction on the structure of $M'$.

  \end{proof}

  \begin{lemma}

  $ $

  Let $x_1 : \phi_1, \dots, x_n : \phi_n \vdash M : \phi$ and for all $i \in \{ 1, \dots, n\}$, $N_i
  \in SC_{\phi_i}$, then $(M [x_1 := N_1, \dots, x_n := N_n]) \in SC_{\phi}$.

  \end{lemma}

  \begin{proof}

$ $

  1) If $\phi$ is an atomic and $M$ is a variable, then this condition holds straightforwardly.

  2) Let $\Gamma = \{ x_1 : \phi_1, \dots, x_n : \phi_n \}$, $\Gamma \vdash {\bf pure \: } M :
  \textbf{K}\alpha$ and for all $i \in \{1, \dots, n\}, N_i \in SC_{\phi_i}$.

  Then by $\Gamma \vdash M : \alpha$ by generation
  and $(M [x_1 := N_1, \dots, x_n := N_n]) \in SC_{\alpha}$ by induction hypothesis.

  Hence, $\Gamma \vdash {\bf pure \: } M : \textbf{K}\alpha$ and $({\bf pure \: } M ([x_1 := N_1, \dots, x_n := N_n])) \in SC_{\textbf{K}\alpha}$ by definition of $SC_{\textbf{K}\alpha}$.

  3) Let $\Gamma = \{ x_1 : \phi_1, \dots, x_n : \phi_n \}$, $\Gamma : \phi_n \vdash M \star P : \textbf{K}\beta$ and forall $i \in \{ 1, \dots, n\}, N_i \in SC_{\phi_i}$.

  Then $\Gamma \vdash M : \textbf{K}(\alpha \to \beta)$, $\Gamma \vdash P : \textbf{K}\alpha$ by generation.

  But by induction hypothesis $M [x_1 := N_1, \dots, x_n := N_n] \in SC_{\textbf{K}(\alpha \to
  \beta)}$ and $P [x_1 := N_1, \dots, x_n := N_n] \in SC_{\textbf{K}\alpha}$.

  Then, by definition of $SC_{\textbf{K}\beta}$, $((M [x_1 := N_1, \dots, x_n := N_n]) \star (P [x_1 := N_1, \dots, x_n := N_n])) \in SC_{\textbf{K}\beta}$, i.e. $(M \star N ([x_1 := N_1, \dots, x_n := N_n])) \in
  SC_{\textbf{K}\beta}$.
  \end{proof}

  \begin{col}

  $ $

  If $\vdash M : \alpha$, then $M$ is strongly normalizing.

  \end{col}

  \begin{proof}

  $M \in SC_{\alpha}$ by Lemma 10, so $M$ is strongly normalizing.

  \end{proof}


  \end{proof}

\section{Confluence}

In the confluence proof (below) we treat the cases with ${\bf pure}$ and $\star$ similar to \cite{Baren} \cite{Baren2}.

\begin{defin} Alphabet for the labelled terms:

variables: $x, y, z, x_1, y_1, z_1, ...$;

lambdas: $\lambda, \lambda_0, \lambda_1, \lambda_2, ...$;

constructors for an applicative functor: ${\bf pure \: }, \star$;

parentheses (,).

\end{defin}

\begin{defin} The set of labelled terms $\Lambda^{'}_{\textbf{K}}$ inductively defined as a set of words on the alphabet
described above:

1) $x \in \Lambda^{'}$;

2) If $M \in \Lambda^{'}_{\textbf{K}}$, then $(\lambda x. M) \in \Lambda^{'}_{\textbf{K}}$;

3) If $M, N \in \Lambda^{'}_{\textbf{K}}$, then $(MN) \in \Lambda^{'}_{\textbf{K}}$;

4) If $M \in \Lambda^{'}_{\textbf{K}}$, then ${\bf pure \: } M \in \Lambda^{'}_{\textbf{K}}$;

5) If $M, N \in \Lambda^{'}_{\textbf{K}}$, then $M \star N \in \Lambda^{'}_{\textbf{K}}$;

6) If $M, N \in \Lambda^{'}_{\textbf{K}}$, then for all $i \in \mathbb{N}$, $((\lambda_i x. M) N) \in
\Lambda^{'}_{\textbf{K}}$.

\end{defin}

\begin{defin} Erasing map

Erasing map is a map $|.| : \Lambda^{'}_{\textbf{K}} \to \Lambda_{\textbf{K}}$, such that:

1) $|x| = x$;

2) $|(\lambda x. M)| = \lambda x. |M|$;

3) $|(M N)| = |M| |N|$;

4) $|({\bf pure \: } M)| = {\bf pure \: } |M|$;

5) $|M \star N| = |M| \star |N|$;

6) $|((\lambda_i x. M) N)| = (\lambda x. |M|) |N|$

\end{defin}

\begin{ex}

$ $

$|{\bf pure \: } ((\lambda_i x. M) N) \star P| = {\bf pure \: } (\lambda x. |M|) |N|) \star |P|$

\end{ex}

\begin{defin} Substitution for $\Lambda^{'}_{\textbf{K}}$:

1) $x [x := N] = N$, $x [y := N] = x$;

2) $(M N) [x := N] = M[x := N] N [x := N]$;

3) $(\lambda x. M) [x := N] = \lambda x. M [x := N]$;

4) $({\bf pure \: } M) [x := P] = {\bf pure \: } \: (M [x := P])$;

5) $(M \star N) [x := P] = (M [x := P]) \star (N [x := P])$;

6) $(\lambda_i x. M) N [y := P] = (\lambda_i x. M [y := P]) (N [y := P])$.

\end{defin}

\begin{defin} One-step reduction $\rightarrow_{\beta^{'}}$ for $\Lambda^{'}_{\textbf{K}}$:

1) $(\lambda x. M) N \rightarrow_{\beta^{'}} M [x := N]$;

2) ${\bf pure \: } (\lambda x. x) \star M \rightarrow_{\beta^{'}} M$;

3) ${\bf pure \: } (\lambda f g x. f (g x)) \star M \star N \star P \rightarrow_{\beta^{'}} M \star (N \star
P)$;

4) $({\bf pure \: } M) \star ({\bf pure \: } N) \rightarrow_{\beta^{'}} {\bf pure \: } (M N)$;

5) $M \star ({\bf pure \: } N) \rightarrow_{\beta^{'}} {\bf pure \: } (\lambda f. f N) \star M$;

6) $(\lambda_i x. M) N \rightarrow_{\beta^{'}} M [x := N]$.


Multi-step reduction $\twoheadrightarrow_{\beta^{'}}$ is a reflexive-transitive closure of
$\rightarrow_{\beta^{'}}$.

\end{defin}

\begin{defin} Let us define a map $\phi : \Lambda^{'}_{\textbf{K}} \to \Lambda_{\textbf{K}}$ inductively as
follows:

1) $\phi(x) = x$;

2) $\phi(M N) = \phi(M)\phi(N)$;

3) $\phi(\lambda x. M) = \lambda x. \phi(M)$;

4) $\phi({\bf pure \: } M) = {\bf pure \: } (\phi(M))$;

5) $\phi(M \star N) = \phi(M) \star \phi(N)$;

6) $\phi((\lambda_i x. M) N) = \phi(M) [x := \phi(N)]$.

\end{defin}

\begin{ex}

$ $

$\phi({\bf pure \: } ((\lambda_i x. M) N) \star P) = {\bf pure \: } (\phi(M) [x := \phi(N)]) \star \phi(P)$

\end{ex}

\begin{lemma}

$ $

1) Let $M, N \in \Lambda^{'}_{\textbf{K}}$ and $|M| \twoheadrightarrow_{\beta} |N|$, then $M \twoheadrightarrow_{\beta^{'}} N$.

2) Let $M, N \in \Lambda^{'}_{\textbf{K}}$ and $M \twoheadrightarrow_{\beta^{'}} N$, then $|M| \twoheadrightarrow_{\beta} |N|$.

\end{lemma}

\begin{proof}

$ $

Induction on the generation of $\twoheadrightarrow_{\beta}$ ($\twoheadrightarrow_{\beta^{'}}$).

1) Let us consider homomorphism rule. The rest applicative reduction rules are considered similary.

Let $({\bf pure \: } M') \star ({\bf pure \: } N'), {\bf pure \: } (M' N') \in \Lambda^{'}_{\textbf{K}}$.

So $|({\bf pure \: } M') \star ({\bf pure \: } N')| = ({\bf pure \: } |M'|) \star ({\bf pure \: } |N'|)$ and
$|{\bf pure \: } (M' N')| = {\bf pure \: } (|M'| |N'|)$.

By reduction rule, $({\bf pure \: } |M'|) \star ({\bf pure \: } |N'|) \rightarrow_{\beta} {\bf pure \: }
(|M'||N'|)$.

But $({\bf pure \: } M') \star ({\bf pure \: } N') \rightarrow_{\beta^{'}} {\bf pure \: } (M' N')$ by
reduction rule for $\rightarrow_{\beta^{'}}$.

  \vspace{\baselineskip}

2) Let us consider interchange rule.

Let $M \star ({\bf pure \: } N), {\bf pure \: } (\lambda f. f N) \star M \in \Lambda^{'}_{\textbf{K}}$ and
$M \star ({\bf pure \: } N) \rightarrow_{\beta^{'}} {\bf pure \: } (\lambda f. f N) \star M$.

But $|M \star ({\bf pure \: } N)| = |M| \star ({\bf pure \: } |N|)$ and $|{\bf pure \: } (\lambda f. f N)
\star M| = {\bf pure \: } (\lambda f. f |N|) \star |M|$.

So $|M| \star ({\bf pure \: } |N|) \rightarrow_{\beta} {\bf pure \: } (\lambda f. f |N|) \star |M|$ by
$\beta$-reduction rule.

\vspace{\baselineskip}

It is easy to see, that the statement for $\twoheadrightarrow_{\beta^{'}}$ and $\twoheadrightarrow_{\beta}$
immedeatly follows from transitivity of multi-step rediction for labelled terms and for usual terms
respectively.

\end{proof}

\begin{lemma}

$ $

$\phi(M [x := N]) = \phi(M) [x := \phi(N)]$.

\end{lemma}

\begin{proof} Induction on $M$.

1) Let $M = x$. Then $\phi(x [x := N]) = \phi(N)$.

On the other hand, $\phi(x) [x := \phi(N)] = x [x := \phi(N)] = \phi(N)$.

So $\phi(x [x := N]) = \phi(x) [x := \phi(N)]$.

  \vspace{\baselineskip}

2) Let $M = y$ and $y \neq x$. Then $\phi(y [x := N]) = \phi(y) = y$.

But $\phi(y) [x := \phi(N)] = y [x := \phi(N)] = y$.

Therefore $\phi(y [x := N]) = \phi(y) [x := \phi(N)]$.

  \vspace{\baselineskip}

3) Let $M = {\bf pure \: } M'$. Then $\phi({\bf pure \: } M' [x := N]) = {\bf pure \: } \phi(M' [x := N])$.

By hypothesis, ${\bf pure \: } (\phi(M' [x := N])) = {\bf pure \: } (\phi(M') [x := \phi(N)])$, which is
$({\bf pure \: } \phi(M')) [x := \phi(N)]$ by substitution definition.

\vspace{\baselineskip}

4) Let $M = M' \star N'$. So $\phi((M' \star N') [x := N])) = \phi(M' [x := N] \star N' [x := N])$.

By definition of $\phi$,

\quad $\phi(M' [x := N] \star N' [x := N]) = \phi(M' [x := N]) \star \phi(N' [x := N])$.

But by induction hypothesis,

\quad $\phi(M' [x := N]) = \phi(M') [x := \phi(N)]$ and

\quad $\phi(N' [x := N]) = \phi(N') [x := \phi(N)]$.

Hence,

\quad $\phi(M' [x := N]) \star \phi(N' [x := N]) = \phi(M') [x := \phi(N)] \star \phi(N') [x := \phi(N)]$.

So,

\quad $\phi(M') [x := \phi(N)] \star \phi(N')[x := \phi(N)] = (\phi(M') \star \phi(N')) [x := \phi(N)]$.

And by definition of $\phi$, $(\phi(M') \star \phi(N')) [x := \phi(N)] = \phi(M' \star N') [x := \phi(N)]$.

\end{proof}

\begin{lemma}

$ $

  Let $M, N \in \Lambda^{'}_{\textbf{K}}$ and $M \twoheadrightarrow_{\beta^{'}} N$, then $\phi(M)
  \twoheadrightarrow_{\beta} \phi(N)$.

\end{lemma}

\begin{proof}

$ $

1) Let ${\bf pure \: } (\lambda x. x) \star M, M \in \Lambda^{'}_{\textbf{K}}$ and ${\bf pure \: } (\lambda
x. x) \star M \rightarrow_{\beta^{'}} M$.

But $\phi ({\bf pure \: } (\lambda x. x) \star M) = {\bf pure \: } (\lambda x. x) \star \phi(M)$.\

So ${\bf pure \: } (\lambda x. x) \star \phi(M) \rightarrow_{\beta} \phi(M)$ by $\beta$-reduction rule.

\vspace{\baselineskip}

2) Let ${\bf pure \: } (\lambda f g x. f (g x)) \star M \star N \star P, M \star (N \star P) \in
\Lambda^{'}_{\textbf{K}}$ and ${\bf pure \: } (\lambda f g x. f (g x)) \star M \star N \star P
\rightarrow_{\beta^{'}} M \star (N \star P)$.

By the definition of $\phi$:

$\phi({\bf pure \: } (\lambda f g x. f (g x)) \star M \star N \star P) = {\bf pure \: } (\lambda f g x. f (g
x)) \star \phi(M) \star \phi(N) \star \phi(P)$;

$M \star (N \star P) = \phi(M) \star (\phi(N) \star \phi(P))$.

Hence, ${\bf pure \: } (\lambda f g x. f (g x))
\star \phi(M) \star \phi(N) \star \phi(P) \rightarrow_{\beta} \phi(M) \star (\phi(N) \star \phi(P))$ by
$\beta$-reduction rule.

\vspace{\baselineskip}

3) Let $({\bf pure \: } M) \star ({\bf pure \: } N), {\bf pure \: } (M N) \in \Lambda^{'}_{\textbf{K}}$ and
$({\bf pure \: } M) \star ({\bf pure \: } N) \rightarrow_{\beta} {\bf pure \: } (M N)$.

By the definition of $\phi$:

$\phi(({\bf pure \: } M) \star ({\bf pure \: } N)) = ({\bf pure \: } \phi(M)) \star ({\bf pure \: }
\phi(N))$;

$\phi({\bf pure \: } (M N)) = {\bf pure \: } (\phi(M) \phi(N))$.

So, by reduction rule, $({\bf pure \: } \phi(M)) \star ({\bf pure \: } \phi(N)) \rightarrow_{\beta} {\bf
pure \: } (\phi(M) \phi(N))$.

\vspace{\baselineskip}

4) Let $M \star ({\bf pure \: }), {\bf pure \: } (\lambda f. f N) \star M$ and $M \star ({\bf pure \: } N)
\rightarrow_{\beta^{'}} (\lambda f. f N) \star M$.

$\phi(M \star ({\bf pure \: } N)) = \phi(M) \star ({\bf pure \: } \phi(N))$

$\phi((\lambda f. f N) \star M) = (\lambda f. f \phi(N)) \star \phi(M)$.

So, $\phi(M) \star ({\bf pure \: } \phi(N)) \rightarrow_{\beta} {\bf pure \: } (\lambda f. f \phi(N)) \star
\phi(M)$.
\end{proof}

\begin{lemma}

$ $

Let $M \in \Lambda^{'}_{\textbf{K}}$. Then $|M| \twoheadrightarrow_{\beta} \phi(M)$.

\end{lemma}

\begin{proof}

Induction on the structure of $M$.

\end{proof}

\begin{lemma} Strip lemma.

$ $

If $M \rightarrow_{\beta} N$ and $M \twoheadrightarrow_{\beta} P$. Then there exists some term $Q$, such that
$N \twoheadrightarrow_{\beta} Q$ and $P \twoheadrightarrow_{\beta} Q$.

\end{lemma}

\begin{proof}

  Proof is similar to \cite{Baren} \cite{Baren2}. We build the following diagram

\vspace{\baselineskip}

\xymatrix{
&&& M \ar[dd]_{\beta} \ar@{->>}[rrr] &&& P \ar@{-->>}[dd]^{\beta} \\
&&&& N' \ar[ul]^{|.|} \ar[dl]_{\phi} \ar@{-->>}[rrr]^{\beta^{'}} &&& P' \ar[ul]^{|.|} \ar[dl]_{\phi} \\
&&& N \ar@{-->>}[rrr]_{\beta} &&& Q
}

\end{proof}

which is commutes by lemmas 11 -- 14.

\begin{theorem} Confluence.

$ $

If $M \twoheadrightarrow_{\beta} N$ and $M \twoheadrightarrow_{\beta} P$. Then there exists some term $Q$,
such that $N \twoheadrightarrow_{\beta} Q$ and $P \twoheadrightarrow_{\beta} Q$.

\end{theorem}

\begin{proof}

$ $

  By unfolding $M \twoheadrightarrow_{\beta} N$ as the sequence of one-step reductions $M \rightarrow_{\beta}
  M_1 \rightarrow_{\beta} ... \rightarrow_{\beta} M_n \rightarrow_{\beta} N$ and applying strip lemma on every step.

\end{proof}

  \section{Acknowledgement.}

  Author would like to thank his supervisior
  V.L.Vasukov, V.N. Krupski for general idea and wise
  advice, V. de Paiva, V.I. Shalack, A.V. Rodin and M.
  Taldykin for discussing, critics and consulting.

  \addcontentsline{toc}{section}{References}

  %далее сам список используевой литературы
  \begin{thebibliography}{}

  \bibitem{Artemov} Artemov S. and Protopopescu T., \/ ``Intuitionistic Epistemic Logic'', \textit{The
  Review of Symbolic Logic}, 2016, vol. 9, no 2. pp. 266-298.\parskip=1mm

  \bibitem{Krupski} Krupski V. N. and Yatmanov A., \/ ``Sequent Calculus for Intuitionistic Epistemic Logic
  IEL'', \textit{Logical Foundations of Computer Science: International Symposium, LFCS 2016, Deerfield
  Beach, FL, USA, January 4-7, 2016. Proceedings}, 2016, pp. 187-201.\parskip=1mm

  \bibitem{Haskell} Haskell Language. // URL: https://www.haskell.org. (Date: 1.08.2017) \parskip=1mm

  \bibitem{Idris} Idris. A Language with Dependent Types.// URL:https://www.idris-lang.org. (Date:
  1.08.2017) \parskip=1mm

  \bibitem{Purs} Purescript. A strongly-typed functional programming language that compiles to JavaScript.
  URL: http://www.purescript.org. (Date: 1.08.2017) \parskip=1mm

  \bibitem{Elm} Elm. A delightful language for reliable webapps. // URL: http://elm-lang.org. (Date:
  1.08.2017) \parskip=1mm

  \bibitem{Base} Hackage, \/ ``The base package'' // URL: https://hackage.haskell.org/package/base-4.10.0.0
  (Date: 1.08.2017) \parskip=1mm

  \bibitem{Miran} Lipovaca M, \/ ``Learn you a Haskell for Great Good!''. //URL:
  http://learnyouahaskell.com/chapters (Date: 1.08.2017) \parskip=1mm

  \bibitem{McP} McBride C. and Paterson R., ``Applicative programming with effects", \textit{Journal of
  Functional Programming}, 2008, vol. 18, no 01. pp 1-13. \parskip=1mm

  \bibitem{McP2} McBride C. and Paterson R, ``Functional Pearl. Idioms: applicative programming with
  effects'', \textit{Journal of Functional Programming}, 2005. vol. 18, no 01. pp 1-20. \parskip=1mm

  \bibitem{Neder} R. Nederpelt and H. Geuvers, ``Type Theory and Formal Proof: An Introduction''.
  \textit{Cambridge University Press}, New York, NY, USA, 2014. pp. 436. \parskip=1mm

  \bibitem{Morten} Sorensen M. H. and Urzyczyn P, ``Lectures on the Curry-Howard isomorphism'',
  \textit{Studies in Logic and the Foundations of Mathematics}, vol. 149, \textit{Elsevier Science}, 1998.
  pp 261. \parskip=1mm

  \bibitem{Pierce} Pierce B. C., ``Types and Programming Languages''. \textit{Cambridge, Mass: The MIT
  Press}, 2002. pp. 605. \parskip=1mm

  \bibitem{Girard} Girard J.-Y., Taylor P. and  Lafont Y, ``Proofs and Types'', \textit{Cambridge University
  Press}, New York, NY, USA, 1989. pp. 175. \parskip=1mm

  \bibitem{Baren} Barendregt. H. P., ``Lambda calculi with types" // Abramsky S., Gabbay Dov M., and S. E.
  Maibaum, ``Handbook of logic in computer science (vol. 2), Osborne Handbooks Of Logic In Computer
  Science'', Vol. 2. \textit{Oxford University Press, Inc.}, New York, NY, USA, 1993. pp 117-309.
  \parskip=1mm

  \bibitem{Hindley} Hindley J. Roger, ``Basic Simple Type Theory''. \textit{Cambridge University Press}, New
  York, NY, USA, 1997. pp. 185. \parskip=1mm

  \bibitem{Lax} Pfenning F. and Davies R., ``A judgmental reconstruction of modal logic'',
  \textit{Mathematical Structures in Computer Science}, vol. 11, no 4, 2001, pp. 511-540. \parskip=1mm

  \bibitem{Baren2} H.P. Barendregt. The Lambda Calculus --- Its Syntax and Semantics. Studies in Logic and
  the Foundations of Mathematics, vol. 103. Amsterdam: North-Holland, 1985.

  \bibitem{ModalK} Yoshihiko KAKUTANI, A Curry-Howard Correspondence for Intuitionistic Normal Modal Logic, Computer Software, Released February 29, 2008, Online ISSN , Print ISSN 0289-6540.
  \end{thebibliography}


\end{document}
